\documentclass{article}
\usepackage{layout}
\usepackage[a4paper, total={5in,9in}]{geometry}
\usepackage[T1]{fontenc}
\usepackage[italian]{babel}
\usepackage{mathtools}
\usepackage{amsthm}
\usepackage[framemethod=TikZ]{mdframed}
\usepackage{amsmath}
\usepackage{amssymb}
\usepackage{cancel}
\usepackage{xcolor}
\usepackage{tikz}
\usepackage{tikz-cd}
\usepackage{pgfplots}
\pgfplotsset{compat=1.18}
\usepackage[many]{tcolorbox}
\usepackage{import}
\usepackage{pdfpages}
\usepackage{transparent}
\usepackage{xcolor}
\usepackage{enumitem}
\usepackage[colorlinks]{hyperref}

\newcommand*{\sminus}{\raisebox{1.3pt}{$\smallsetminus$}}

\newcommand*{\transp}[2][-3mu]{\ensuremath{\mskip1mu\prescript{\smash{\mathrm t\mkern#1}}{}{\mathstrut#2}}}%

% newcommand for span with langle and rangle around
\newcommand{\Span}[1]{{\left\langle#1\right\rangle}}

\newcommand{\incfig}[2][1]{%
    \def\svgwidth{#1\columnwidth}
    \import{./figures/}{#2.pdf_tex}
}

\pdfsuppresswarningpagegroup=1

\newcounter{theo}[section]\setcounter{theo}{0}
\renewcommand{\thetheo}{\arabic{section}.\arabic{theo}}

\newcounter{excounter}[section]\setcounter{excounter}{0}
\renewcommand{\theexcounter}{\arabic{section}.\arabic{excounter}}

\numberwithin{equation}{section}

\newenvironment{theorem}[1][]{
    \refstepcounter{theo}
     \ifstrempty{#1}
    {\mdfsetup{
        frametitle={
            \tikz[baseline=(current bounding box.east),outer sep=0pt]
            \node[anchor=east,rectangle,fill=blue!20,rounded corners=5pt]
            {\strut Teorema~\thetheo};}
        }
    }{\mdfsetup{
        frametitle={
            \tikz[baseline=(current bounding box.east),outer sep=0pt]
            \node[anchor=east,rectangle,fill=blue!20,rounded corners=5pt]
            {\strut Teorema~\thetheo:~#1};}
        }
    }
    \mdfsetup{
        roundcorner=10pt,
        innertopmargin=10pt,linecolor=blue!20,
        linewidth=2pt,topline=true,
        frametitleaboveskip=\dimexpr-\ht\strutbox\relax,
        % nobreak=false
    }
\begin{mdframed}[]\relax}{
\end{mdframed}}

% \newenvironment{definition}[1][]{
%     \refstepcounter{theo}
%      \ifstrempty{#1}
%     {\mdfsetup{
%         frametitle={
%             \tikz[baseline=(current bounding box.east),outer sep=0pt]
%             \node[anchor=east,rectangle,fill=violet!20,rounded corners=5pt]
%             {\strut Definizione~\thetheo};}
%         }
%     }{\mdfsetup{
%         frametitle={
%             \tikz[baseline=(current bounding box.east),outer sep=0pt]
%             \node[anchor=east,rectangle,fill=violet!20,rounded corners=5pt]
%             {\strut Definizione~\thetheo:~#1};}
%         }
%     }
%     \mdfsetup{
%         roundcorner=10pt,
%         innertopmargin=10pt,linecolor=violet!20,
%         linewidth=2pt,topline=true,
%         frametitleaboveskip=\dimexpr-\ht\strutbox\relax,
%         nobreak=true
%     }
% \begin{mdframed}[]\relax}{
% \end{mdframed}}

\newtcolorbox[auto counter, number within=section]{definition}[2][]{
    colframe=violet!0,
    coltitle=violet, % Title text color
    fonttitle=\bfseries, % Title font
    title={Definizione~\thetcbcounter:~#2}, % Title format
    sharp corners, % Less rounded corners
    boxrule=0pt, % Line width of the box frame
    toptitle=1mm, % Distance from top to title
    bottomtitle=1mm, % Distance from title to box content
    colbacktitle=violet!5, % Background color of the title bar
    left=0mm, right=0mm, top=1mm, bottom=1mm, % Padding around content
    enhanced, % Enable advanced options
    before skip=10pt, % Space before the box
    after skip=10pt, % Space after the box
    breakable, % Allow box to split across pages
    colback=violet!0,
    borderline west={2pt}{-5pt}{violet!40},
    #1
}

\newenvironment{lemmao}[1][]{
    \refstepcounter{theo}
     \ifstrempty{#1}
    {\mdfsetup{
        frametitle={
            \tikz[baseline=(current bounding box.east),outer sep=0pt]
            \node[anchor=east,rectangle,fill=green!20,rounded corners=5pt]
            {\strut Lemma~\thetheo};}
        }
    }{\mdfsetup{
        frametitle={
            \tikz[baseline=(current bounding box.east),outer sep=0pt]
            \node[anchor=east,rectangle,fill=green!20,rounded corners=5pt]
            {\strut Lemma~\thetheo:~#1};}
        }
    }
    \mdfsetup{
        roundcorner=10pt,
        innertopmargin=10pt,linecolor=green!20,
        linewidth=2pt,topline=true,
        frametitleaboveskip=\dimexpr-\ht\strutbox\relax,
        % nobreak=true
    }
\begin{mdframed}[]\relax}{
\end{mdframed}}

\theoremstyle{plain}
\newtheorem{lemma}[theo]{Lemma}
\newtheorem{corollary}{Corollario}[theo]
\newtheorem{proposition}[theo]{Proposizione}

\theoremstyle{definition}
\newtheorem{example}[excounter]{Esempio}

\theoremstyle{remark}
\newtheorem*{note}{Nota}
\newtheorem*{remark}{Osservazione}

\newtcolorbox{notebox}{
  colback=gray!10,
  colframe=black,
  arc=5pt,
  boxrule=1pt,
  left=15pt,
  right=15pt,
  top=15pt,
  bottom=15pt,
}

\DeclareRobustCommand{\rchi}{{\mathpalette\irchi\relax}} % beautiful chi
\newcommand{\irchi}[2]{\raisebox{\depth}{$#1\chi$}} % inner command, used by \rchi

% \newtcolorbox[auto counter, number within=section]{eser}[1][]{
%     title=Esercizio~\thetcbcounter,
%     label={eser:\thetcbcounter},
% 	% skin=bicolor,
% 	breakable,
% 	enhanced,
% 	frame hidden,
% 	boxrule = 0sp,
% 	borderline west = {1.5pt}{-5pt}{black},
% 	sharp corners,
% 	detach title,
% 	before upper = \tcbtitle\par\smallskip,
% 	coltitle = black,
% 	fonttitle = \bfseries\sffamily,
% 	description font = \mdseries,
% 	separator sign none,
% 	colback = black!0,
%     colbacktitle=black!5
% }
\newtcolorbox[auto counter, number within=section]{eser}[1][]{
    colframe=black!0,
    coltitle=black!70, % Title text color
    fonttitle=\bfseries\sffamily, % Title font
    title={Esercizio~\thetcbcounter~#1}, % Title format
    sharp corners, % Less rounded corners
    boxrule=0pt, % Line width of the box frame
    toptitle=1mm, % Distance from top to title
    bottomtitle=1mm, % Distance from title to box content
    colbacktitle=black!5, % Background color of the title bar
    left=0mm, right=0mm, top=1mm, bottom=1mm, % Padding around content
    enhanced, % Enable advanced options
    before skip=10pt, % Space before the box
    after skip=10pt, % Space after the box
    breakable, % Allow box to split across pages
    colback=black!0,
    borderline west={1pt}{-5pt}{black!70},
}
\newcommand{\seminorm}[1]{\left\lvert\hspace{-1 pt}\left\lvert\hspace{-1 pt}\left\lvert#1\right\lvert\hspace{-1 pt}\right\lvert\hspace{-1 pt}\right\lvert}

\title{Appunti di Analisi 3 \-- Equazioni Differenziali}
\author{Osea}
\date{Primo semestre A.A. 2024 \-- 2025, prof. Enrico Vitali}
\begin{document}

\maketitle
Libro suggerito: \emph{Hirsch, Smale, Devaney}


\setcounter{section}{-1}
\section{Esempi}
Sia abbia una ``popolazione'' la quantità $x(t)$ evolva nel tempo. Allora il
rapporto \(x'(t) / x(t)\) è il tasso di variazione della popolazione.
L'equazione
\[
    \frac{x'(t)}{x(t)} = r(t, x(t))
\]
è un'equazione differenziale. Tipicamente si considera il problema ai valori
iniziali (di Cauchy) dove è aggiunta anche una condizione \(x(0) = x_{0}\) 

\begin{example}[Malthus]
    Se \(r\) è costante abbiamo la cosiddetta legge di Malthus.
    \begin{align*}
        \frac{x'(t)}{x(t)} &= r \quad x(t) > 0 \\
        \log x(t) &= rt + c \quad c \in \mathbb{R} \\
        x(t) &= e^{c} \cdot e^{rt}
    \end{align*}
    dove se \(x(0) = e^{c} = x_{0}\) si ottiene \(x(t) = x_{0} e^{rt}\) 

    Un esempio dove questo si potrebbe verificare è il decadimento radioattivo.
\end{example}

\begin{example}[Logistica]
    Non si può invece applicare in una situazione realistica di uno studio di
    una popolazione l'esempio della crescita esponenziale, non essendo
    ragionevole in quanto non considera la capacità di carico ambientale. La si
    introduce quindi. Allora possiamo modellizzare il fatto con:
    \[
        \frac{x'(t)}{x(t)} = r\left(1 - \frac{x(t)}{K}\right)
    \]
    Da cui si ottiene dopo qualche calcolo
    \[
        x(t) = k \frac{Ce^{rt}}{1 + Ce^{rt}}
    \]
    con \(C \in \mathbb{R}\), in particolare scriviamo \(C = e^{-rt_{0}}\) con
    \(t_{0}\) definito opportunamente. Allora le soluzioni sono 
    \[
        x(t) = k \frac{e^{r(t-t_{0})}}{1 + e^{r(t-t_{0})}}
    \]
    Questo rende evidente che i grafici sono tutti uno traslato temporale
    dell'altro. Si può quindi studiare solo una soluzione, ad esempio per
    \(t_{0}=0\) essendo le altre semplici traslazioni temporali di questa.
    \[
        x(t) = k \frac{e^{rt}}{1 + e^{rt}} = \frac{k}{1 + e^{-rt}}
    \]
\begin{figure}[ht]
    \centering
    \incfig{logistica}
    \caption{logistica}\label{fig:logistica}
\end{figure}
\end{example}

Ciò che abbiamo ottenuto riguardo alle soluzioni, dove ogni soluzione è una
traslazione temporale di un'altra, è un caso generale delle equazioni
differenziali non dipendenti dal tempo. Il che è una facile osservazione del
fatto che se \(x'(t) = f(x(t))\) e \(x_\tau(t) := x(t - \tau)\) allora anche
\(x_\tau\) è soluzione poiché \(x_\tau'(t) = x'(t-\tau) = f(x(t-\tau)) =
f(x_\tau(x))\).

\begin{example}
    Le equazioni precedenti modellizzano anche fenomeni differenti. Ad esempio in
    chimica se considero la reazione chimica \(H_{2}+I_{2} \to 2HI\) posso
    considerare la velocità di cambiamento della concentrazione nella soluzione
    della reazione e quindi in questo caso ottenere
    \[
        -\frac{1}{2} \frac{d[HI]}{dt} = \frac{d[H_{2}]}{dt} = \frac{d[I_{2}]}{dt}
    \]
    dove la prima derivata è definita essere la ``velocità'' \(v(t)\) della
    reazione. La legge cinetica dice che \(v\) è proporzionale al prodotto delle
    concentrazioni di \(H_{2}\) e \(I_{2}\) quindi 
    \begin{align*}
        v(t) =  k {[H_{2}]}_t {[I_{2}]}_t = \frac{d{[HI]}_t}{dt} \\
    \end{align*}
    E il resto è lasciato come esercizio
    con l'obiettivo di mostrare che la reazione ha uno sviluppo logistico.

La reazione inversa invece si modellizza con \(x'(t) = -r{x(t)}^2\). Proviamo a
risolverla e otteniamo che
\begin{align*}
    x' = rx^2 \implies \int \frac{1}{x^2} dx = rt + c \\
    -\frac{1}{x(t)} = rt + c \implies x(t) = -\frac{1}{rt + c}
\end{align*}
Se poniamo \(x(0) = x_{0}\) otteniamo \(x_{0} = -\frac{1}{c}\) da cui \(c =
-\frac{1}{x_{0}}\) quindi 
\[
    x(t) = -\frac{1}{rt - \frac{1}{x_{0}}}
\]
\begin{figure}[ht]
    \centering
    \incfig{xpisxsr}
\caption{\(x'(t) = rx{(t)}^2\) per \(r = 1\) }\label{fig:xpisxsr}
\end{figure}
\end{example}

\begin{example}[Oscillatore armonico]
\begin{figure}[ht]
    \centering
    \incfig{oscillatore_armonico}
    \caption{oscillatore armonico}
    \label{fig:oscillatore_armonico}
\end{figure}
    Sia \(m\) una massa vincolata in modo elastico ad una posizione fissa. Sia
    \(x=0\) l'ascissa quando la molla è a riposo. Se la massa è nella posizione
    di ascissa \(x(t)\) al tempo \(t\) allora subisca la forza di richiamo
    (Hooke) \(F = -kx(t)\). Allora dalla seconda legge della dinamica
    \[
        m \ddot{x} = -kx^2 \iff \ddot{x}(t) + \omega^2 x(t) = 0, \quad \omega^2
        = \frac{k}{m}
    \]
    È un'equazione differenziale del secondo ordine lineare a coefficienti
    costanti, omogenea. Allora le soluzioni sono della forma
    \[
        x(t) = c_{1}\cos(\omega t) + c_{2}\sin(\omega t), \quad c_{1}, c_{2} \in
        \mathbb{R}
    \]
    Ho due costanti poiché devo definire posizione e velocità iniziale, e
    infatti il problema tipico ai valori iniziali per un'equazione del secondo
    ordine è 
    \[
        \begin{cases}
            \ddot{x} + \omega^2x = 0 \\
            x(0) = x_{0} \\
            \dot{x}(x) = v_{0}
        \end{cases}
    \]
    Posso riscrivere la precedente soluzione nella forma 
    \[
        x(t) =
        \sqrt{c_{1}^2+c_{2}^2}(\frac{c_{1}}{\sqrt{c_{1}^2+c_{2}^2}}\cos(\omega
        t) + \frac{c_{2}}{\sqrt{c_{1}^2+c_{2}^2}}\sin(\omega t))
    \]
    dove i coefficienti di seno e coseno sono le coordinate di un punto sulla
    circonferenza unitaria, quindi sono coseno e seno di un angolo \(\varphi\),
    da cui otteniamo che \(x(t) = A\cos(\omega t + \varphi)\) 
\end{example}
\begin{example}[Varianti dell'oscillatore armonico]
    In presenza di attrito viscoso le precedenti diventano
    \begin{align*}
        m\ddot{x} = -kx - h\dot{x} \\
        \ddot{x} + \gamma \dot{x} + \omega^2x = 0
    \end{align*}
    E se aggiungiamo anche una forza esterna costante 
    \begin{align*}
        m\ddot{x} = -kx - h\dot{x} + \psi \\
        \ddot{x} + \gamma \dot{x} + \omega^2x = f
    \end{align*}
    che è non omogenea
\end{example}
\begin{example}[Corrente elettrica]
    Preso un circuito \(RC\) abbiamo una forza elettromotrice che crea una
    differenza di potenziale \(\mathcal{E}\), un condensatore
    di capacità \(C\) e una resistenza \(R\). Sia \(q(t)\) la carica sulle
    piastre del condensatore. Allora risulta che in ogni momento
    \(\frac{q(t)}{V} = C\) costante. La resistenza invece per la legge di ohm
    \(V = i(t)R\), allora otteniamo una legge
    \[
       \mathcal{E} = Ri + \frac{q}{C} = Rq' + \frac{q}{C} 
    \]
    che è una equazione lineare del primo ordine.

    In un circuito reale invece è presente un terzo termine legato
    all'induttanza del filo, creando un circuito \(RLC\), consideriamo quindi di
    aggiungere al circuito precedentemente analizzato un'induttanza \(L\).
    Allora abbiamo
    \[
        \mathcal{E} - \mathcal{E}_L = Ri + \frac{q}{C}
    \]
    dove \(\mathcal{E}_L = Li'\) è l'inerzia elettrica del sistema, l'equazione
    diventa
    \[
        L\ddot{q} + R\dot{q} + \frac{q}{C} = \mathcal{E}
    \]
    Ossia un'ODE lineare a coefficienti costanti del secondo ordine non omogenea.
\end{example}

\begin{example}[Equazione di Schrödinger]
    \[
        -\frac{\hbar}{2m} \nabla^2_x \psi(x, t) + U(x, t)\psi(x, t) = i
        \frac{\partial \psi}{\partial t}
    \]
    Supponiamo che il potenziale \(U\) sia indipendente dal tempo \(t\), ossia
    \(U = U(x)\) e cerchiamo soluzioni della forma a variabili separate, cioè
    \(\psi(x, t) = u(x) \varphi(t)\), inoltre poniamoci in una sola dimensione,
    otteniamo:
    \[
        -\frac{\hbar}{2m} u''(x)\varphi(t) + U(x)u(x) \varphi(t) =
        iu(x)\varphi'(t)
    \]
    Da cui dividendo per \(u(x)\varphi(t)\) 
    \[
        -\frac{\hbar}{2m}\frac{u''(x)}{u(x)} + U(x) =
        i\frac{\varphi'(t)}{\varphi(t)}
    \]
    E poiché abbiamo che un'equazione in \(x\) è uguale ad una in \(t\) deve
    essere che entrambi i membri sono costanti, diciamo \(E\).
    \[
        \begin{cases}
            \frac{\hbar^2}{2m}u''(x) = (U(x) - E)u(x) \\
            \frac{\varphi'(t)}{\varphi(t)} = -iE
        \end{cases}
    \]
    La prima è un'equazione differenziale lineare del secondo ordine a
    coefficienti non costanti (in generale) e si può riscrivere come 
    \[
        Lu := -\frac{\hbar^2}{2m} u''(x) + U(x)u(x)
    \]
    è un operatore lineare su \(u\) e allora l'equazione diventa
    \[
        Lu = Eu
    \]
    ossia un'equazione agli autovalori per l'operatore \(L\).

    Ad esempio nel caso \(U(x) = \frac{1}{2}k x^2\) e con opportuni cambi di
    variabile si arriva alla cosiddetta \emph{equazione di Hermite}
    \[
        H''(\xi) - 2\xi H' + \cos t H = 0
    \]
    che dà luogo ai cosiddetti \emph{polinomi di Hermite}
\end{example}
Altri due esempi importanti sono, nel caso dei sistemi (tra cui l'esempio famoso
del preda-predatore di Lotka-Volterra) e le elastiche piane.
Ulteriore esempio è il moto di un pendolo, che si può modellizzare con un'ODE di
secondo ordine non lineare.
\begin{example}[Elastiche piane]
    Sia abbia una verga elastica (barra) che supponiamo in una posizione di
    equilibrio della forma grafico \(y = u(x)\) 
\begin{figure}[ht]
    \centering
    \incfig{elastica_piana}
    \caption{Elastica Piana}\label{fig:elastica_piana}
\end{figure}
    Con ipotesi modellistica, ossia \textbf{il momento flettente è proporzionale
    alla curvatura}, dove la curvatura in un punto \(x\) si può calcolare con
    \[
        \kappa(x) = \frac{u''(x)}{{(1 + {u'(x)}^2)}^{3/2}}
    \]
    e il momento flettente si può immaginare intuitivamente come la forza locale
    sulla verga ed è proporzionale al valore di \(u(x)\). Allora si ottiene
    l'equazione
    \[
        \kappa(x) = -\lambda u(x) \iff u''(x) + \lambda u(x){(1 +
        {u'(x)}^2)}^{3/2} = 0
    \]
    dove \(\lambda \in \mathbb{R}\) è una costante.
\begin{remark}
    La costante \(\lambda\) e la lunghezza della verga sono legate (significato
    modellistico)
\end{remark}
\end{example} 
\begin{example}[Equazioni di Lotka-Volterra]
    Siano \(x_{1}(t)\) e \(x_{2}(t)\) le popolazioni di due specie che interagiscono
    tra loro. Allora si può modellizzare il fatto che la popolazione di una
    specie è proporzionale alla sua popolazione e alla popolazione dell'altra
    specie. Sia allora \(\frac{x_{i}'(t)}{x_{i}(t)}\) il tasso di variazione
    della popolazione \(x_{i}\). Supponiamo adesso che \(x_{1}\) sia la
    ``preda'' e \(x_{2}\) il ``predatore''. Allora si può modellizzare
    l'evoluzione del sistema esprimendo il tasso di variazione delle
    popolazioni. In particolare per la preda abbiamo
    \[
        \frac{x_{1}'(t)}{x_{1}(t)} = \alpha - \beta x_{2}(t)
    \]
    dove \(\alpha\) è il tasso in assenza di predatori e \(b\) è il tasso di
    morte causato dai predatori. Per il predatore invece si ha
    \[
        \frac{x_{2}'(t)}{x_{2}(t)} = \delta x_{1}(t) - \gamma
    \]
    In conclusione, il sistema di equazioni differenziali è
    \[
        \begin{cases}
            x_{1}'(t) &= (\alpha - \beta x_{2}(t))\,x_{1}(t) \\
            x_{2}'(t) &= (\delta x_{1}(t) - \gamma)\,x_{2}(t)
        \end{cases}
    \]
    che è un sistema non lineare di due equazioni differenziali. Il problema ai
    valori iniziali naturalmente associato è quello che si ottiene aggiungendo i
    dati \(x_{1}(t_{0}) = x_{1}^{0}\) e \(x_{2}(t_{0}) = x_{2}^{0}\).

    Anche questo sistema può essere trasformato ad una versione logistica,
    cambiando i fattori \(\alpha \) in \(\alpha(1 - \frac{x_{1}}{k_{1}})\) e
    analogamente per \(x_{2}\), ottenendo il sistema
    \[
        \begin{cases}
            x_{1}' = x_{1}(\alpha - \lambda x_{1} - \beta x_{2}) \\
            x_{2}' = x_{2} (-\gamma + \delta x_{1} - \mu x_{2})
        \end{cases}
    \]
\end{example}

\section{Equazioni differenziali ordinarie}
In generale quindi abbiamo diversi esempi (\(x' = rx\), \(x' = rx(1-x)\), \(Rq'
+ \frac{1}{C} q = \mathcal{E}\)) di equazioni della forma \(x' = f(t, x)\).
Anche in Lotka-Volterra abbiamo una forma simile ma con due variabili, infatti
considerando \(\mathbf{x}(t) = (x_{1}(t), x_{2}(t))\) abbiamo che possiamo
riscrivere il sistema come un sistema vettoriale della forma \(\mathbf{x}'(t) =
\mathbf{f}(t, \mathbf{x}(t))\).
\begin{definition}{soluzione di un'equazione differenziale}
    Sia \(D \subseteq \mathbb{R} \times \mathbb{R}^{n} \), con \(n \ge 1\)
    aperto. Sia \(f : D \to \mathbb{R}^{n}\) continua. Consideriamo l'equazione
    differenziale \(x'(t) = f(t, x(t))\), che è detta quindi in \emph{forma
    normale}.

    Diciamo \textbf{soluzione} dell'equazione differenziale ogni funzione \(x :
    J \to \mathbb{R}^{n}\) con \(J\) intervallo, di classe \(C^{1}\) e tale che 
    \[
        \forall t \in J, (t, x(t)) \in D
    \]
    e su \(D\) è verificata l'uguaglianza 
    \[
        x'(t) = f(t, x(t))
    \]
\end{definition}
Nel caso scalare (ossia \(n=1\)) dunque abbiamo che \(x(\cdot )\) è soluzione se ogni punto
\((t, x(t))\) del grafico ha pendenza pari al valore di \(f(t, x(t))\).
\begin{figure}[ht]
    \centering
    \incfig{pendenzagrafico}
    \caption{Soluzione di un'equazione differenziale nel caso \(n=1\) }\label{fig:pendenzagrafico}
\end{figure}

Nel caso invece di equazioni differenziali del secondo ordine (come il moto
armonico, il pendolo, l'elastica piana, il circuito RLC ecc\dots) si ha che 
\[
    x''(t) = f(t, x(t), x'(t))
\]
con \(x(\cdot )\) incognita. Questa forma è detta \emph{forma normale} per
un'equazione del secondo ordine.
In generale se \(\mathbf{x}(t) = (x_{1}(t), x_{2}(t), x_{3}(t)) \) è un punto
mobile in \(\mathbb{R}^{3}\) sotto l'azione di un campo di forza \(\mathbf{F}(t,
\mathbf{x}(t), \mathbf{x}'(t))\) allora per la seconda legge della dinamica si
ottiene il sistema del secondo ordine
\[
    m\mathbf{x}''(t) = \mathbf{F}(t, \mathbf{x}(t), \mathbf{x}'(t)) 
\]
Un'equazione del secondo ordine in forma normale può essere ricondotta a un
sistema del primo ordine in forma normale, come
\[
    x''(t) = f(t, x(t), x'(t)) \iff
    \begin{cases}
        x'(t) &= v(t) \\
        v'(t) &= f(t, x(t), v(t))
    \end{cases}
\]
Dove abbiamo introdotto la variabile \(v(t) = x'(t)\). A questo punto il sistema
può essere scritto in forma vettoriale come \(\mathbf{x}(t) = (x(t), v(t))\), e
quindi \(\mathbf{x}'(t) = \mathbf{f}(t, \mathbf{x}) = (v(t), f(t, x(t), v(t)))\) 

Analogalmente equazioni di ordine superiore possono essere ricondotte a sistemi
di primo ordine introducendo le variabili \(x_{i} = x^{(i)}\) dove \(x^{(i)}\) è
la \(i\)-esima derivata di \(x\). Il sistema che analizzeremo sarà quindi
\[
    \begin{cases}
        x_{1}'(t) &= x_{2}(t) \\
        x_{2}'(t) &= x_{3}(t) \\
        &\vdots \\
        x_{n}'(t) &= f(t, x_{1}(t), x_{2}(t), \ldots, x_{n}(t))
    \end{cases}
\]
dove \(n\) è il grado dell'equazione differenziale.

Il problema ai valori iniziali associato al sistema del primo ordine
\(\mathbf{x}'(t) = \mathbf{f}(t, \mathbf{x}(t))\) si ha aggiungendo il requisito
\(\mathbf{x}(0) = \mathbf{y}_0\), per cui ovviamente nel caso di un'equazione di
ordine superiore il problema ai valori iniziali è dato dal richiedere, per ogni
derivata \(x^{(i)}\), il valore iniziale \(x^{(i)}(0) = y_{i}\).

È particolare il caso in cui \(f\) non dipende da \(t\) e viene detto
\textbf{caso autonomo} allora se \(\Omega \subseteq \mathbb{R}^{n} \) è aperto e
\(f: \Omega \to \mathbb{R}^{n}\) è continua allora l'equazione è \(x' = f(x)\)
dove la soluzione è una funzione \(x : J \to \Omega\) con \(J\) intervallo.
\begin{example}[Logistica]
    L'equazione, come visto in precedenza è 
    \[
        x' = rx(1- \frac{x}{k}) = f(x)
    \]
    Allora il campo di pendenza non dipende da \(t\), vedasi
    figura~\ref{fig:pendenza_logistica}
\begin{figure}[ht]
    \centering
    \incfig{pendenza_logistica}
    \caption{Pendenza della logistica}\label{fig:pendenza_logistica}
\end{figure}

Allora abbiamo un'invarianza delle soluzioni per traslazione temporali, infatti
se \(x\) è soluzione e \(x_\tau(t) := x(t - \tau)\) allora \(x_\tau'(t) =
f(x(t-\tau)) = f(x_\tau(t))\).
In tal modo possiamo creare uno ``Spazio degli stati'' ossia un'immagine in cui
raffiguriamo soltanto le posizioni di \(x\), come si evolve il sistema in quei
punti, poiché rappresentare \(t\) è ridondante, otteniamo per la logistica il
seguente schema:

\begin{figure}[ht]
    \centering
    \incfig{logistica_stati}
    \caption{Spazio degli stati dell'equazione logistica}\label{fig:logistica_stati}
\end{figure}
\end{example}

Possiamo espandere l'idea dello spazio degli stati a equazioni vettoriali
autonome del primo ordine, dove rappresentiamo in ogni punto \(\mathbf{x}\) dello spazio
\(\mathbb{R}^{n}\) la pendenza, data dal vettore \(\mathbf{f}(\mathbf{x})\),
successivamente possiamo rappresentare l'evoluzione del sistema in questo
spazio, ottenendo le \emph{orbite} (cercare su internet ``campo di pendenza
Lotka-Volterra'' oppure ``orbite Lotka-Volterra'' per vedere esempi, in tal caso
in \(\mathbb{R}^2\)). Le orbite quindi in generale (anche nel caso non autonomo)
sono \textbf{la proiezione della curva della soluzione su uno spazio degli
stati, che è dato dalle coordinate che non sono quella del tempo}.

\subsection{Alcuni esempi di risoluzione esplicita}
\begin{example}[Equazioni a variabili separabili]
    Si tratta di equazione differenziale della forma
    \[
        x'(t) = g(t) \, h(x)
    \]
    con \(g\) e \(h\) funzioni continue su un intervallo (ad esempio). Se
    \(\overline{x}\) è uno zero di \(h\) allora \(x(t) = \overline{x}\) è una
    soluzione costante. Un esempio chiaro è la logistica
    (figura~\ref{fig:logistica}) in cui se \(h(x) = x\left( 1- \frac{x}{K}
    \right) \) allora \(x(t) = 0\) e \(x(t) = K\) sono soluzioni costanti.

    Sia \(x: J \to \mathbb{R}\) una soluzione che non assume mai come valore uno
    degli zeri di \(h\). Allora possiamo scrivere
    \[
        \frac{x'(t)}{ h(x(t)) } = g(t) \quad \text{(su \(J\))}
    \]
    Da cui integrndo entrambi i membri
    \[
        \int \frac{x'(t)}{h(x(t))} dt = \int g(t) dt \iff \int \frac{1}{h(\xi)}
        d\xi = \int g(t) dt 
    \] 
    Dove si è fatto il cambio di variabile \(\xi = x(t)\). Allora se \(H\) e
    \(G\) sono primitive di \(\frac{1}{h}\) e di \(g\) rispettivamente si ha che
    \[
        H(x(t)) = G(t) + c
    \]
    che è un'equazione che definisce implicitamente la soluzione.
    \begin{example}
        Si consideri l'equazione \(x' = rt \left( x^2 - x \right) \) con \(r \in
        \mathbb{R}\). Allora si ha che \(h(x) = x^2 - x\) e \(g(t) = rt\).
        Gli zeri di \(h\) sono \(0\) e \(1\) per cui \(x(t) = 1\) e \(x(t) = 0\)
        sono le soluzioni costanti.

        Se invece \(x(t) \neq 0, 1\) allora si ha
        \[
            \frac{x'}{x^2 - x} = 2t \iff \int \frac{1}{x^2 - x} dx = \int 2t dt
            \iff \log \left| \frac{x-1}{x} \right| = t^2 + c
        \]
        quindi abbiamo che, elevando entrambi i membri alla base \(e\), e
        sostituendo \(e^{c} \in \mathbb{R}\) con \(C > 0\).
        \[
            \left|\frac{x-1}{x}\right| = Ce^{t^2}
        \]
        Poiché in questo caso il segno di \(\frac{x-1}{x}\) è costante su \(J\)
        (infatti \(x(J)\) è connesso e non contiene né \(0\) né \(1\)).
        Scegliendo quindi \(K \in \mathbb{R} \sminus \{0\}\) abbiamo
        \[
            x(t) = \frac{1}{1 - Ke^{t^2}} \quad K \neq 0
        \]
        Notiamo che invece per \(K = 0\) si recupera la soluzione costante \(x(t)
        = 1\).
    \end{example}
\end{example}

\begin{example}[Equazioni lineari del primo ordine]
    \[
        x'(t) + p(t)\,x(t) = q(t)
    \]
    con \(p, q\) funzioni continue su un intervallo \(I\).

    Sia \(P(t)\) una primitiva di \(p(t)\). Moltiplichiamo per \(e^{P(t)}\).
    \[
        e^{P(t)}x'(t) + e^{P(t)}p(t)x(t) = e^{P(t)}q(t)
    \]
    Dove il primo membro è evidentemente la derivata di \((e^{P(t)}x(t))\),
    quindi integrando entrambi i membri si ottiene
    \[
        e^{P(t)}x(t) = \int e^{P(t)}q(t) dt
    \]
    (Cioè \(e^{P(t)} x(t)\) è una primitiva di \(e^{P(t)} q(t)\)). Esplicitando
    \(x(t)\) otteniamo
    \[
        x(t) = e^{-P(t)} \int e^{P(t)}q(t) dt
    \]
    Se consideriamo il problema con dato \(x(t_{0}) = x_{0}\) la soluzione si
    può ottenere direttamente integrando tra \(t_{0}\) e \(t\), ossia
    \[
        \begin{cases}
            \displaystyle
            P(t) = \int_{t_{0}}^{t} p(s) ds \\
            \displaystyle
            x(t) = e^{-P(t)} \left( x_{0} + \int_{t_{0}}^{t} e^{P(s)}q(s) ds
                \right)
        \end{cases}
    \]
    Nel caso particolare dell'equazione \(x' = ax\), con \(a \in \mathbb{R}\)
    otteniamo (anche con variabili separabili) che \(x(t) = Ce^{at}\). In modo
    analogo si vede che le soluzioni di \(\varphi' = a\varphi\) con \(a \in
    \mathbb{C}\) e \(\varphi : J \to \mathbb{C}\) sono della forma \(\varphi(t)
    = C e ^{a t}\) con \(C \in \mathbb{C}\) 
    
    Per \emph{esercizio}, 
        Sapendo che la derivata \(\varphi'\) di \(\varphi : t \mapsto
        \varphi_{1} + i\varphi_{2}\) con \(\varphi_{1}, \varphi_{2} : J \to
        \mathbb{R} \) è \(\varphi' : t \mapsto \varphi_{1}' + i\varphi_{2}'\)
        mostrare che \(\frac{d}{dt}e^{at} = a e^{at}\), con \(a \in
        \mathbb{C}\).
\end{example}

\begin{example}[Equazioni lineari del secondo ordine a coefficienti costanti]
    \[
        ax'' + bx' + cx = 0
    \]
\begin{remark}
\begin{itemize}[label = --]
    \item L'insieme \(V\) delle soluzioni è uno spazio vettoriale
    \item La modellistica suggerisce che \(\dim V = 2\) 
\end{itemize}
    Ipotizziamo un tentativo di soluzione della forma \(x(t) = e^{\lambda t}\) 
    allora abbiamo \(x'(t) = \lambda e^{\lambda t}\) e \(x''(t) = \lambda^2
    e^{\lambda t}\). Sostituendo nella equazione e dividendo per \(e^{\lambda
    t} \neq 0\) otteniamo
    \[
        a\lambda^2 + b\lambda + c = 0 \quad \quad (A)
    \]
    Se \(\Delta = b^2 - 4ac > 0\) allora \((A)\) ha due soluzioni \(\lambda_{1},
    \lambda_{2}\) distinte, quindi \(e^{\lambda_{1} t}\) e \(e^{\lambda_{2} t}\)
    sono due soluzioni linearmente indipendenti.

    Se \(\Delta = 0\) (A) ha una soluzione \(\lambda\), per cui \(e^{\lambda
    t}\) è una soluzione e si verifica che anche \(te^{\lambda t}\) è soluzione
    linearmente indipendente con la precedente.

    Se \(\Delta < 0\) passiamo in campo complesso e cerchiamo \(u(t) = x(t) +
    iy(t)\) con \(x, y : J \to \mathbb{R}\) e \(u : J \to \mathbb{C}\), tale che
    anche \(u\) sia soluzione dell'equazione differenziale. Considereremo poi
    \(x(t) = Re(u(t))\). Come sopra ipotizziamo \(t \mapsto e^{\lambda t}\)
    soluzione e otteniamo la stessa equazione \((A)\), allora esistono due
    soluzioni \(\lambda_{1, 2} = \alpha \pm i\beta \), poiché i coefficienti di
    \((A)\) sono reali. Allora otteniamo le due soluzioni dell'equazione
    differenziale
    \[
        e^{(\alpha + i\beta)t} \quad e^{(\alpha - i\beta)t}
    \]
    che sono linearmente indipendenti. Ma lo spazio delle soluzioni generato da
    queste equazioni ammette anche la base \textbf{reale} data da 
    \[
        \frac{u + \overline{u}}{2} \text{ e } \frac{u - \overline{u}}{2i}
    \]
    che in questo caso sono \(e^{\alpha t}\cos(\beta t)\) e \(e^{\alpha t}
    \sin(\beta t)\).

    In tutti e tre i casi, se \(x_{1}(t)\) e \(x_{2}(t)\) sono due soluzioni
    linearmente indipendenti, allora una generica soluzione \(x(t)\) è della
    forma 
    \[
        x(t) = c_{1}x_{1}(t) + c_{2}x_{2}(t) \quad c_{1}, c_{2} \in \mathbb{R}
    \]

    Nel caso completo l'equazione è del tipo 
    \[
        ax'' + bx' + cx = f(t)
    \]
    Sia allora \(\overline{x}(\cdot )\) una soluzione (``nota''). L'integrale
    generale è dato dalle funzioni della forma 
    \[
        x(t) = x_{o}(t) + \overline{x}(t)
    \]
    al variare di \(x_{o} \) tra le soluzioni dell'equazione omogenea associata,
    ossia \(ax'' + bx' + cx = 0\).
    Questo perché se \(L : C^{2}(\mathbb{R}) \to C^{0}(\mathbb{R})\) un
    operatore lineare, ad esempio proprio \(Lx = (t \mapsto ax''(t) + bx'(t) +
    cx(t))\), allora il problema \(Lx = f\)  ha soluzione \(L^{-1}(f) =
    \overline{x} + \ker L\) con \(\overline{x}\) tale che \(L\overline{x} = f\)
    e \(\ker L\) lo spazio delle soluzioni dell'equazione omogenea associata
    \(Lx = 0\).

    In alcuni casi è particolarmente facile trovare una soluzione
    \(\overline{x}\) particolare, ad esempio ipotizzando ne esista una della
    forma di un polinomio di \(t\), e cercando i coefficienti per cui funzioni.
    In altri casi ha più senso cercare una soluzione particolare di forme
    diverse, ad esempio con esponenziali o seni e coseni o come ti pare, alla
    fine se il problema ti chiede di trovare una soluzione particolare è perché
    si può fare e allora ci pensi un attimo a occhio si vede facilmente.
\end{remark}
\end{example}

Sia \(D \subseteq \mathbb{R} \times \mathbb{R}^{n} \) aperto e \(f : D \to
\mathbb{R}^{ n } \) continua. Fissato \((t_{0}, x_{0}) \in D\) consideriamo il
problema ai valori iniziali
\[
    \begin{cases}
        x' = f(t, x) \\
        x(t_{0}) = x_{0}
    \end{cases}
\]
\begin{proposition}\label{proposition:equivalenza_pvi}
    Sia \(x : J \to  \mathbb{R}^{n}\) funzione continua tale che \((t, x(t)) \in
    D\)  per ogni \(t \in J\), ossia ha grafico in \(D\).
    Allora sono equivalenti:
\begin{enumerate}[label = \alph*)]
    \item \(x\) è soluzione del problema ai valori iniziali
    \item \(\displaystyle x(t) = x_{0} + \int_{t_{0}}^{t} f(s, x(s)) ds\) per ogni \(t \in J\) 
\end{enumerate}
\end{proposition}

Per quanto la proposizione~\ref{proposition:equivalenza_pvi} sia abbastanza
ovvia, ci serve per poter dimostrare il problema di esistenza e unicità delle
soluzioni, poiché permette di trasformare il problema di trovare una soluzione
in un problema del punto fisso.

\begin{definition}
Una funzione \(g: G \subseteq \mathbb{R}^{m} \to \mathbb{R}^{k} \) è detta
\textbf{lipschiitziana} se esiste una costante \(L > 0\) tale che 
\[
    \|g(x) - g(y)\| \le L \|x - y\| \quad \forall x, y \in G
\]
\end{definition}
\begin{proposition}
    Sia \(G\) un aperto convesso limitato e \(g \in C^{1}(\overline{G},
    \mathbb{R}^{k})\). Allora \(g\) è lipschitziana.
\end{proposition}
\begin{proof}
    Siano \(x_{1}, x_{2} \in G\) aperto convesso limitato.
    Per ogni \(t \in [0, 1]\) Sia
    \[
        \varphi(t) = g(x_{1} + t(x_{2} - x_{1}))
    \]
    Allora si ha che
    \begin{align*}
        g(x_{2}) - g(x_{1}) = \varphi(1) - \varphi(0) =
        \int_{0}^{1}\varphi'(t)dt = \int_{0}^{1} (Dg)(x_{1} +
        t(x_{2}-x_{1}))(x_{2}-x_{1})dt \\
        \le \int_{0}^{1} |Dg(x_{1} + t(x_{2}-x_{1}))| |x_{2}-x_{1}| dt \le
        \left(\max_{\overline{G}} |Dg|\right) |x_{2}-x_{1}|
    \end{align*}
\end{proof}
\begin{proposition}
    Sia \(G \in \mathbb{R}^{m}\) aperto e \(g \in C^{1}(G, \mathbb{R}^{k})\).
    Allora \(g\) è lipschitziana su ogni compatto \(K \subseteq G\) (anche detto
    \emph{localmente lipschitziana}).
\end{proposition}
\begin{example}
    La funzione \(x \mapsto x^2\) non è ovviamente lipschitziana su
    \(\mathbb{R}\). Eppure per ogni compatto \(K \subseteq \mathbb{R}\) (ad
    esempio \([-M, M]\)) la funzione è lipschitziana su \(K\), infatti 
    \[
        |x^2 - y^2| = |x+y||x-y| \le 2M|x-y|
    \]
\end{example}

\begin{definition}
    Sia \(D \subseteq \mathbb{R}^{N + 1} \) aperto e \(f : D \to
    \mathbb{R}^{N}\) (con \(f(t, \mathbf{x})\)). Diciamo che \(f\) è localmente lipschitziana nella seconda
    variabile, uniformemente rispetto alla prima se per ogni compatto \(K
    \subseteq D\) esiste una costante \(L_K >0\) tali che:
    \[
        \forall (t, \mathbf{x}), (t, \mathbf{y}) \in K, \quad \|f(t, \mathbf{x})
        - f(t, \mathbf{y})\| \le L_K \|\mathbf{x} - \mathbf{y}\|
    \]
\end{definition}
\begin{remark}
    Se ho una funzione \(C^{1}(D, \mathbb{R}^{N})\) allora è per la proposizione
    precedente localmente lipschitziana in entrambe le variabile, quindi per
    ogni \(K\subseteq D  \) compatto esiste \(L_K > 0\) tale che
    \[
        \forall (t_{1}, \mathbf{x}), (t_{2}, \mathbf{y}) \in K \quad \|f(t_{1},
        \mathbf{x}) - f(t_{2}, \mathbf{y})\| \le L_K \|(t_{1}, \mathbf{x}) - (t_{2},
        \mathbf{y})\|
    \]
    quindi in particolare se \(t_{1}=t_{2}=t\) abbiamo la locale lipschitzianità
    nella seconda variabile uniformemente rispetto alla prima
\end{remark}

\subsection{Esistenza e unicità delle soluzioni}
Sia \(D \subseteq \mathbb{R}^{N+1} \) aperto e \(f : D\to \mathbb{R}^{N}\) e
\((t_{0}, x_{0}) \in D\) e consideriamo il problema
\[
    (P) \quad
    \begin{cases}
        x' = f(t, x) \\
        x(t_{0}) = x_{0}
    \end{cases}
\]
\begin{figure}[ht]
    \centering
    \incfig[.6]{problema}
    \caption{Problema ai valori iniziali, \(\alpha = \frac{b}{M}\) in questo
    caso, evidentemente}\label{fig:problema}
\end{figure}
Siano \(a, b > 0\) tali che 
\[
    R_{a, b}  = [t_{0} - a, t_{0} + a] \times \overline{B}_{b}(x_{0}) \subseteq
    D
\]

\begin{theorem}[Esistenza e Unicità]
Sia \(f\) localmente lipschitziana nella seconda variabile uniformemente
rispetto alla prima. Allora
\begin{enumerate}[label = \alph*)]
    \item Esiste una soluzione di \((P)\) definita in \([t_{0} - \alpha, t_{0} +
        \alpha]\), con \(\alpha = \min\left(a, \frac{b}{M}\right)\), dove \(M = \max_{R_{a,
        b}}|f| \) 
    \item Nell'intervallo \([t_{0} - \alpha, t_{0} + \alpha]\) la soluzione,
        ossia se \(\varphi_{1} : J_{1} \to \mathbb{R}^{N}\) e \(\varphi_{2} :
        J_{2} \to \mathbb{R}^{N}\) sono soluzioni di \((P)\) allora
        \(x_{1}(t) = x_{2}(t) \forall t \in J_{1} \cap J_{2} \subseteq [t_{0} -
        \alpha, t_{0} + \alpha]\) 
\end{enumerate}
\end{theorem}
\begin{proof}
    Abbiamo visto che se \(x : J \to \mathbb{R}^{N}\) con \(J\) intervallo è una
    funzione continua con grafico in \(D\), allora \(x\) è soluzione di \((P)\)
    se e solo se 
    \[
        x(t) = x_{0} + \int_{t_{0}}^{t} f(s, x(s)) ds \quad \forall t \in J
    \]
    Sia \(x : [t_{0} - \alpha , t_{0} + \alpha] \to \overline{B}_b(x_{0})\).
    Poniamo ora
    \[
        (Tx)(t) = x_{0} + \int_{t_{0}}^{t} f(s, x(s)) ds \quad \forall t \in
        [t_{0} - \alpha, t_{0} + \alpha]
    \]
    Dove risolvere \((P)\) diventa mostrare che esiste un punto fisso di \(T\).
    Verifichiamo che \((Tx)(\cdot )\) assume valori in
    \(\overline{B}_b(x_{0})\). Infatti 
    \[
        |(Tx)(t) - x_{0}| = \left| \int_{t_{0}}^{t} f(s, x(s)) ds \right| \le
        \left|\int_{t_{0}}^{t} |f(s, x(s))| ds \right| \le M|t - t_{0}| \le M\alpha \le b
    \]
    Allora otteniamo che \(T : C^{0}([t_{0}-\alpha, t_{0} + \alpha];
    \overline{B}_b(x_{0})) \to C^{0}([t_{0}-\alpha, t_{0} + \alpha];
    \overline{B}_b(x_{0}))\). Possiamo quindi iterare l'applicazione di \(T\),
    quindi abbiamo la successione 
    \[
        x_{0}, x_{1} = Tx_{0}, x_{2} = Tx_{1}, \ldots, x_{n} = Tx_{n-1}
    \]
    Ora quindi mostriamo che la successione \(x_{n}\) converge uniformemente con
    il criterio di Cauchy
    \begin{align*}
        |x_{1}(t) - x_{0}(t)| &= \left| \int_{t_{0}}^{t} f(s, x_{0}(s))\right|
        \le M|t - t_{0}| \\
        |x_{2}(t) - x_{1}(t)| &= \left| \int_{t_{0}}^{t} f(s, x_{1}(s)) - f(s,
        x_{0}(s))\right| ds \le  \left| \int_{t_{0}}^{t} |f(s, x_{1}(s)) - f(s,
        x_{0}(s))| ds \right| \le \\
                                   &\le L_R \left|\int_{t_{0}}^{t} |x_{1}(s) - x_{0}(s)| ds\right| \le ML_R
        |\int_{t_{0}}^{t}|s-t_{0}| ds| = \frac{ML_R}{2} |t - t_{0}|^2
    \end{align*}
    \begin{align*}
        |x_{3}(t) - x_{2}(t)| &= \left| \int_{t_{0}}^{t} f(s, x_{2}(s)) - f(s,
        x_{1}(s))\right| ds \le \left| \int_{t_{0}}^{t} |f(s, x_{2}(s)) - f(s,
        x_{1}(s))| ds \right| \le \\
                              &\le L_R \left|\int_{t_{0}}^{t} |x_{2}(s) - x_{1}(s)| ds\right| \le
        \frac{ML_R^2}{2}
        \left|\int_{t_{0}}^{t}|s-t_{0}|^2 ds\right| = \frac{ML_R^2}{6} |t - t_{0}|^3
                           \\ &\vdots
    \end{align*}
    E in generale abbiamo 
    \[
        |x_{k+1}(t) - x_{k}(t)| \le \frac{ML_R^{k}}{(k+1)!} |t - t_{0}|^{k+1}
    \]
    Quindi abbiamo che se fissiamo \(n, m \in \mathbb{N}\) con \(m > n\) 
    \[
        |x_{m}(t) - x_{n}(t)| \le \sum_{k=n}^{m-1} |x_{k+1}(t) - x_{k}(t)| \le 
        \sum_{k=n}^{m-1} \frac{ML_R^{k}}{(n+1)!} |t - t_{0}|^{n+1} \le
        \frac{M}{L_R}
    \sum_{k=m}^{\infty} \frac{{(L_R \alpha)}^{k+1}}{(k+1)!}
    \]
    Che è il resto di una serie esponenziale, quindi per \(m, n\)
    sufficientemente grandi la serie converge a zero, per cui abbiamo mostrato
    la convergenza uniforme di \(x_{k}\). Ora sappiamo 
    \[
        x_k \longrightarrow x \in C^{0}([t_{0} - \alpha, t_{0} + \alpha];
        \overline{B}_b(x_{0})) \quad \text{uniformemente}
    \]
    Ora semplicemente possiamo portare il limite sotto il segno di integrale
    perché \(f(\cdot, x_k(\cdot )) \to f(\cdot, x(\cdot ))\) uniformemente in
    \([t_{0}-\alpha, t_{0}+\alpha]\) e
    quindi otteniamo che \(x\) è soluzione di \((P)\).

    Ora procediamo con l'unicità. Sia \(z : J \to \mathbb{R}^{N}\) una soluzione
    di \((P)\). Considieriamo i valori di \(t \in J \cap [t_{0}, t_{0} +
    \alpha]\) e analogalmente per \(t \le  t0\) e vogliamo mostrare che \(z(t) =
    x(t)\) con \(x(\cdot )\) la soluzione costruita primala soluzione costruita
    prima. Allora abbiamo per ipotesi che
    \[
        z(t) = x_{0} + \int_{t_{0}}^{t} f(s, z(s)) ds
    \]
    E valutiamo quindi
    \[
        |z(t) - x_{0}| \le \int_{t_{0}}^{t} |f(s, z(s))| ds
    \]
    Fissiamo ora un valora \(\overline{t} \in [t_{0}, t_{0}+ \alpha]\)
    arbitrario in \(J\) i \(t \in [t_{0}, \overline{t}]\). Allora \((s, z(s))\)
    per \(s in [t_{0}, t]\) è in un compatto di \(D\). Sia \(M_{1}\) il massimo di
    \(f\) su tale compatto. Allora 
    \[
        |z(t) - x_{0}| \le M_{1} |t - t_{0}| \quad \forall t \in [t_{0},
        \overline{t}]
    \]
    Ora procediamo con 
    \begin{align*}
        |z(t) - x_{1}(t)| &\le \int_{t_{0}}^{t} |f(s, z(s)) - f(s, x_{0})| ds \le
        L_{1} \int_{t_{0}}^{t} |z(s) - x_{0}| ds \le \\ &\le  L_{1} M_{1}
        \int_{t_{0}}^{t} |s - t_{0}| ds = \frac{L_{1}M_{1}}{2} |t - t_{0}|^2
    \end{align*}
    Con \(L_{1}\) una costante di Lipschitz per \(f\) relativa a un compatto che
    contiene sia \(R\) che il grafico di \(z\) 
    Procedendo in questo modo nuovamente si ottiene in generale
    \[
        |z(t) - x_{k}(t)| \le \frac{M_{1}L_{1}^{k}}{(k+1)!} |t - t_{0}|^{k+1}
        \longrightarrow 0 \text{ per \(k \to \infty\)}
    \]
\end{proof}
\begin{remark}[Unicità ``globale'']
    Sia \(x' = f(t, x)\) localmente lipschitziana in \(x\) uniformemente in
    \(t\). Siano \(x_{1}: J_{1} \to \mathbb{R}^{N}\) e \(x_{2}: J_{2} \to
    \mathbb{R}^{N}\) due soluzioni tali che esista \(t_{0} \in J := J_{1}\cap
    J_{2}\) con \(x_{1}(t_{0}) = x_{2}(t_{0})\) allora \(x_{1} = x_{2}\) su
    \(J\).
    Quindi la funzione 
    \[
        x(t) = \begin{cases}
            x_{1}(t) & t \in J_{1} \\
            x_{2}(t) & t \in J_{2}
        \end{cases}
    \]
    è soluzione sud \(J_{1} \cup J_{2}\) 
\end{remark}
\begin{proof}
    Consideriamo \(t > t_{0}\) Allora sia 
    \[
        \overline{t} := \sup \{ t \in J \mid x_{1}(t) = x_{2}(t) \text{ su }
        [t_{0}, t] \}
    \]
    Allora se \(\overline{t} = \sup J\) abbiamo finito, altrimenti consideriamo
    \(\overline{t} < \sup J\). Per continuità quindi \(x_{1}(\overline{t}) =
    x_{2}(\overline{t})\). Considerando ora il problama di Cauchy
    \[
        \begin{cases}
            x' = f(t, x) \\
            x(\overline{t}) = x_{1}(\overline{t}) = x_{2}(\overline{t})
        \end{cases}
    \]
    e ora per il teorema di esistenza e unicità esiste un intorno destro di
    \(\overline{t}\) si ha \(x_{1}(t) = x_{2}(t)\), che è in contraddizione con
    la definizione di \(\overline{t}\).
\end{proof}

Se manca l'ipotesi di lipschitzianità di \(f\) allora può cadere l'unicità.
Costruiamo un famoso esempio.
\begin{example}
    \(N = 1\), \(f(x) = \sqrt{|x|}\). È evidente non lipschitziana in quanto la
    derivata non è limitata.
    \begin{figure}[ht]
        \centering
        \begin{tikzpicture}
            \begin{axis}[
                xmin= -4, xmax= 4,
                ymin= -4, ymax = 4,
                axis lines = middle,
            ]
            \addplot[domain=-4:4, samples=101]{sqrt(abs(x))};
            \end{axis}
        \end{tikzpicture}
        \caption{Grafico di \(f(x) = \sqrt{|x|}\)}\label{fig:grafico_di_sqrt}
    \end{figure}
    Consideriamo il problema di Cauchy
    \[
        \begin{cases}
            x' = \sqrt{|x|} \\
            x(0) = 0
        \end{cases}
    \]
   Sia ora \(x : J \to \mathbb{R}\) una soluzione, con \(0 \in J\). Allora
   \(x(\cdot )\) è monotona non decrescente; consideriamo \(t \in J, t \ge 0\) e
   sia
   \[
       c_+ = \sup \{t\ge 0 : x(\cdot )= 0 \text{ su } [0, t]\} 
   \]
   Ora se \(c_+ = +\infty\) allora \(x(\cdot ) = 0\) su \([0, +\infty]\).
   Altrimenti, per \(t > c_+\) si ha che \(x(t) > 0\) quindi otteniamo
   \[
       \frac{x'(t)}{\sqrt{|x(t)|}} = 1 \iff \int
       \frac{x'(s)}{\sqrt{|x(s)|}} ds = t + c \iff
       2\sqrt{|x(t)|} = t + c
   \]
   Allora \(\displaystyle x(t) = \frac{1}{4} {(t + c)}^2\) e poiché \(x(c_+) =
   0\) allora \(c = -c_+\) quindi 
   \[
       x(t) = \frac{1}{4} {(t - c_+)}^2 \quad \forall t \ge c_+ 
   \]
   Similmente per \(t < 0 \) definendo similmente \(c_-\) abbiamo che le
   soluzioni del problema sono del tipo
   \[
       x(t) = \begin{cases}
           \frac{1}{4} {(t - c_+)}^2 & t \ge c_- \\
           0 & t \in [c_-, c_+] \\
           -\frac{1}{4} {(t - c_-)}^2 & t > c_+
       \end{cases}
   \]
   Dove \(c_-, c_+ \ge 0\) arbitrari (eventualmente anche \(+\infty\)). Quindi
   abbiamo infinite soluzioni. 
\begin{figure}[ht]
    \centering
    \incfig{pennellopeano}
    \caption{Pennello di Peano}\label{fig:pennellopeano}
\end{figure}
\end{example}
    Comunque abbiamo l'esistenza della soluzione. Esiste anche un teorema a
    riguardo
\begin{theorem}[Peano]
    Sia \(f : D \to \mathbb{R}^{N}\) continua e \((t_{0}, x_{0}) \in D\). Sia
    \(R_{a, b} \) come nel teorema di esistenza e unicità. Allora il problema
    \[
        \begin{cases}
            x' = f(t, x) \\
            x(t_{0}) = x_{0}
        \end{cases}
    \]
    ammette almeno una soluzione definita in \([ t_{0} - \alpha, t_{0} + \alpha
    ]\), dove \(\alpha = \min(a, \frac{b}{M})\) con \(M = \max_{R_{a, b}}|f|\).
\end{theorem}
Dimostreremo questo risultato usando un teorema di compattezza abbastanza
potente, che invece non dimostriamo. Procediamo con della terminologia.

    Sia \(\varphi_n : [a_{0}, b_{0}] \to \mathbb{R}\) continua. Diciamo che
\begin{itemize}[label = --]
    \item La successione \(\varphi_n\) è \textbf{equilimitata} se esiste \(M \in
        \mathbb{R}\) tale che
        \[
            \forall n \in \mathbb{N} \,\, \forall x \in [a_{0}, b_{0}] \quad
            |\varphi_n(x)| \le M
        \]
    \item La successione \(\varphi_n\) è \textbf{equicontinua} se 
        \[
            \forall \varepsilon > 0 \,\, \exists \delta > 0 \,\, \forall n \in
            \mathbb{N} \,\, \forall t',
            t'' \in [a_{0}, b_{0}] \quad |t' - t''| < \delta \implies
            |\varphi_n(t') - \varphi_n(t'')| < \varepsilon
        \]
\end{itemize}
\begin{theorem}[Ascoli \-- Arzelà]
    Sia \(\varphi_n\) una successione di funzioni equicontinue ed equilimitate
    su \([a_{0}, b_{0}]\). Allora esiste una sottosuccessione uniformemente
    convergente in \([a_{0}, b_{0}]\) 
\end{theorem}
\begin{proof}[Dimostrazione del teorema di Peano]
    Sia \(N = 1\) 
\begin{figure}[ht]
    \centering
    \incfig[.4]{teo_peano}
    \caption{Costruzione successione \(x_{n}\)}\label{fig:teo_peano}
\end{figure}
Fissato \(n \in \mathbb{N}\) suddividiamo \\ \([t_{0}, t_{0}+\alpha]\) in \(n\)
parti. Definiamo poi \(x_{n} : [t_{0}, t_{0} + \alpha] \to  \mathbb{R}\) come
segue (affine a tratti).
\begin{align*}
    x_{n}(t) &:= x_{0} + f(t_{0}, x_{0})(t - t_{0}) \quad \forall t \in [t_{0},
    t_{1}] \\
    x_n(t) &:= x_{n}(t_{1}) + f(t_{1}, x_{n}(t_{1}))(t - t_{1}) \quad \forall t
    \in [t_{1}, t_{2}] \\
    \vdots \\
    x_{n}(t) &:= x_{n}(t_{n-1}) + f(t_{n-1}, x_{n}(t_{n-1}))(t - t_{n-1}) \quad
    \forall t \in [t_{n-1}, t_{n}]
\end{align*} 
Notare che la pendenza di ogni tratto non supera, in valore assoluto, \(M\),
quindi \(M\alpha \le M \frac{b}{M} =b\) 

Ora vogliamo mostrare che la successione \(\{x_{n}\}\) soddisfa le iptesi del
teorema di Ascoli \-- Arzelà, infatti
\begin{itemize}[label = --]
    \item \emph{equilimitatezza} I valori sono in \([x_{0} - b, x_{0} + b]\) 
    \item \emph{equicontinuità} Sia \(\gamma_n = f(t_k, x_{n}(t_k))\) per \(t
        \in [t_k, t_{k+1}]\) costante a tratti. Allora comunque presi \(t', t''
        \in [t_{0}, t_{0}+\alpha]\) si ha che 
        \[
            (\star) \quad x_{n}(t') - x_{n}(t'') = \int_{t'}^{t''} \gamma_n ds
            \implies 
            |x_{n}(t') - x_{n}(t'')| \le M|t' - t''|
        \]
\end{itemize}
Allora esiste una sottosuccessione uniformemente convergnte \(x_{n_k} \to x\) in
\([t_{0}, t_{0}+\alpha]\). Per semplicità notazionale supponiamo che \(n_k = n\)
quindi abbiamo \(x_{n}\to x\). Ora da \((\star)\) otteniamo che per \(t'=t, t''
= t_{0}\)
\[
    x_{n}(t) = x_{0} + \int_{t_{0}}^{t} \gamma_n(s) ds
\]
Ora vorremmo poter passare al limite sotto il segno di integrale e ottenere la
soluzione del problema di cauchy, ma necessitiamo di convergenza uniforme di
\(\gamma_n(\cdot )\) a \(f(\cdot , x(\cdot ))\). Infatti abbiamo che, fissando
\(\varepsilon>0\) esiste per continuità uniforme di \(f\) un \(\delta > 0\) tale che
\[
    \forall (t_{1}, x_{1}), (t_{2}, x_{2}) \in R_{a, b} \quad |(t_{1}, x_{1}) -
    (t_{2}, x_{2})| \le \delta \implies |f(t_{1}, x_{1}) - f(t_{2}, x_{2})| \le
    \varepsilon
\]
Ora se \(s \in [t_k, t_{k+1} ]\) 
\begin{align*}
    |\gamma_n(s) - f(s, x(s))| = |f(t_k, x_{n}(t_k)) - f(s, x(s))| \le \\ \le |f(t_k,
    x_{n}(t_k)) - f(s, x_{n}(s))| + |f(s, x_{n}(s)) - f(s, x(s))|
\end{align*}
Ricordando ora che \(|x_{n}(t') - x_{n}(t'')| \le M |t' - t''\) abbiamo che se
\(n\) è sufficiente grande si ha che \(M / \frac{\alpha}{n} \le \delta\) e
allora \(|x_{n}(t_k) - x_{n}(s)| \le \delta\) e quindi \(|f(t_k, x_{n}(t_k)) -
f(s, x_{n}(s))| \le \varepsilon\). Inoltre per la convergenza uniforme di
\(x_n\) a \(x\) si ha che \(|f(s, x_{n}(s)) - f(s, x(s))| \le \varepsilon\).
Mettendo assieme i pezzi abbiamo quindi la convergenza uniforme di \(\gamma\) e
quindi possiamo passare al limite sotto il segno di integrale e ottenere che
\(x\) è soluzione del problema di Cauchy in \([t_{0}, t_{0} + \alpha]\)
\end{proof}
\begin{proposition}
    Sia \(f : D \to \mathbb{R}^{N}\) continua e \(K \subseteq D \) compatto.
    Allora esiste \(\alpha > 0\) (dipendente solo da \(K\)) tale che, per ogni
    \((t_{0}, x_{0}) \in K\) il problema
    \[
        \begin{cases}
            x' = f(t, x) \\
            x(t_{0}) = x_{0}
        \end{cases}
    \]
    ammette soluzione definita in \([t_{0} - \alpha, t_{0} + \alpha]\)
\end{proposition}
\begin{figure}[ht]
    \centering
    \incfig[.4]{compatto}
    \caption{compatto}
    \label{fig:compatto}
\end{figure}
\begin{remark}
    Modificando la costruzione della dimostrazione del teorema di Peano
    operando una suddvisione in \(n\) parti anche di \([x_{0}-b, x_{0}+b]\) e
    scegliendo su ogni tratto la pendenza \(\max f\) o \(\min f\) sul quadratino
    contenente \((t_{k}, x_{n}(t_k))\) (con qualche complicazione in più se
    becchi il punto di cambio quadratino) si ottengono due soluzioni del
    problema di Cauchy: le soluzioni massimale e minimale rispettivamente.
    Nell'esempio \(x' = \sqrt{|x|}\) abbiamo in particolare che \(0\) è la
    soluzione minimale e \(t^2\) è la soluzione massimale.
\end{remark}
\begin{example}
    Sia \(f(t, x) = x^2\) e consideriamo il problema di Cauchy
    \[
        \begin{cases}
            x' = x^2 \\
            x(0) = x_{0}
        \end{cases}
    \]
    Allora la soluzione è \(x(t) = -\frac{1}{t-\frac{1}{x_{0}}}\) che a
    prescindere di come si fa non può essere definita in \(t = x_{0}\).
    \begin{figure}[ht]
        \centering
        \begin{tikzpicture}
            \begin{axis}[
                xmin= -2, xmax= 5,
                ymin= -3, ymax = 3,
                axis lines = middle,
            ]
                \addplot[domain=-2:5, samples=100]{ - 1 / (x - 1) };
            \end{axis}
        \end{tikzpicture}
    \end{figure}
    Un obiettivo dei prossimi teoremi sarà capire quando si può prolungare il
    dominio su tutto \(\mathbb{R}\) o su molta parte.
\end{example}

Proviamo ora a guardare alla stima sull'intervallo di esistenza, data dal
Teorema di esistenza e unicità. Allora, ricordando
l'immagine~\ref{fig:problema} % TODO check ref
abbiamo che è definita in \([-\alpha, \alpha]\) con \(\alpha = \min(a, b/M)\) e
\(M = \max |f| = \max_{R_{a,b} }  |x^2| = \max_{[1-b, 1+b]}{(1+b)}^2\).
Quindi abbiamo \(\alpha = \min(a, \frac{b}{{(1+b)}^2})\). Poiché possiamo
prendere \(a\) arbitrariamente grande (\(f\) è definito su tutto \(\mathbb{R}^2
\ni (t, x)\)) allora \(\alpha = \frac{b}{{(1+b)}^2}\).

\subsection{Prolungamento}
Sia in seguito \(f : D\subseteq \mathbb{R}^{N+1} \to \mathbb{R}^{N} \) continua
e \(x : J \to \mathbb{R}^{N}\) soluzione di \(x' = f(t, x)\).

\begin{definition}
    Diciamo \textbf{prolungamento} di \(x\) ogni soluzione
    \[
        \hat{x} : \hat{J} \to \mathbb{R}^{N} \quad \hat{J} \not\supseteq J,
        \quad \hat{x}|_J = x
    \]
    se \(x(\cdot )\) non ammette prolungamento diciamo che è definita su un
    intervallo massimale.
\end{definition}
\begin{theorem}[Prolungamento]\label{th:prolungamento}
    Sia \(f\) continua, allora
\begin{enumerate}[label = \alph*)]
    \item Se \(x : J \to \mathbb{R}^{N}\) è definita su un intervallo massimale,
        questo è aperto. Posto \(J = (\omega_-, \omega_+)\) risulta:
        Per ogni \(K \subseteq D \) compatto esiste \(U\) intorno di
        \(\omega_+\) (\(\omega_-\)) tale che
        \[
            \forall t \in U \cap J \quad (t, x(t)) \not\in K
        \]
        ossia il grafico di \(x\) abbandona definitivamente ogni compatto, per
        \(t \to \omega_+ \) (\(\omega_-\)). Scrivereemo anche
        \[
            (t, x(t)) \to \partial D \quad \text{per } t \to \omega_{\pm} 
        \]
    \item Ogni soluzione ammete un prolungamento a un itervallo massimale

\end{enumerate}
\end{theorem}

\begin{figure}[ht]
    \centering
    \incfig[.5]{proprieta_massimale}
    \caption{Illustrazione del teorema di prolungamento}\label{fig:proprieta_massimale}
\end{figure}

\paragraph{Conseguenze per l'esistenza globale}
\begin{proposition}

    Sia \(D = I \times \Omega\), \(I \subseteq \mathbb{R} \) intervallo e
        \(\Omega \subseteq \mathbb{R}^{N} \) aperto. 
        Sia \(x : J \to \mathbb{R}^{N}\) una soluzione e \(J\) massimale.
        Supponiamo esiste \(Y \subseteq \Omega \) compatto tali che
        \[
            \forall  t \in J \quad x(t) \in Y
        \]
        Allora \(J = I\) e viene chiamata \textbf{esistenza globale}
\end{proposition}
\begin{proof}
    Mostriamo che \(\omega_+ = \sup I\). Se fosse \(\omega_+ < \sup I\), preso
    comunque \(t_{0} \in J\) si avrebbe
    \[
        (t, x(t)) \in  [t_{0}, \omega_+] \times Y \quad t_{0} \le  t < \omega_+
    \]
    che sarebbe assurdo perché significherebbe che il grafico di \(x\) è contenuta in un
    compatto
\end{proof}
\begin{figure}[ht]
    \centering
    \incfig[.5]{esistenza_globale}
    \caption{Esistenza Globale}\label{fig:esistenza_globale}
\end{figure}

\begin{example}
Consideriamo la logistica per esempio
\[
    x' = rx(1-\frac{x}{K})
\]
Consideriamo il problema
\[
    \begin{cases}
        x' = rx(1-\frac{x}{K}) \\
        x(0) = x_{0} \in (0, K)
    \end{cases}
\]
Il teorema di esistenza e unicità garantisce l'esistenza in \([t_{0} - \alpha,
t_{0}+ \alpha] = [-\alpha, \alpha]\). Sia \(\hat{x}\) un prolungamento
assicurato dal teorema di prolungamento. Per semplicità sia \(x : J \to
\mathbb{R}^{N}\) la soluzione di \((P)\) con \(J\) massimale. Sappiamo che le
costanti \(0\) e \(K\) sono soluzioni. Allora per unicità \(\forall t \in  J\),
\(x(t) \in (0, K)\) e quindi \(J = \mathbb{R}\).
\end{example}

\begin{example}
    Similmente abbiamo per (esercizio)
    \[
        x' = (t^2 + t + 1) x^2 \log ( 1 + {(1-\frac{x}{K})}^2 )
    \]
\end{example}

\begin{proposition}
    Sia \(D = I \times \mathbb{R}\) (quindi \(N = 1\)) e \(x : J \to
    \mathbb{R}\) soluzione con \(J\) massimale.
    Esistono
    \[
        x_{1}, x_{2} : I \to \mathbb{R}
    \]
    funzioni continue (non necessariamente soluzioni) tali che 
    \[
        \forall  t \in J : x_{1}
    \]
    Allora \(J = I\) 
\end{proposition}
\begin{proof}
    Similmente a prima, l'idea è che deve raggiungere il bordo di \(\Omega\) e
    non potendolo raggiungere ``in verticale'' deve raggiungerlo ``in
    orizzontale''. Più precisamnte, fisiamo \(t_{0} \in J\) e \(\beta \in
    (t_{0}, \sup I)\). Sia \(m = \min_{[t_{0}, \beta]} x_{1} \) e \(M =
    \max_{[t_{0}, \beta]}x_{2} \). Allora 
    \[
        \forall t \in J \cap [t_{0}, \beta] \quad x(t) \in [m, M] =: Y
    \]
    Allora per il risutato precedente \(J \supseteq [t_{0}, \beta) \) 
    Per l'arbitrarietà di \(\beta\) si ha \(\sup J = \sup I\) 
\end{proof}

\begin{proposition}
    Similmente a prima, ma sia ora \(N \ge 1\). Sia \(D = I \times
    \mathbb{R}^{N}\). Sia \(x : J \to \mathbb{R}^{N}\) soluzione con \(J\)
    masssimale. Se esiste \(\rho : I \to \mathbb{R}\) continua, tale che 
    \[
        \forall t \in J \quad |x(t)| \le \rho(t)
    \]
    Allora \(J = I\) 
\end{proposition}
\begin{proof}
    Similmente a sopra
\end{proof}

\begin{proposition}
    Sia \(D = I \times \mathbb{R}^{N}\) e \(f\) limitata.
    Sia \(x : J \to \mathbb{R}^{N}\) soluzione con \(J\) massimale. Allora \(J =
    I\) 
\end{proposition}
\begin{proof}
    se \(t_{0} \in J\) per ogni \(t \in J\) si ha che
    \begin{align*}
        x(t) = x(t_{0}) + \int_{t_{0}}^{t} f(s, x(s)) ds \iff |x(t)| &\le
        |x(t_{0})| + \int_{t_{0}}^{t} |f(s, x(s))| ds \le \\
        &\le |x(t_{0})| + M|t - t_{0}| =: \rho(t) \quad \forall t \in J
    \end{align*}
    dove \(M\) è un maggiorante per \(|f|\). A questo punto si applica il caso
    precedente 
\end{proof}

\begin{proof}[Dimostrazione del Teorema di Prolungamento] \(\) 
\begin{enumerate}[label = \alph*)]
    \item Sia \(x : J \to \mathbb{R}^{N}\) soluzione con \(J\) massimale. Allora
        \(J\) è aperto, infatti se ad esempio fosse
        \[
            \omega_+ := \sup J \in J
        \]
        Allora \((\omega_+, x(\omega_+)) \in D\). In tal caso \(x\) sarebbe
        prolungabile considerando la soluzione di \(z' = f(t, z) ; z(\omega_+) =
        x(\omega_+)\) che è chiaramente assurdo perché \(J\) è massimale.

        Non dimostriamo che \((t, x(t)) \to \partial D\) per \(t \to \omega_+\).
    \item Sia \(x : J \to \mathbb{R}^{N}\) una soluzione. Mostriamo che ammette
        un prolungamento massimale (destro wlog). Sia \(b = \sup J\). Se
        \(x(\cdot )\) non fosse prolungabile in \(t=b\) allora \(J\) sarebbe
        massimale. Supponiamo allora che
        \[
            x : [t_{0}, b] \to \mathbb{R}^{N} \text{ soluzione }
        \]
        Il grafico è compatto. Sia \(K \subseteq D \) un compatto contenente il
        grafico di \(x|_{[t_{0}, b]}\). Sappiamo ora che esiste \(\alpha_K >0\)
        tale che \(\forall (\tau_0, z_0) \in K\), il problema di Cauchy \(z' =
        f(t, z) ; z(\tau_0) = z_{0}\) ha soluzione in \([\tau_0 - \alpha_K, \tau_0
        + \alpha_K]\). Concludiamo quindi che possiamo prolungare \(x\) fino a
        \(b + \alpha_K\) considerando \((\tau_0, z_{0}) = (b, x(b))\). Se ora
        \((b + \alpha_K, x(b + \alpha_K)) \in K\) si ripete il ragionamento.
        Poiché \(K\) è compatto, dopo l'applicazione del ragionamento un numero
        finito di volte, poiché \(\alpha_K\) è il medesimo ad ogni passo, il
        grafico esce da \(K\), cioè esiste una \(b_K\) tale che \(x\) è definita
        su \([t_{0}, b_K]\) e \((b_K, x(b_K)) \not\in K\). Applichiamo il metodo
        considerando una famiglia \({\{V_{j}\}}_{j \in \mathbb{N}}\) con
        \(V_{j}\) aperto e \(V_{j} \subset\subset D \), \( V_{j} \to D \) crescenti,
        ossia \(V_{j} \subseteq V_{j+1}  \) e \(\bigcup_{j} V_{j} = D\) e
        inoltre \(\overline{V_{1}} = \text{graf}(x|_{[t_{0},b]})\). Applichiamo
        ora lo schema precedente con \(K = \overline{V_{1}}\) e allora esiste
        \(b_{1} = b_{\overline{V_{1}}} \) tale che \(x\) è definita su \([t_{0},
        b_{1}]\) e \((b_{1}, x(b_{1})) \not\in \overline{V_{1}}\) (per
        semplicità notazionale supponiamo che il grafico di
        \(x|_{[t_{0},b_{1}]}\) sia contenuto in \(V_{2}\)) allora ora iterando
        abbiamo che esiste \(b_{2} = b_{V_{2}}\) tale che \(x\) è definita su 
        \([t_{0}, b_{2}]\) e \((b_{2}, x(b_{2})) \not\in V_{2}\) e così via.
        Abbiamo ora \(b_{1} < b_{2}<\dots < b_k <\dots\) tale che \(x\) è
        definita su \([t_{0}, b_k]\) e \((b_k, x(b_k)) \not\in V_{k}\). Ne
        seguirà che, posto \(\omega_+ = \sup_{k \in \mathbb{N}}b_k \) si a che
        \(x\) è definita in \(J := [t_{0}, \omega_+)\). Se \(\omega_+ = +\infty\)
        allora banalmente \(J\) è massimale destro. Se \(J\) non fosse massimale
        si avrebbe \(x(\cdot )\) definita in \(\omega_+\) e \((\omega_+,
        x(\omega_+)) \in D\). Ciò è incompatibile con 
        \[
            \overline{V_k} \not\ni (b_k, x(b_k)) \to (\omega_+, x(\omega_+)) \quad \text{per }
            k \to \infty
        \]
\end{enumerate}
\end{proof}

\subsection{Stime}
\begin{theorem}[Lemma di Gronwall]
    Sia \(\beta \in C^{0}(I)\), \(\beta \ge 0\); \(a \in I\) e \(\alpha \in
    \mathbb{R}\) e \(u \in C^{0}(I)\) tali che
    \[
        u(t) \le \alpha + \int_{a}^{t} \beta(s) u(s) ds \quad \forall t \in I, t
        \ge a
    \]
    Allora
    \[
        u(t) \le \alpha e^{\int_{a}^{t} \beta(s) ds} \quad \forall t \in I, t
        \ge a
    \]
\end{theorem}
\begin{proof}
    % TODO
\end{proof}

\paragraph{Applicazione al problema di Cauchy}
Consideriamo l'equazione differenziale \(x' = f(t, x)\) con \(f : D \to
\mathbb{R}^{N}\) continua localmente lipschitziana. Siano \(x_{1},x_{2} : J \to
\mathbb{R}^{N}\) due soluzioni. Fissiiamo \(t_{0} \in J\) si ha che
\begin{align*}
    x_{1}(t) &= x_{1}(t_{0}) + \int_{t_{0}}^{t} f(s, x_{1}(s)) ds \\
    x_{2}(t) &= x_{2}(t_{0}) + \int_{t_{0}}^{t} f(s, x_{2}(s)) ds
\end{align*}
Sottraendo membro a membro si ha
\[
    |x_{1}(t) - x_{2}(t)| \le |x_{1}(t_{0}) - x_{2}(t_{0})| + \int_{t_{0}}^{t}
    |f(s, x_{1}(s)) - f(s, x_{2}(s))| ds
\]
Fissato \(\overline{t} > t_{0}\), se \(t \in [t_{0}, \overline{t}]\) e \(K\) è
un compatto di \(D\) contenente i grafici di \(x_{1, 2}|_{[t_{0}, \overline{t}]}
\). Sia allora \(L_K\) una costante di Lipschitz di \(f\) relativa a \(K\).
Allora
\[
    \underbrace{|x_{1}(t) - x_{2}(t)|}_{u(t)} \le \underbrace{|x_{1}(t_{0}) -
    x_{2}(t_{0})|}_{\alpha} + \underbrace{L_K}_{\beta}
    \int_{t_{0}}^{t} \underbrace{|x_{1}(s) - x_{2}(s)|}_{u(s)} ds
\]
Che rispetta le ipotesi del lemma di Gronwall e quindi sappiamo che
\[
    \underbrace{|x_{1}(t) - x_{2}(t)|}_{u(t)} \le \underbrace{|x_{1}(t_{0}) -
    x_{2}(t_{0})|}_{\alpha} e^{L_K(t-t_{0})}
    \quad \forall t \ge t_{0}
\]
Da tale schema scende anche l'unicità della soluzione del problema di Cauchy.
Infatti se \(x_{1}(t_{0}) = x_{2}(t_{0})\) abbiamo
\[
    |x_{1}(t) - x_{2}(t)| \le 0 \quad \forall t \in J, t \ge t_{0}
\]

Iniziamo ora con un'idea del prossimo teorema, con un'intuizione geometrica. Sia
dato un campo di pendenze (\(N=1\)). Ossia 
\[
    (t, x) \mapsto \omega(t, x) \in \mathbb{R}^2
\]
\begin{figure}[ht]
    \centering
    \incfig[.8]{confronto}
    \caption{La curva \(x(\cdot)\) ha pendenza sempre minore della pendenza che
    il campo di pendenze assegna in ogni punto, mentre la curva \(u(\cdot )\)
segue la pendenza in ogni punto e ha valore iniziale \(u(a)\) non minore di \(x(a)\)  }\label{fig:confronto}
\end{figure}

\begin{theorem}[Teorema del confronto]
    Sia \(D \subseteq \mathbb{R}^2 \) aperto e \(\omega : D \to \mathbb{R}\)
    continua e localmente lipschitziana in \(x\) uniformemente rispetto a \(t\).
    Sia \(x, u : J \to \mathbb{R}\) di classe \(C^{1}\), sia \(a \in J\) 
        Se \(x'(t) \le \omega(t, x(t))\) e \(u'(t) = \omega(t, u(t))\), se
        \(x(a) \le u(a)\) allora
        \[
            x(t) \le u(t) \quad \forall t \in J, t \ge a
        \]
\end{theorem}
\begin{proof}
    Supponiamo per assurdo che esista un valore \(\overline{t} \in J\) con
    \(\overline{t} > a\) tale che \(x(\overline{t})> u(\overline{t})\).
    Sia \(t_{0} = \sup \{t \in [0, \overline{t}] : x(t) \le u(t)\} \). Allora
    \(x(t_{0}) = u(t_{0})\) (per continuità)e \(x(t) > u(t)\) per \(t \in
    (t_{0}, \overline{t}]\).
    Adesso per ogni \(t \in [t_{0}, \overline{t}]\) abbiamo
    \[
        x(t) = x(t_{0}) + \int_{t_{0}}^{t} x'(s) ds \le x(t_{0}) +
        \int_{t_{0}}^{t} \omega(s, x(s)) ds
    \]
    e
    \[
        u(t) = u(t_{0}) + \int_{t_{0}}^{t} u'(s) ds = u(t_{0}) + \int_{t_{0}}^{t}
        \omega(s, u(s)) ds
    \]
    Sottraendo membro a membro e prendendone il module otteniamo
    \[
        x(t) - u(t) \le x(t_{0}) - u(t_{0}) + \int_{t_{0}}^{t} \omega(s, x(s)) - \omega(s, u(s)) ds
    \]
    Sia ora \(L_K\) una costante di Lipschitz relativa a un compatto \(K\)
    contenente i grafici di \(x, u\) su \([t_{0}, \overline{t}]\). Allora
    \[
        x(t) - u(t) \le x(t_{0}) - u(t_{0}) + L_K \int_{t_{0}}^{t} |x(s) -
        u(s)| ds
    \]
    Ma ora in \([t_{0}, \overline{t}]\) possiamo togliere il modulo perché
    abbiamo che \(x(t) \ge u(t)\). Abbiamo allora
    \[
        \underbrace{x(t) - u(t)}_{u(t)} \le \underbrace{x(t_{0}) -
        u(t_{0})}_{\alpha=0} + \underbrace{L_K}_\beta \int_{t_{0}}^{t}
        \underbrace{x(s) - u(s)}_{u(s)} ds
    \]
    Che rispetta le ipotesi del lemma di Gronwall e quindi sappiamo che
    \[
        x(t) - u(t) \le 0 \quad \forall t \in [t_{0}, \overline{t}]
    \]
    che è assurdo.
\end{proof}

\begin{example}
    Consideriamo 
    \[
        \begin{cases}
        x' = rx\left( 1 - \frac{x}{k} \right)  \\
        x(0) = x_{0} < 0
        \end{cases}
    \]
    Allora evidentemente \(x' \le -\frac{r}{k} x^2\) per \(x < 0\) e quindi
    abbiamo
    \[
        \begin{cases}
            x' \le  -\frac{r}{k} x^2 \\
            x(0) = x_{0} < 0
        \end{cases}
        \quad 
        \begin{cases}
            u' =  -\frac{r}{k} u^2 \\
            u(0) = x_{0}
        \end{cases}
    \]
    Ne concludiamo che \(x(t) \le u(t)\) per il teorema del confronto. In
    particolare abbiamo che \(\omega_+ < +\infty\) 

    Vogliamo ora fare un ragionamento simile per studiare il caso \(x_{0} > k\).
    Se consideriamo la parabola di equazione \(y = rx(1-\frac{x}{k})\) abbiamo
    che sicuramente esiste una parabola \(-\gamma x^2\) tale per cui \(y \le
    -\gamma x^2  \) per \(x \ge x_{0}\) e quindi abbiamo che
    \[
        \begin{cases}
            x' \le -\gamma x^2 \\
            x(0) = x_{0} > k
        \end{cases}
        \quad
        \begin{cases}
            u' = -\gamma u^2 \\
            u(0) = x_{0}
        \end{cases}
    \]
    e quindi \(x(t) \ge u(t)\) per il teorema del confronto. In particolare
    abbiamo che \(x\) ha un asintoto verticale a sinsitra (per \(t < 0\)).
\end{example}
\begin{remark}[Teorema dell'asintoto]
    Sia \(x : [a, +\infty) \to \mathbb{R}\) una funzione derivabile tale che 
    \[
        \lim_{t \to +\infty} x(t) = l \in \mathbb{R} \quad \text{ (finito,
        \(x(\cdot )\) ha un asintoto)}
    \]
    Inoltre esiste \(\lim_{t \to +\infty} x'(t)\). 
    Allora
    \[
        \lim_{t \to +\infty} x'(t) = 0
    \]
    Infatti abbiamo
    \[
        x(t + 1) - x(t) = x'(\xi_t) \quad \xi_t \in (t, t+1)
    \] e portando al limite otteniamo
    \[
        0 = l - l = \lim_{t \to +\infty} x'(\xi_t) = 0
    \]

    La conseguenza è che se ho \(x(\cdot )\) soluzione del problema di Cauchy
    della logistica, allora se \(x \to l\) per \(t \to +\infty\) ho che 
    \[
        x'(t) = rx(t)(1-\frac{x(t)}{k}) \to rl (1 - \frac{l}{k}) = 0
    \]
    quindi necessariamente \(l=0\) (no) oppure \(l=k\) (sì).
\end{remark}

\subsection{Dipendenza continua}
\begin{theorem}
    Siano \(f_{0}, f_{j} : D \to \mathbb{R}^{N}\), con \(j \in \mathbb{N}\) e
    \(D \in \mathbb{R}^{N+1}\) funzioni continue. Consideriamo i problemi
    \[
        (P_{j})\,\,\begin{cases}
            x' = f_{0}(t, x) \\
            x(t_{0}) = x_{0}
        \end{cases}
        \quad
        (P_{0}) \,\,
        \begin{cases}
            x' = f_{j}(t, x) \\
            x(t_{0}^{j}) = x_{0}^{j}
        \end{cases}
    \]
    dove \({(t_{0}^{j}, x_{0}^{j})}\) e \((t_{0},x_{0})\) sono assegnati in
    \(D\). Supponiamo ora che \(f_{j}\to f_{0}\) uniformemente sui compatti di
    \(D\), che \((t_{0}^{j}, x_{0}^{j}) \to {(t_{0},x_{0})}\) e supponiamo che
    \(f_{0}\) sia localmente Lipschitziana in \(x\) uniformemente rispetto a
    \(t\).
    Sia \([a,b]\) un intervallo su cui è definita la soluzione \(\varphi_{0}\)
    di \((P_{0})\). Se \(\varphi_j\) risolve \((P_{j})\) allora per \(j\)
    sufficientemente grande \(\varphi_j\) è definita in \([a,b]\) e \(\varphi_j
    \to \varphi_{0}\) uniformemente su \([a,b]\) 
\end{theorem}
\begin{figure}[ht]
    \centering
    \incfig[.4]{dipcontinua}
    \caption{dipcontinua}\label{fig:dipcontinua}
\end{figure}
\begin{remark}
    Si noti il caso particolare in cui \(f_{j}=f_{0}\).
\end{remark}
Per il precedente teorema esiste anche una forma variante.
    \[
        \begin{cases}
            x'=f(t, x, \lambda) \\
            x{(t_{0})} = x_{0}
        \end{cases}
    \]
    con \(f : G \to \mathbb{R}^{N}\) e \(G \subseteq \mathbb{R}\times
    \mathbb{R}^{N} \times \mathbb{R}^{m} \) aperto e chiamiamo \(G_\lambda :=
    \{{(t,x)} \in \mathbb{R}^{N+1} : {(t,x,\lambda)} \in G\} \). Supponiamo ora
    che \(f\) sia continua e tale che \(f{(\cdot , \cdot ,\lambda)}\) sia
    localmente lipschitziana su \(G_\lambda\) nella seconda variabile
    uniformemente rispetto alla prima, per ogni \(\lambda\).~\emph{In ipotesi di
unicità} indichiamo con \(x(\cdot ,t_{0}, x_{0})\) l'unica soluzione, definita
sul suo intervallo massimale \({(\omega_-{(t_{0},x_{0})},
\omega_+{(t_{0},x_{0})})}\). Consideriamo ora il problema
\[
    (P_\lambda) \begin{cases}
        x' = f(t, x, \lambda) \\
        x(t_{0}) = x_{0}
    \end{cases}
\]
    dove \({(t_{0},x_{0},\lambda)}\) è assegnato in \(G\). Indichiamo con
    \(x{(\cdot ,t_{0},x_{0},\lambda)}\) la soluzione di \((P_\lambda)\) definita
    nel proprio intervallo massimale \({(\omega_-{(t_{0},x_{0},\lambda)},
    \omega_+{(t_{0},x_{0},\lambda)})}\). Allora abbiamo
\begin{theorem}
\begin{itemize}
    \item \(\omega_-\) (\(\omega_+\)) è semicontinua superiormente
        (inferiormente) in \(G\) 
    \item l'insieme 
        \[
            E = \{{(t,t_{0},x_{0},\lambda)} \in \mathbb{R}\times G :
            \omega_-{(t_{0},x_{0},\lambda)} < t < \omega_+{(t_{0},x_{0},\lambda)}\} 
        \]
        è aperto
    \item \(x : E \to \mathbb{R}^{N}\) definita da
        \({(t,t_{0},x_{0},\lambda)} \mapsto x{(t,t_{0},x_{0},\lambda)}\) è
        continua
\end{itemize}
\end{theorem}
\begin{definition}{Semicontinuità}
    Sia \(X\) uno spazio topologico, \(x_{0} \in X\) e \(f: X \to \mathbb{R}\).
    Diciamo che \(f\) è \textbf{semicontinua inferiormente} in \(x_{0}\) se
    \[
        \forall \varepsilon > 0 \quad \exists U \text{ intorno di } x_{0} :
        f(x) > f(x_{0}) - \varepsilon \quad \forall x \in U
    \]
\end{definition}
Dove vale la seguente caratterizzazione
\begin{proposition}
    Sono equivalenti: 
\begin{enumerate}[label = \roman*)]
    \item \(f\) è semicontinua inferiormente in \(X\)
    \item \(\{x \in X : f{(x)}>\alpha\} \) è aperto per ogni \(\alpha \in \mathbb{R}\) 
    \item l'epigrafico di \(f\), cioè 
        \[
            \text{epi}(f) = \{{(x, \alpha)} \in X \times \mathbb{R} : f(x) \le
            \alpha\} 
        \]
        è chiuso
    \item \(\liminf_{x \to x_{0}} f(x) \ge f(x_{0})\)
\end{enumerate}
\end{proposition}
\begin{proof}\( \)
\begin{itemize}
    \item[\(i) \implies ii)\)] Fissiamo \(x_{0} \in \{f>\alpha\}\); allora
        \(f(x_{0}) > \alpha\). Dalla definizione esiste \(U\) intorno di
        \(x_{0}\) tale che \(\forall x \in U\) si ha che \(f(x) \ge f(x_{0})
        -\varepsilon > \alpha\) e quindi \(U \subseteq \{f > \alpha\}\) 
    \item[\(ii) \implies iii) \)] 
    \item[\(iii) \implies iv) \)]
    \item[\(iv) \implies i)\)]
\end{itemize}
\end{proof}

\begin{proof}[Dimostrazione parziale del Teorema]
    Dalla caratterizzazione di semicontinuità inferiore sappiamo che l'insieme
    \(\{(t, t_{0},x_{0}, \lambda) : t < \omega_+{(t_{0},x_{0},\lambda)}\} \) è
    aperto. Analogamente per \(\{{(t,t_{0},x_{0},\lambda)}: t>
        \omega_-{(t_{0},x_{0},\lambda)} \). L'intersezione dà \(E\) aperto.
        Dimostrare che \(\omega_-\) e \(\omega_+\) sono semicontinue
        rispettivamente superiormente e inferiormente è molto complicato.
\end{proof}
\begin{figure}[ht]
    \centering
    \incfig[.5]{prooftheorema}
    \caption{prooftheorema}
    \label{fig:prooftheorema}
\end{figure}
\begin{eser}
    \[
        (P) \begin{cases}
            x' = 1-te^{x} \\
            x(0) = \alpha \in \mathbb{R}
        \end{cases}
    \]
     Studiamo quando \(x' > 0\) e abbiamo \(1 - te^{x}> 0\) e quindi \(te^{x}<1\)
     e \(t < e^{-x}\) e quindi \(x < -\log t\).
     Sia ora 
     \[
         \overline{t} = \sup \{t \ge t_{1} : x(\tau) > \psi {(\tau)} \forall \tau
         \in [t_{1}, t]\} 
     \]
    Supponiamo ora \(\overline{t} < +\infty\) e allora sia \(\delta(t) =
    x{(t)}-\psi {(t)}\). Allora \(\delta {(\overline{t})}=0\) per continuità e
    \(\delta'{(\overline{t})} = \underbrace{x'{(\overline{t})}}_{0}  -
    \underbrace{\psi'{(\overline{t})}}_{<0} > 0\). Ne consegue che in un intorno
    sinistro di \(\overline{t}\) si ha che \(\delta(t) < 0\) che è assurdo. Ne
    conseguo che \(\overline{t} = +\infty\) e quindi si ottiene che \(\omega_+ =
    +\infty\), infatti il grafico di \(x{(\cdot )}\) si deve trovare sotto al
    valore \(x{(t_{1})}\) e sopra al grafico di \(\psi\). Per monotonia esiste
    \(\lim_{t\to +\infty} x{(t)}\). Se fosse finito si avrebbe
    \[
        \lim_{t\to +\infty} x'{(t)} = \lim_{t \to +\infty} 1 - te^{x{(t)}} =
        -\infty
    \]
    escluso per il teorema dell'asintoto. Ne concludiamo che \(\lim_{t\to
    +\infty} = -\infty\).

    Per \(t < 0\) invece abbiamo che
    \[
        0 \le 1-te^{x} \le 1 - t e^{\alpha} = 1 + e^{\alpha} |t|
    \]
    che è una crescita sottolineare e quindi \(\omega_- = -\infty\) e per un
    argomento simile al precedente si ha che \(\lim_{t \to -\infty} x{(t)} =
    -\infty\)
\end{eser}
\begin{figure}[ht]
    \centering
    \incfig[.6]{eser1}
    \caption{Esercizio 1.1}\label{fig:eser1}
\end{figure}
\begin{eser}
    Si studi qualitativamente il problema di Cauchy
    \[
        \begin{cases}
            x' = x{(x^3-t^2)} \\
            x{(1)}=1
        \end{cases}
   \]
\end{eser}
\begin{remark}[sulla dipendenza continua]
    Consideriamo il problema di Cauchy
    \[
        \begin{cases}
            x' = x^2 \\
            x{(0)} = x_{0} > 0
        \end{cases}
    \]
    allora otteniamo \( \displaystyle x{(t)} = - \frac{1}{t-\frac{1}{x_{0}}}\).

    Fissato \(T > 0\) consideriamo i valori \(x_{0}\) per i quali la soluzione è
    definita (almeno) in \([0, T]\), quindi \(x_{0} < \frac{1}{T}\). Per \(x_{0}
    \to 0\) per la dipendenza continua la soluzione tende (uniformemente perché
    su \([0, T]\)) alla soluzione 0. In particolare \(x{(T)} < \varepsilon\) se 
    \[
        \frac{1}{x_{0}} - T > \frac{1}{\varepsilon} \iff \frac{1}{x_{0}} > T +
        \frac{1}{\varepsilon} \iff x_{0} < \frac{1}{T + \frac{1}{\varepsilon}}
        =: \delta_{\varepsilon, T} 
    \]
    per cui \(\delta\) dipende sia da \(\varepsilon\) che da \(T\) e non è
    possibile individuarlo uniformemente rispetto a \(T\).

    La proprietà, diversa, di ``dipendenza continua'' su intervalli illimitati
    darà luogo alla definizione di stabilità.
\end{remark}

\section{Sistemi autonomi}
Sia \(\Omega \subseteq \mathbb{R}^{N} \) aperto e sia \(f : \Omega \to
\mathbb{R}^{N}\). Supponiamo \(f\) localmente lipschitziana in \(\Omega\). Il
sistema che consideriamo è 
\[
    \begin{cases}
        x' = f(x) \\
        x(t_{0}) = x_{0}
    \end{cases}
\]
Indichiamo con \(x{(\cdot , t_{0}, x_{0})}\) la soluzione definita
sull'intervallo massimale.
\begin{proposition}
\begin{enumerate}[label = \arabic*.]
    \item Sia \(x : J \to \mathbb{R}^{N}\) una soluzione di \(x' = f{(x)}\); sia
    \(\tau > 0\). Allora \(x_\tau : J+\tau \to \mathbb{R}^{N}\) definita da
    \(x_\tau{(t)} = x{(t - \tau)}\) è soluzione.
\item Se \(x_{1}\) e \(x_{2}\) sono due soluzioni di \(x' = f{(x)}\) che assumono
    un valore comune allora sono traslate temporali l'una dell'altra.
\end{enumerate}
\end{proposition}
\begin{proof}
\begin{enumerate}[label = \arabic*.]
    \item Già vista 200 volte: 
        \[
            x'_\tau{(t)} = x'{(t - \tau)} = f{(x{(t-\tau)})} = f{(x_\tau{(t)})}  
        \]
\begin{figure}[ht]
    \centering
    \incfig[.4]{prop_2}
\end{figure}
    \item usando la notazione del disegno, supponiamo \(u{(t)} = x_{1}{(t -
        {(t_{2} - t_{1})})}\). Allora \(u\) è soluzione di \(x' = f{(x)}\) in
        quanto traslata di \(x'\). Inoltre \(u{(t_{2})} = x_{1}{(t_{1})} =
        \overline{x} = x_{2}{(t_{2})}\), per unicità quindi \(u \equiv x_{2}\) 
\end{enumerate}
\end{proof}
Allora stesso modo si vede che \(x{(t, t_{0}, x_{0})} = x{(t-t_{0}, 0,
x_{0})}\). Infatti \(x{(\cdot , t_{0}, x_{0})}\) risolve il problema originale e
allora
\[
    x{(\cdot - t_{0}, 0, x_{0})} \text{ risolve } \begin{cases}
        x' = f{(x)} \\
        x(\cdot - t_{0}, 0, x_{0})|_{t = t_{0}}  = x{(0, 0, x_{0})} = x_{0}
    \end{cases}
\]
che per unicità coincidono.

Ne consegue che posso sempre ricondurre un problema di Cauchy autonomo a uno in
cui \(t_{0} = 0\). È una cosa così comune che motiva la seguente definizione

\begin{definition}
    Diciamo \textbf{flusso} associato all'equazione differenziale \(x' = f
    {(x)}\) la funzione
    \[
        \varphi {(t, \xi)} = x{(t , 0, \xi)}
    \]
    con \(\xi \in \Omega\) e \( t\) variabile nell'intervallo massimale
    \({(\omega_-{(\xi)}, \omega_+{(\xi)})}\) della soluzione \(x{(\cdot , 0, \xi)}\) 
\end{definition}

\begin{proposition}
    \begin{itemize}
    \item \(\omega_+\, [\omega_-]\) è semicontinua inferiormente [superiormente] in
    \(\Omega\)
\item \[
        \{{(t, \xi)} \in \mathbb{R} \times \Omega : \xi \in \Omega,
        \omega_-{(\xi)} < t < \omega_+{(\xi)}\} 
    \]
    è aperto.

\item \(\varphi : E \to \mathbb{R}^{N}\) è continua

\item \(\varphi {(0, \cdot )} = \text{id}_\Omega\) 

\item \(\varphi {(s, \varphi {(t, \xi)})} = \varphi {(s+t, \xi)}\) 
    \end{itemize}
\end{proposition}
    
\begin{proof}[Dimostrazione ultimo punto]
    Le funzioni \(\varphi {(\cdot , \varphi {(t, \xi)})}\)  e \(\varphi {(\cdot
    +t, \xi)}\) sono due soluzioni di \(x'=f{(x)}\) che coincidono in \(t = 0\),
    poiché valgono entrambe \(\varphi {(t, \xi)}\). Per unicità quindi
    coincidono.
\end{proof}

\begin{remark}
    Sia \({(X, d)}\) uno spazio metrico che sia \(\varphi : \mathbb{R} \times X
    \to X\) continua tale che
    \[
        \begin{cases}
            \varphi {(0, \cdot )} = \text{id}_X \\
            \varphi {(s, \varphi {(t, \xi)})} = \varphi {(s+t, \xi)}
        \end{cases}
    \]
    allora \(\varphi \) è anche detta ``sistema dinamico''
\end{remark}

\begin{definition}

Dato \(x_{0} \in \Omega\) diciamo \textbf{orbita} per \(x_{0}\) l'insieme 
\[
    \gamma_{x_{0}} = \{\varphi {(t, x_{0})} : t \in {(\omega_-{(x_{0})},
    \omega_+{(x_{0})})}\}
\]
\end{definition}
\begin{remark}
    Utilizzando \(t \mapsto x{(t, t_{0},x_{0})}\) in luogo di \(x{(t, 0,
    x_{0})}\) si ottiene lo stesso insieme \(\gamma_{x_{0}}\), perché è
    semplicemente una traslata temporale.
\end{remark}
\begin{note}
    \(\gamma_{x_{0}} \) è la proiezione del grafico di \(x{(\cdot ,0, x_{0})}\)
    su \(\Omega\) 
\end{note}
\begin{proposition}
    Se due orbite hanno un punto in comune, allora coincidono
\end{proposition}
\begin{proof}
    Solito discorso di traslazione temporale, perché due soluzioni che assumono
    lo stesso valore devono essere traslate temporali una dell'altra, e quindi
    la proiezione su \(\Omega\) coincide.
\end{proof}
\begin{proposition}
    Se un'orbita non è un singoletto allora è una curva regolare.
\end{proposition}
\begin{proof}
    Sia \(x\) soluzione di \(x' = f{(x)}\). Se \(x'{(t_{0})} = 0\) per un
    qualche \(t_{0}\) allora
    \[
        0 = x'{(t_{0})} = f{(x{(t_{0})})}
    \]
    ma allora posto \(x{(t_{0})} = x_{0}\) la funzione costante \(u{(t)} =
    x_{0}\) è una soluzione, e per unicità è l'unica soluzione, quindi l'orbita
    è un singoletto.
\end{proof}
\begin{eser}
    Si consideri il sistema
    \[
        \begin{cases}
            x' = \frac{x - t^2}{x^2 + t^2} \\
            x{(0)} = a > 0
        \end{cases}
    \]
    Allora \(f{(t, x)} > 0\) per \(x > t^2\) 
    Per il teorema di esistenza e unicità esiste \(\delta > 0\) tale che la
    soluzione esiste in \([-\delta, \delta]\). Possiamo supporre \(\delta < a\).
    Per \(t \ge \delta\) 
    \[
        \left| \frac{x-t^2}{x^2 + t^2} \right| \le \frac{1}{\delta} {\left(
        |x|+t^2 \right)} 
    \]
    e quindi per crescita sottolineare \(\omega_+ = +\infty\) 
    Ora l'obiettivo è trovare una funzione \(u\) tale che \(x' \le u'\) e \(u\)
    taglia \(x = t^2\) 
\end{eser}
\begin{eser}
    Si risolva il problema di Cauchy
    \[
        \begin{cases}
            x' = \frac{x{(1-x)}}{2t}\\
            x{(0)} = x_{0}
        \end{cases}
    \]
    in \(t > 0\). Poi si studi qualitativamente il problema di Cauchy
    \[
        \begin{cases}
            x' = \frac{g{(x)}}{2t} \\
            x{(1)} = x_{1}
        \end{cases}
    \]
    dove \(g{(x)}\) ha lo stesso segno di \(x {(1-x)}\) 
\end{eser}

\begin{figure}[ht]
    \centering
    \incfig[.6]{esercizio2-1}
    \caption{esercizio2-1}
    \label{fig:esercizio2-1}
\end{figure}


\section{Tecniche elementari di Integrazione}
Ne conosciamo già diverse, come le equazioni a variabili separabili, le
equazioni lineari del primo ordine e secondo ordine. Cambiare poco nelle forme
porta a equazioni differenziali molto diverse e con soluzioni difficili.
Nell 1800 si è cercato di mettere ordine e di capire in quali casi si può solo
trovare un modello qualitativo (esempio trovare che Lotka-Volterra abbia le
orbite chiuse). E molto spesso anche se ci sono soluzioni a volte basta sapere
una soluzione approssimata, e quindi le tecniche esplicite di soluzione sono un
po' meh. E allora a cosa servono le tecniche esplicite di risoluzione? Boh
diciamo che quelle facili sono utili perché dai, servono (cit.) mentre altre più
complicate si fanno solo per vedere quanto è difficile risolvere equazioni
differenziali ma è simpatico quindi vederle come esempi.
\end{document}
