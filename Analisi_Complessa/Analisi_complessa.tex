\documentclass{article}
\usepackage{layout}
\usepackage[a4paper, total={5in,9in}]{geometry}
\usepackage[T1]{fontenc}
\usepackage[italian]{babel}
\usepackage{mathtools}
\usepackage{amsthm}
\usepackage[framemethod=TikZ]{mdframed}
\usepackage{amsmath}
\usepackage{amssymb}
\usepackage{cancel}
\usepackage{xcolor}
\usepackage{tikz}
\usepackage{tikz-cd}
\usepackage{pgfplots}
\pgfplotsset{compat=1.18}
\usepackage[many]{tcolorbox}
\usepackage{import}
\usepackage{pdfpages}
\usepackage{transparent}
\usepackage{xcolor}
\usepackage{enumitem}
\usepackage[colorlinks]{hyperref}

\newcommand*{\sminus}{\raisebox{1.3pt}{$\smallsetminus$}}

\newcommand*{\transp}[2][-3mu]{\ensuremath{\mskip1mu\prescript{\smash{\mathrm t\mkern#1}}{}{\mathstrut#2}}}%

% newcommand for span with langle and rangle around
\newcommand{\Span}[1]{{\left\langle#1\right\rangle}}

\newcommand{\incfig}[2][1]{%
    \def\svgwidth{#1\columnwidth}
    \import{./figures/}{#2.pdf_tex}
}

\pdfsuppresswarningpagegroup=1

\newcounter{theo}[section]\setcounter{theo}{0}
\renewcommand{\thetheo}{\arabic{section}.\arabic{theo}}

\newcounter{excounter}[section]\setcounter{excounter}{0}
\renewcommand{\theexcounter}{\arabic{section}.\arabic{excounter}}

\numberwithin{equation}{section}

\newenvironment{theorem}[1][]{
    \refstepcounter{theo}
     \ifstrempty{#1}
    {\mdfsetup{
        frametitle={
            \tikz[baseline=(current bounding box.east),outer sep=0pt]
            \node[anchor=east,rectangle,fill=blue!20,rounded corners=5pt]
            {\strut Teorema~\thetheo};}
        }
    }{\mdfsetup{
        frametitle={
            \tikz[baseline=(current bounding box.east),outer sep=0pt]
            \node[anchor=east,rectangle,fill=blue!20,rounded corners=5pt]
            {\strut Teorema~\thetheo:~#1};}
        }
    }
    \mdfsetup{
        roundcorner=10pt,
        innertopmargin=10pt,linecolor=blue!20,
        linewidth=2pt,topline=true,
        frametitleaboveskip=\dimexpr-\ht\strutbox\relax,
        % nobreak=false
    }
\begin{mdframed}[]\relax}{
\end{mdframed}}

% \newenvironment{definition}[1][]{
%     \refstepcounter{theo}
%      \ifstrempty{#1}
%     {\mdfsetup{
%         frametitle={
%             \tikz[baseline=(current bounding box.east),outer sep=0pt]
%             \node[anchor=east,rectangle,fill=violet!20,rounded corners=5pt]
%             {\strut Definizione~\thetheo};}
%         }
%     }{\mdfsetup{
%         frametitle={
%             \tikz[baseline=(current bounding box.east),outer sep=0pt]
%             \node[anchor=east,rectangle,fill=violet!20,rounded corners=5pt]
%             {\strut Definizione~\thetheo:~#1};}
%         }
%     }
%     \mdfsetup{
%         roundcorner=10pt,
%         innertopmargin=10pt,linecolor=violet!20,
%         linewidth=2pt,topline=true,
%         frametitleaboveskip=\dimexpr-\ht\strutbox\relax,
%         nobreak=true
%     }
% \begin{mdframed}[]\relax}{
% \end{mdframed}}

\newtcolorbox[auto counter, number within=section]{definition}[2][]{
    colframe=violet!0,
    coltitle=violet, % Title text color
    fonttitle=\bfseries, % Title font
    title={Definizione~\thetcbcounter:~#2}, % Title format
    sharp corners, % Less rounded corners
    boxrule=0pt, % Line width of the box frame
    toptitle=1mm, % Distance from top to title
    bottomtitle=1mm, % Distance from title to box content
    colbacktitle=violet!5, % Background color of the title bar
    left=0mm, right=0mm, top=1mm, bottom=1mm, % Padding around content
    enhanced, % Enable advanced options
    before skip=10pt, % Space before the box
    after skip=10pt, % Space after the box
    breakable, % Allow box to split across pages
    colback=violet!0,
    borderline west={2pt}{-5pt}{violet!40},
    #1
}

\newenvironment{lemmao}[1][]{
    \refstepcounter{theo}
     \ifstrempty{#1}
    {\mdfsetup{
        frametitle={
            \tikz[baseline=(current bounding box.east),outer sep=0pt]
            \node[anchor=east,rectangle,fill=green!20,rounded corners=5pt]
            {\strut Lemma~\thetheo};}
        }
    }{\mdfsetup{
        frametitle={
            \tikz[baseline=(current bounding box.east),outer sep=0pt]
            \node[anchor=east,rectangle,fill=green!20,rounded corners=5pt]
            {\strut Lemma~\thetheo:~#1};}
        }
    }
    \mdfsetup{
        roundcorner=10pt,
        innertopmargin=10pt,linecolor=green!20,
        linewidth=2pt,topline=true,
        frametitleaboveskip=\dimexpr-\ht\strutbox\relax,
        % nobreak=true
    }
\begin{mdframed}[]\relax}{
\end{mdframed}}

\theoremstyle{plain}
\newtheorem{lemma}[theo]{Lemma}
\newtheorem{corollary}{Corollario}[theo]
\newtheorem{proposition}[theo]{Proposizione}

\theoremstyle{definition}
\newtheorem{example}[excounter]{Esempio}

\theoremstyle{remark}
\newtheorem*{note}{Nota}
\newtheorem*{remark}{Osservazione}

\newtcolorbox{notebox}{
  colback=gray!10,
  colframe=black,
  arc=5pt,
  boxrule=1pt,
  left=15pt,
  right=15pt,
  top=15pt,
  bottom=15pt,
}

\DeclareRobustCommand{\rchi}{{\mathpalette\irchi\relax}} % beautiful chi
\newcommand{\irchi}[2]{\raisebox{\depth}{$#1\chi$}} % inner command, used by \rchi

% \newtcolorbox[auto counter, number within=section]{eser}[1][]{
%     title=Esercizio~\thetcbcounter,
%     label={eser:\thetcbcounter},
% 	% skin=bicolor,
% 	breakable,
% 	enhanced,
% 	frame hidden,
% 	boxrule = 0sp,
% 	borderline west = {1.5pt}{-5pt}{black},
% 	sharp corners,
% 	detach title,
% 	before upper = \tcbtitle\par\smallskip,
% 	coltitle = black,
% 	fonttitle = \bfseries\sffamily,
% 	description font = \mdseries,
% 	separator sign none,
% 	colback = black!0,
%     colbacktitle=black!5
% }
\newtcolorbox[auto counter, number within=section]{eser}[1][]{
    colframe=black!0,
    coltitle=black!70, % Title text color
    fonttitle=\bfseries\sffamily, % Title font
    title={Esercizio~\thetcbcounter~#1}, % Title format
    sharp corners, % Less rounded corners
    boxrule=0pt, % Line width of the box frame
    toptitle=1mm, % Distance from top to title
    bottomtitle=1mm, % Distance from title to box content
    colbacktitle=black!5, % Background color of the title bar
    left=0mm, right=0mm, top=1mm, bottom=1mm, % Padding around content
    enhanced, % Enable advanced options
    before skip=10pt, % Space before the box
    after skip=10pt, % Space after the box
    breakable, % Allow box to split across pages
    colback=black!0,
    borderline west={1pt}{-5pt}{black!70},
}
\newcommand{\seminorm}[1]{\left\lvert\hspace{-1 pt}\left\lvert\hspace{-1 pt}\left\lvert#1\right\lvert\hspace{-1 pt}\right\lvert\hspace{-1 pt}\right\lvert}


\title{Appunti di Analisi 3 \-- Analisi Complessa}
\author{Osea}
\date{Primo semestre A.A. 2024 \-- 2025, prof. Enrico Vitali}

\begin{document}

\maketitle

\section{Convergenza puntuale e uniforme}
Sia \(E\) un'insieme (non vuoto) e \(\{f_{n}\} \) una successione di funzioni
\(E \to \mathbb{R}\) (o \(E \to \mathbb{R}^{n}\) o \(E \to \mathbb{C}\)). Sia
\(f: E \to \mathbb{R}\).
\begin{definition}
    Diciamo che \(\{f_{n}\} \) converge \textbf{puntualmente} ad \(f\) se 
    \[
        \lim_{n \to \infty} f_{n}(x) = f(x) \quad \forall x \in E
    \]
\end{definition}
\begin{example}
    \(E = \mathbb{R}\) e \(f_{n}(x) = \frac{1}{n + x^2}\), \(f_{n} \to 0\) su
    \(\mathbb{R}\) 
\end{example}
\begin{example}
\(f_{n}(x) = {\left( x - \frac{1}{n} \right)}^2 \to x^2\)
\end{example}
\begin{example}
    \(f_{n}(x) = x^2 - \frac{1}{n}\) 
\end{example}
\begin{example}
    \(f_{n}(x) = e^{x - n}\) 
    \(f_{n} \to 0\) 
\end{example}

\begin{example}
    \(E = [0, 1]\), \(f_{n}(x) \) funzione che è a triangolo con vertici
    \((\frac{1}{4n}, 0), (\frac{1}{2n}, 1)\), \( (\frac{1}{n}, 0)\). Allora
    \(f_{n} \to 0\)
\end{example}

In questi esempi l'idea è che per ogni \(\varepsilon\) esiste un
\(n_\varepsilon\) tale che per \(n\ge n_\varepsilon\), \(f_{n}(x) <
\varepsilon\). La domanda è se si riesce a esprimere \(n_\varepsilon\) senza che
dipenda da \(x\).
Nell'esempio di \(f_{n}(x) = \frac{1}{n+x^2}\) si può perché \(f_{n}\) ha un
massimo in \(x = 0\), in tal caso infatti se prendo \(n_\varepsilon \) tale che
\(\frac{1}{n_\varepsilon} < \varepsilon\) allora \(\frac{1}{n+x^2}\le
\frac{1}{n} \le \frac{1}{n_\varepsilon} < \varepsilon\).

Nell'esempio 1.2 invece vogliamo un \(n_\varepsilon\) tale che \(\forall n\ge
n_\varepsilon\), \(|f_{n}(x) - f(x)| \le \varepsilon\) ossia \(|-\frac{2}{n} x +
\frac{1}{n^2} | \le  \varepsilon\). Da questo troviamo che
\[
    \frac{1}{n^2} - \varepsilon \le \frac{2x}{n} \le \frac{1}{n^2} + \varepsilon
\]
Ma è sempre possibile, per qualsiasi \(\frac{1}{n^2} + \varepsilon\) è
possibile trovare un \(x\) tale che sia maggiore, quindi non è possibile non
esprimere \(n_\varepsilon\) anche in funzione di \(x\).

\begin{definition}
    Sia \(f\), \(f_{n} : E \to \mathbb{R}\). Diciamo che \(f_{n} \to f\)
    uniformemente in \(E\) se:
    \[
        \forall \varepsilon > 0 \quad \exists n_\varepsilon \in \mathbb{N} :
        \forall n \ge n_\varepsilon, \forall x \in E, \quad |f_{n}(x) - f(x)| <
        \varepsilon
    \]
\end{definition}
\begin{remark}
    La condizione della definizione di convergenza uniforme è equivalente a
    richiedere che \(\sup_{x \in E} |f_n(x) - f(x)| < \varepsilon\). Da questo
    concludiamo che \(f_{n} \to f\) uniformemente se e solo se
    \[
        \lim_{n \to \infty} \sup_{x \in E} |f_{n}(x) - f(x)| <\varepsilon
    \]
\end{remark}
Allora con questa nuova osservazione è facile notare la non convergenza uniforme
dell'esempio 1.2. Infatti se \(f_{n}(x) = {(x-\frac{1}{n})}^2\) e \(f(x) = x^2\)
allora \(\sup_{x \in \mathbb{R}} |f_{n}(x) - f(x)| \ge |f_n(n) - f(n)| = |2 -
\frac{1}{n}| \to 2 > 0\).

Abbiamo però che converge uniformemente sugli insiemi limitati (esercizio).
Similmente nell'esempio 1.4 \(f_{n}\) converge uniformemente sugli insiemi
\((-\infty, a]\) infatti \(0 \le f_{n}(x) \le e^{a-n} \to 0\) per \(n \to +\infty\) 

Geometricamente la convergenza uniforme dice che il grafico di \(f_{n}\) è
contenuta in un intorno tubolare arbitrario di \(f\) per \(n\) sufficientemente
grande.

\begin{proposition}[Criterio di Cauchy / completezza di \(\mathbb{R}\) ]
    Se \(\{a_{n}\} \) è una successione di numeri reali si ha:
    \(a_{n}\) converge se e solo se \(a_{n}\) è una successione di Cauchy, ossia
    se \(\forall \varepsilon > 0 \exists n_\varepsilon\) tale che \(\forall
    n_{1},n_{2} \ge n_\varepsilon\), \(|a_{n_{1}} - a_{n2}| <\varepsilon\) 
\end{proposition}

\begin{theorem}[Criterio di Cauchy per la convergenza uniforme]
    Siano \(f, f_{n} : E \to \mathbb{R}\), con \(f_{n} \to f\) in \(E\). Allora la
    convergenza è uniforme in \(E\) se e solo se 
    \[
        \forall \varepsilon>0 \quad \exists n_\varepsilon \in \mathbb{N} : \forall n,m
        \ge n_\varepsilon \text{ e } \forall x \in E, \quad |f_{n}(x) -
        f_{m}(x)| < \varepsilon
    \]
\end{theorem}
\begin{proof}\( \)
\begin{itemize}
    \item[\(\implies \)] Sia \(f_{n} \to f\) uniformemente in \(E\). Fissato
        \(\varepsilon>0\), sia \(n_\varepsilon\) tale che (convergenza uniforme)
        \(\forall k \ge n_\varepsilon\) e \(\forall x \in E\), \(|f_k(x) - f(x)|
        < \frac{\varepsilon}{2}\) allora presi \(n,m \ge n_\varepsilon\) ho
        che \(|f_n(x) - f_m(x)| \le |f_n(x) - f(x)| + |f_m(x) - f(x)| <
        \frac{\varepsilon}{2} + \frac{\varepsilon}{2} = \varepsilon\) 
    \item[\(\impliedby \)] Valga la condizione di Cauchy. Allora \(\forall x \in
        E\) la successione \(\{f_{n}(x)\} \) è una successione di Cauchy, quindi
        è convergente, quindi \(\exists f : E \to \mathbb{R}\) tale che
        \(f_{n} \to f\). Allora dalla condizione di Cauchy, tenendo \(n\) fisso
        e facendo tendere \(n \to +\infty\) si ottiene esattamente la
        convergenza uniforme.
\end{itemize}
\end{proof}

Fun fact: esistono dei cosiddetti ``Spazi uniformi'', che sono spazi topologici
ma non metrici.

\begin{example}
    Sia \(f_{n} = \frac{n^2 - x}{n^3 + e^{nx}}\). È evidente per \(x \in
    \mathbb{R}\) che \(f_{n}(x) \to 0\).

    C'è convergenza uniforme sui limitati, infatti se \(|x| \le M\) allora
    \(|f_{n}(x)| \le \frac{n^2 + M}{n^3} \to 0\). Consideriamo ora \(x \ge  0\)
    (esercizio). Invece per \(x \le 0\), posso prendere per ogni \(n\) \(x_{n} =
    -n^{4}\) e allora ottengo che \(f_{n}(x_{n}) \to +\infty\) 
\end{example}

\begin{remark}
    Sia \(f_{n}: [a, b) \to  \mathbb{R}\) continua, suppongo che \(\{f_{n}\} \)
    converga uniformemente a \(f\) in \((a, b)\). Allora converge uniformemente
    in \([a, b)\) 
\end{remark}
\begin{proof}
    Per Cauchy
    \[
        \forall \varepsilon > 0 \quad \exists n_\varepsilon \in \mathbb{N} :
        \forall  n, m \ge n_\varepsilon, \forall x \in (a, b), \quad |f_{n}(x) -
        f_{m}(x)| \le \varepsilon
    \]

    Per \(x \to  a\) abbiamo per continuità che \(|f_{n}(a) - f_{m}(a)| \le
    \varepsilon\) per \(n, m \ge \overline{n} \in \mathbb{N}\), quindi preso
    \(\tilde{n} = \max{n_\varepsilon, \overline{n}}\) si ha che \(f_{n}\)
    soddisfa il criterio di Cauchy in \([a, b)\) e quindi converge
    uniformemente.
\end{proof}
Da questa osservazione noto anche che vale il contrapositivo: se \(f_{n}\) non
converge uniformemente in \([a, b)\) non può neanche convergere uniformemente in
\((a, b)\)
\begin{example}
    \(\displaystyle f_{n}(x) = \frac{1}{1 + {n^2 \left( x - \frac{q}{\sqrt{n}}
\right)}^2}\), allora ho che \(\displaystyle f_{n}(0) = \frac{1}{1+n} \to 0\), e
per \(x \neq 0\) pure, infatti
\[
    0 \le \frac{1}{1+ n^2 {\left( x - \frac{1}{\sqrt{n}} \right)}^2}
    \overset{\text{definitivamente}}{\le} \frac{1}{1 + n^2 {\left(
    \frac{x^2}{2} \right) }^2} \to 0
\]
è convergente uniformemente su tutto \(\mathbb{R}\)
\end{example}
Sia \(E\) un insieme non vuoto e sia \(\mathcal{B}(E)\) l'insieme delle funzioni
reali e limitate su \(E\).
\begin{definition}[Norma dell'estremo superiore]
    Sia \(f : E \to \mathbb{R}^{n}\) una funzione. Allora
    \[
        \|f \|_{\infty} := \sup_{x \in E} |f(x)| 
    \]
    è la norma dell'estremo superiore (anche denotata semplicemente \(\|f\|\)).
\end{definition}
\begin{proof}[Buona definizione]
Perché sia una buona definizione, serve che sia una norma. 
\begin{enumerate}[label = \alph*.]
    \item \(\|f\| \ge 0\) e \(\|f\| = 0 \iff f = 0\)
    \item \(\|\lambda f\| = |\lambda| \|f\|\)
    \item \(\|f + g\| \le \|f\| + \|g\|\)
\end{enumerate}
\end{proof}
\begin{proposition}
    \(\mathcal{B}(E)\) è uno spazio metrico normato con la norma dell'estremo
    superiore, e quindi distanza \(d(f, g) = \|f - g\|\) 
\end{proposition}
\begin{proof}
    ovvio
\end{proof}

\subsection{Scambi di limite, derivate, integrali}
\begin{example}
    Dimostrare che se \(f \in C^{0}([a, b] \times [c, d])\) a valori reali e 
    \[
        g(y) = \int_{a}^{b} f(x, y) \, dx \quad y \in [c, d]
    \]
    Allora \(g\) è continua in \([c,d]\) 

    Infatti \(\forall \overline{y} \in [c, d]\) abbiamo che comunque presa
    \(y_{n} \to \overline{y}\) chiaramente \(g(y_{n}) \to g(\overline{y})\).
    Ponendo ora \(f_{n} = f(\cdot , y_{n})\). Allora vogliamo mostrare che
    \(f_{n}(\cdot ) \to f(\cdot , \overline{y})\) uniformemente in \([a, b]\).
    Poiché \(f\) è uniformemente continua in \([a, b] \times [c, d]\) perché
    continua su un compatto, allora \(\forall \varepsilon > 0 \quad \exists
    \delta > 0\) tale che \(\forall x, x' \in [a, b]\) e \(\forall y, y' \in [c,
    d]\) se \(\sqrt{{(x - x')}^2 + {(y - y')}^2} < \delta\) allora \(|f(x, y) -
    f(x', y')| < \varepsilon\). Allora fissato \(\varepsilon>0\) sia \(\delta\)
    come sopra; sia quindi \(n_\varepsilon\) tale che \(n \ge n_\varepsilon
    \implies |y_{n} - \overline{y}| \le \delta\) e quindi, per ogni \(x \in [a,
    b]\) abbiamo \(|(x, y_{n}) - (x,
    \overline{y})| = |y_{n}-y| \le \delta\), da cui \(|f_{n}(x, y_{n}) - f(x,
    \overline{y})|\le \varepsilon\). Abbiamo quindi mostrato l'uniforme
    convergenza.
\end{example}
\begin{proposition}[Derivation under the integral sign]
    Sia \(f \in C^{1}([a,b] \times [c, d])\) e \(g(y) = \int_a^b f(x, y) dx\)
    per \(y \in [c, d]\), allora
    \[
        g \in C^{1}([c, d]) \text{ e } g'(y) = \int_a^b \frac{\partial
        f}{\partial y}(x, y) \, dx
    \]
\end{proposition}
\begin{proof}
    Fissiamo \(\overline{y} \in [c, d]\) e consideriamo 
    \[
        \frac{g(y) - g(\overline{y})}{y - \overline{y}} = \int_a^b
        \frac{f(x, y) - f(x, \overline{y})}{y - \overline{y}} \, dx = \int_a^b 
        \varphi(x, y) dx
    \]
    con \(\varphi(x, y)\) l'integrando.
    Siappiamo che \(\lim_{y \to \overline{y}} \varphi(x, y) = \frac{\partial
    f}{\partial y} (x, \overline{y})\) e vogliamo mostrare che questa
    convergenza è uniforme al variare di \(x\). Per il teorema di Lagrange si ha
    che 
    \[
        \varphi(x, y) = \frac{f(x, y) - f(x, \overline{y})}{y - \overline{y}} =
        \frac{\partial f}{\partial y} (x, \xi_{x, y}) \quad \xi_{x, y} \in
        (\overline{y}, y) \text{ oppure } (\overline{y}, y)
    \]
    Poiché \(\frac{\partial f}{\partial y}\) è uniformemente continua in \([a,
    b] \times [c, d]\) allora
    \begin{align*}
        \forall \varepsilon > 0 \quad \exists \delta > 0 : \forall x, x' \in [a,
        b] \text{ e } \forall y, y' \in [c, d]\\
        |(x, y) - (x', y')| < \delta \implies \left| \frac{\partial f}{\partial y}(x,
        y) - \frac{\partial f}{\partial y}(x', y') \right| < \varepsilon
    \end{align*}
    e ora prendiamo come coppie \((x, \overline{y})\) e \((x, \xi_{x, y})\) e
    abbiamo
    \[
        |(x, \xi_{x, y}) - (x, \overline{y})| = |\xi_{x, y} - \overline{y}| \le
        |y - \overline{y}|
    \]
    Ora come prima ciò dimostra che \(\varphi(x, y) \to \frac{\partial
    f}{\partial y}(x, \overline{y})\) uniformemente in \([a, b]\) e quindi
    \[
        \frac{d}{dy} \int_a^b f(x, y) \, dx = \int_a^b \frac{\partial
        f}{\partial y}(x, y) \, dx
    \]
\end{proof}

\subsection{Serie di funzioni}
I risultati visti per le successioni di funzioni danno luogo ad analoghi
risultati per le serie di funzioni.
Sia quindi \(E\) un insieme \(f_{n} : E \to \mathbb{R}\) (oppure
\(\mathbb{R}^{m}, \mathbb{C}\)) e si considera la serie 
\[
    \sum_{n=1}^{\infty} f_{n}(x)\quad x \in E
\]
che è una serie di funzioni.
\begin{definition}
    Diciamo che la serie \(\sum_{n=1}^{\infty} f_{n}(x)\) converge puntualmente
    in \(E\) se la successione delle somme parziali converge puntualmente in
    \(E\), ossia se 
    \[
        \lim_{N \to \infty} \sum_{n=1}^{N} f_{n}(x) = f(x) \quad \forall x \in E
    \]

    Diciamo che la serie \(\sum_{n=1}^{\infty} f_{n}(x)\) converge uniformemente
    in \(E\) se la successione delle somme parziali converge uniformemente in
    \(E\),
\end{definition}
Ne consegue che alcuni risultati hanno rispettivi analoghi, ad esempio
    \[
        \sum_{i=1}^{\infty} f_{n}(x) 
    \]
    converge uniformemente in \(E\) se e solo se 
    \[
        s_N(x) = \sum_{n=1}^{N} f_{n}(x)
    \]
    converge uniformemente in \(E\) (definizione), ossia questo vale se 
    \[
        \forall \varepsilon > 0 \quad \exists n_\varepsilon \in \mathbb{N} :
        \forall N, M \ge n_\varepsilon, \forall x \in E, \quad |s_N(x) - s_M(x)|
        < \varepsilon
    \]
    Ora assumiamo senza perdita di generalità che \(N \le  M\), allora chiamiamo
    \(M = N + p\) e otteniamo che l'ultima eguaglianza si scrive come
    \[
        \left| \sum_{n=N+1}^{N+p} f_{n}(x) \right| < \varepsilon
    \]
Otteniamo
\begin{proposition}[Criterio di Cauchy]
    La serie \(\sum_{n=1}^{\infty} f_{n}(x)\) converge uniformemente in \(E\) se
    e solo se 
    \begin{equation}
        \forall \varepsilon > 0 \quad \exists n_\varepsilon \in \mathbb{N} :
        \forall n, p \ge n_\varepsilon, \forall x \in E, \quad \left|
        \sum_{k=n+1}^{n+p}
        f_{k}(x) \right| < \varepsilon
    \end{equation}
\end{proposition}
\begin{corollary}
    Condizione necessaria affinche la serie \(\sum_{n=1}^{\infty} f_{n}(x)\)
    converga uniformemente in \(E\) è che \(f_{n} \to 0\) uniformemente in \(E\)
\end{corollary}
\begin{proof}
    prendiamo \(p=1\) in (1) e otteniamo
    \(
        \left| f_{n+1}(x) \right| < \varepsilon
    \)
    ossia \(f_{n} \to 0\) uniformemente in \(E\) 
\end{proof}

\begin{example}
Supponiamo ora che esista una successione numerica \({\{a_{n}\}}_{n \in
\mathbb{N}}  \) tale che
\begin{itemize}
    \item \(|f_{n}(x)| \le a_{n}\) per ogni \(x \in E\) 
    \item \(\sum_{n=1}^{\infty} a_{n} < +\infty\)
\end{itemize}
vogliamo mostrare che allora la serie \(\sum_{n=1}^{\infty} f_{n}(x)\) converge
uniformemente in \(E\), usando \((1)\), infatti abbiamo 
\[
    \left| \sum_{k=n+1}^{n+p} f_{k}(x) \right| \le \sum_{k=n+1}^{n+p} |f_{k}(x)|
    \le \sum_{k=n+1}^{n+p} a_{k} < \varepsilon
\]
dove nell'ultima diseguaglianza si è utilizzato il criterio di Cauchy per le
serie numeriche.
\end{example}
\begin{definition}[Convergenza totale]
    Si dice che la serie \(\sum_{n=1}^{\infty} f_{n}(x)\) converge
    \textbf{totalmente} in \(E\) se esiste \(\{a_{n}\} \) in \(\mathbb{R}\) tale
    che 
    \begin{itemize}
        \item \(|f_{n}(x)| \le a_{n}\) per ogni \(x \in E\) 
        \item \(\sum_{n=1}^{\infty} a_{n} < +\infty\)
    \end{itemize}
\end{definition}
Per quanto visto prima quindi
\begin{proposition}
    Convergenza totale implica convergenza uniforme, e notando dalla
    dimostrazione prima abbiamo anche che implica la convergenza assoluta
    uniforme.
\end{proposition}

\begin{example}
Non vale il contrario, un esempio di serie uniformemente convergente ma non
totalmente convergente è
\[
\sum_{n=1}^{\infty} -1 \cdot \frac{{(-1)}^{n}}{n}
\]
dove \(f_{n}(x)\) è costante per ogni \(n\). Allora la serie converge
uniformenenente in \(\mathbb{R}\) ovviamente perché è costante e converge in
quanto a segno alternato, ma non converge totalmente perché la serie armonica
diverge.
\end{example}
\begin{example}
    Sia \(f_{n}(x) = {(-1)}^{n+1} \frac{x^{n}}{n}\), per \(x \in \mathbb{R}\).
    Allora usiamo il criterio della radice ottenendo
    \[
        \lim_{n \to \infty} \sqrt[n]{|f_{n}(x)|} = \lim_{n \to \infty}
        \frac{|x|}{\sqrt[n]{n}} = |x|
    \]
    quindi per \(|x| < 1\) la serie converge assolutamente, per \(|x| > 1\) la
    serie diverge, per \(x = -1\) la serie è la serie armonica che diverge, per
    \(x = 1\) la serie è una serie a segni alterni che converge.

    Concludiamo quindi che la serie converge puntalmente in \((-1, 1]\) e per
    ogni \(0 < \delta < 1\) la serie converge uniformemente in \([-\delta,
    \delta]\), infatti 
    \[
        \left| {(-1)}^{n+1} \frac{x^{n}}{n} \right| \le \delta^{n}{n}
    \]
    la cui serie converge, quindi la serie converge totalmente.

    Naturalmente però la serie non converge totalmente in \([0, 1]\) poiché
\[\max_{x \in [0, 1]} |f_{n}(x)| = \frac{1}{n}\] ma comunque la serie converge
uniformemente in \([0, 1]\), infatti usiamo il criterio di Cauchy.
\[
    \left| \sum_{k=n+1}^{n+p} f_{k}(x) \right| \le |s_{n+p}(x) - s_{n}(x)| <=
    |s_{n+p} - S(x)| + |S(x) - s_{n}(x)| < \frac{x^{n+p}}{n+p} + \frac{x^{n}}{n}
\]
che converge a \(0\) per \(n\to +\infty\)
e si è usato il fatto che se \(S = \sum_{i=1}^{\infty} {(-1)}^{n+1} a_{n} \)
è una serie convergente a segni alterni, con \(a_{n} > 0, a_{n} \to 0\) allora
\(|S - s_{n}| \le a_{n+1} \) 

Procediamo a chiederci se la serie converge uniformemente in \((-1, 0]\).
Utilizziamo allora la seguente osservazione dedotta direttamente dalle
successioni
\begin{remark}
    Sia \(\sum_{i=1}^{\infty} f_{n}(x) \), \(f_{n} \in C^{0}([a, b])\). Se la
    serie converge uniformemente in \((a, b]\) allora converge uniformemente in
    \([a, b]\) (in particolare converge in \(x=a\))
\end{remark}
\begin{proof}
    Per ipotesi \(s_{n}(x)\) converge uniformemente in \((a, b]\) e \(s_{n}\)
    sono funzioni continue in \(x = a\), quindi per il risultato che avevamo già
    per le successioni (in breve basta enunciare il criterio di Cauchy e usare
la continuità in \(x=a\)) otteniamo che la serie converge uniformemente in \([a,
b]\)
\end{proof}
Ne concludiamo che la serie non può convergere uniformemente in \((-1, 0]\)
altrimenti convergerebbe uniformemente in \([-1, 0]\) ma sappiamo che in \(-1\)
non abbiamo neanche convergenza puntuale.
\end{example}

\begin{example}
    Studiare la convergenza puntuale e uniforme di
    \[
        \sum_{n=1}^{\infty} \frac{x}{{(1+x)}^{n}} 
    \]
\end{example}

\section{Richiami su limiti e serie}
\begin{proposition}
    Sia \(a_{n}\) una successione di numeri reali positivi. Allora 
    \[
       \liminf_{n \to +\infty} \frac{a_{n+1}}{ a_{n} } \le \liminf_{n \to
       +\infty} \sqrt[n]{a_{n}} \le \limsup_{n \to +\infty} \sqrt[n]{a_{n}} \le
       \limsup_{n \to +\infty} \frac{a_{n+1}}{a_{n}}
    \]
\end{proposition}
\begin{proof}
    Sia \(L = \limsup \frac{a_{n+1}}{a_{n}}\). Se \(L=+\infty\) non c'è nulla da
    dimostrare. Sia allora \( L < +\infty\). Fissato un \(\varepsilon>0\) quindi
    esiste \(n_{\varepsilon} \) tale che \(\forall n \ge n_{\varepsilon}\) si ha 
    \[
        \frac{a_{n+1} }{a_{n}} \le  L + \varepsilon
    \]
    Allora iterando otteniamo
    \[
        a_{n} \le (L + \varepsilon)^{n - n_{\varepsilon}} a_{n_{\varepsilon}}
        \implies \sqrt[n]{a_{n}} \le {(L + \varepsilon)}^{1 -
        \frac{n_{\varepsilon}}{n}} \sqrt[n]{a_{n_{\varepsilon}}} \to L +
        \varepsilon 
    \]
    Per \(n \to \infty\), ora per l'arbitrarietà di \(\varepsilon > 0\)
    otteniamo
    \[
        \limsup_{n \to \infty} \sqrt[n]{a_{n}} \le L = \limsup_{n \to \infty}
        \frac{a_{n+1}}{a_{n}}
    \]
    Similmente si dimostra anche l'altra uguaglianza, quella centrale è ovvia.
\end{proof}
\begin{example}
    Sia \(a_{n} = n\). Allora poiché \(\frac{a_{n+1}}{a_{n}} \to 1\) abbiamo che
    anche \(\sqrt[n]{n} \to 1\).

    Sia \(a_{n} = n!\). Allora \(\frac{a_{n+1}}{a_{n}} = n+1 \to +\infty\) e
    quindi anche \(\sqrt[n]{n!} \to +\infty\)

    Sia \(a_{n} = \frac{n^{n}}{n!}\) allora
    \[
        \frac{a_{n+1}}{a_{n}} = \frac{{(n+1)}^{n+1}}{(n+1)!} \frac{n!}{n^{n}} =
        \frac{{(n+1)}^{n}}{n^{n}} = {\left( 1 + \frac{1}{n} \right)}^{n} \to e
    \]
    e quindi \(\displaystyle \sqrt[n]{\frac{n^{n}}{n!}} =
    \frac{n}{\sqrt[n]{n!}}\to e\).

\end{example}
\begin{remark}
    In realtà (e potremmo vederlo più tardi), l'approssimazione di Stirling ci
    dice
    \[
    n! \sim \sqrt{2\pi n} {\left( \frac{n}{e} \right)}^{n}
    \]
\end{remark}

Ora procediamo vedendo un criterio di convergenza (non assoluta) che sarà il
criterio di convergenza di Abel. Procediamo a passi più piccoli.
\begin{lemmao}[Disuguaglianza di (Brunacci) Abel]
    Siano \(\gamma_0, \gamma_{1}, \dots, \gamma_{\ell} \in \mathbb{C}\) e siano
    \(\zeta_{0}, \zeta_{1}, \dots, \zeta_{\ell} \in \mathbb{C} \). Poniamo ora 
    \[
        w_m = \sum_{i=0}^{m} \zeta_i \quad m = 0, 1, \dots, \ell
    \]
    Sia \(M > 0\) tale che 
    \[
        |w_m| \le M \quad \forall m = 0, 1, \dots, \ell
    \]
    Allora 
    \[
        \left| \sum_{i=0}^{\ell} \gamma_i \zeta_i \right| \le (|\gamma_0 -
        \gamma_{1}| + |\gamma_1 - \gamma_{2}| +~\dots + |\gamma_{\ell-1} -
        \gamma_{\ell}| + |\gamma_{\ell}|)M
    \]
\end{lemmao}
\begin{proof}
    \[
        \sum_{i=0}^{\ell} \gamma_{i} \zeta_{i} = \sum_{i=0}^{\ell} \gamma_{i}
        (w_{i} - w_{i-1}) = \sum_{i=0}^{\ell} (\gamma_{i} - \gamma_{i+1})w_{i}
    \]
    Dove si intende che \(\gamma_{\ell+1} = 0\) e \(w_{-1} = 0\). Ora
    semplicemente per disuguaglianza triangolare e applicando l'ipotesi
    definente \(M\) otteniamo la tesi.
\end{proof}

\begin{theorem}[primo criterio di convergenza di Abel]
    Sia \(\sum_{n=1}^{\infty} c_{n} z_{n}\) una serie numerica. Se 
    \begin{itemize}
        \item \(z_{n} \in \mathbb{C}\) (oppure in \(\mathbb{R}^{N}\))
        \item \(\sum_{n=1}^{\infty} z_{n}\) è una serie le cui somme parziali
            sono limitate
        \item \(\{c_{n}\}\) è una successione di numeri reali non creascente e
            infinitesima
    \end{itemize}
    allora la serie \(\sum_{n=1}^{\infty} c_{n} z_{n}\) converge.
\end{theorem}

\begin{proof}
    Utilizziamo il criterio di convergenza di Cauchy. Fissiamo \(N, p \in
    \mathbb{N}\) e consideriamo
    \[
        \left| \sum_{n=N}^{N+p} c_{n} z^{n} \right| \le 2M\sum_{n=N}^{N+p} |c_{n}
        - c_{n+1}| \text{ (diciamo \(c_{N+p+1} = 0 \) per comodità notazionale) }
    \]
    Dove \(M\) è maggiorante per le somme parziali di \(z_{n}\). Infatti abbiamo
    che 
    \[
        w_m = z_N + z_{N+1} +~\dots + z_{N+m} = \sum_{i=0}^{N+m} z_{n} -
        \sum_{i=0}^{N-1} z_{n}
    \]
    per cui effettivamente \(|w_m| \le 2M\) e possimao applicare la
    disuguaglianza di Abel.
    
    Ora possiamo, sapendo che \(c_k \to 0\) da sopra, ottenere che la serie
    precedentemente trovata è telescopica per \(N\) sufficientemente grande e
    quindi
    \[
        \left| \sum_{n=N}^{N+p} c_{n} z^{n} \right| \le 2M|c_{N}|\to 0 \text{ per
        } N \to +\infty
    \]
    e quindi per il criterio di Cauchy la serie converge.
\end{proof}
\begin{example}
    Consideriamo la serie
    \[
    \sum_{n=1}^{\infty} {(-1)}^{n+1} \frac{z^{n}}{n} \quad z \in \mathbb{C}
    \]
    Allora se \(|z| > 1\) manca la condizione necessaria di convergenza.

    Sia allora \(|z| \le  1\). Consideriamo prima il caso \(|z| < 1\). Sia ha
    allora convergenza assoluta, perché
    \[
        \frac{|z|^{n+1}}{n+1} \cdot \frac{n}{|z|^{n}} =  |z| \frac{n}{n+1} \to 0
    \]

    Consideriamo invine il caso \(|z| = 1\). Se \(z = -1\) la serie non
    converge:
    \[
        \sum_{n=1}^{\infty} \frac{{(-1)}^{n+1} \cdot {(-1)}^{n}}{n} = -
        \sum_{i=1}^{\infty} \frac{1}{n}  \to -\infty
    \]
    Ora consideriamo \(|z|=1\) con \(z\neq-1\) e vogliamo applicare il criterio
    di Abel, con \(c_{n} = \frac{1}{n}\) e \(z_{n} = {(-1)}^{n+1} z^{n}\).
    Chiaramente \(c_{n}\) è infinitesima non crescente reale. Inoltre
    \[
        |z_{1} + z_{2} + \dots + z_N| = \left| \sum_{i=1}^{N} {(-1)}^{n+1} z^{n}
        \right| = \left| \sum_{n=1}^{N} (-z)^{N} \right| = |z| \left| \frac{1 -
        {(-z)}^{N}}{1 - (-z)} \right| \le \frac{z}{|1+z|}
    \]
    Quindi sono soddisfatte le ipotesi del criterio di Abel e la serie converge.

\begin{figure}[ht]
    \centering
    \incfig[.4]{esempio2}
    \caption{Punti del piano \(\mathbb{C}\) tali che la serie dell'esempio 2.2
    converge}\label{fig:esempio2}
\end{figure}
\end{example}
\begin{corollary}[Criterio di Leibniz]
    Se una serie è a segni alterni del tipo 
    \[
        \sum_{n=1}^{\infty} {(-1)}^{n} a_{n}
    \]
    con \(a_{n} \to 0\) e \(a_{n} > 0\) non crescente. Allora abbiamo che
    \(z_{n} = {(-1)}^{n}\) e \(c_{n} = a_{n}\) soddisfano le ipotesi del
    criterio di Abel e quindi la serie converge.
\end{corollary}
\begin{theorem}[Secondo criterio di convergenza di Abel]
    Si consideri la serie \[
        \sum_{n=0}^{\infty} c_{n} z_{n} 
    \]
    con
    \begin{itemize}
        \item \(z_{n} \in \mathbb{C}\) (oppure in \(\mathbb{R}^{N}\))
        \item \(\sum_{n=0}^{\infty} z_{n}\) è una serie convergente
        \item \(\{c_{n}\}\) è una successione monotona e convergente
    \end{itemize}
    Allora la serie \(\sum_{n=0}^{\infty} c_{n} z_{n}\) converge.
\end{theorem}
\begin{proof}
    Supponiamo \(c_{n} \to c\) non crescente. Allora
    \[
        \sum_{n=0}^{N} c_{n} z_{n} = \sum_{n=0}^{N} (c_{n} - c)z_{n} + c
        \sum_{n=0}^{N} z_{n}
    \]
    Ora per \(N \to \infty\) abbiamo che \(c_{n} - c \to 0\) decrescente e le
    somme parziali di \(z_{n}\) sono limitate, perché la serie converge. Quindi
    abbiamo che la prima serie converge per il primo criterio di Abel. 
    Anche la seconda serie converge per ipotesi, quindi la tesi è dimostrata.
\end{proof}

Sappiamo che 
\[
    \left( \sum_{n=0}^{N} a_{n} z^{n}  \right) \left( \sum_{n=0}^{M}
        b_{n} z^{n}
    \right) = \sum_{n=0}^{N+M} \sum_{k=0}^{n} a_{k} b_{n-k}  z^{n}
\]
è il prodtto di polinomi. Quindi formalmente, se \(z = 1\) e \(N, M \to \infty\) otteniamo

\begin{definition}[Serie prodotto alla Cauchy]
\[
    \left( \sum_{n=0}^{\infty} a_{n}  \right) \left( \sum_{n=0}^{\infty} b_{n}
    \right) := \sum_{n=0}^{\infty} \sum_{k=0}^{n} a_{k} b_{n-k}  
\]
che è detta serie prodotto alla Cauchy delle serie \(\sum_{n=0}^{\infty} a_{n}\)
e \(\sum_{n=0}^{\infty} b_{n} \) 
\end{definition}

\begin{theorem}[Mertens + Cauchy]
    Se le serie \(\sum_{n=0}^{\infty} a_{n}\) e \(\sum_{n=0}^{\infty} b_{n} \)
    sono convergenti e almeno una è assolutamente convergente, allora la serie
    prodotto è convergente ed ha per somma il prodotto delle serie.
    
    Se entrambe le serie sono assolutamente convergenti, allora tale è anche la
    serie prodotto.
\end{theorem}

Consideriamo serie della forma 
\[
\sum_{n=0}^{\infty} c_{n} {(z - a)}^{n} \text{ con } c_{n}\in \mathbb{C} \text{
e } z, a \in \mathbb{C}
\]
Abbiamo visto alcuni esempi:
\begin{enumerate}[label = \alph*)]
    \item \(\sum_{n=0}^{\infty} z^{n} \) converge se e solo se \(|z| < 1\)
    \item \(\sum_{n=0}^{\infty} {(-1)}^{n+1} \frac{z^{n}}{n} \), converge se e
        solo se \(|z| \le 1\) e \(z \neq -1\)  
\end{enumerate}
In entrambi i casi (e vedremo in generale) la convergenza è nei punti di un
disco (detto cerchio di convergenza) di centro \(z = a\). Il comportamento sul
bordo del cerchio varia da caso a caso.

\begin{theorem}[Abel]
    Si consideri la seire di potenze 
    \[
    \sum_{n=0}^{\infty} c_{n} {\left( z-a \right)}^{n}
    \]
    Se la serie converge in un punto \(z \in \mathbb{C}\) allora converge
    uniformemente su tutto il segmento di estremi \(a\) e \(z\).
\end{theorem}
\begin{figure}[ht]
    \centering
    \incfig[.3]{abel}
    \caption{abel}
    \label{fig:abel}
\end{figure}
\begin{proof}
    Il teorema è significativo quando \(z_{1} \in \partial D_R(a) \) 
    Non è restrittivo supporre \(a =0\). Consideriamo
    \[
        \sum_{n=0}^{\infty} c_{n} z^{n}_t = \sum_{n=0}^{\infty} c_{n}
        {(tz_{1})}^{n} 
    \]
    utilizziamo il criterio di Cauchy per le convergenze uniformi: fissiamo \(
    \varepsilon >0\),
    vogliamo avere che per un \(n_\varepsilon\) allora per ogni \(N \ge
    n_\varepsilon\), \(p \in \mathbb{N}\) e \(t \in [0, 1]\) si abbia che
    \begin{align*}
        \left|\sum_{m=N}^{N+p} \underbrace{t^{n}}_{\gamma_{m-N} }
        \underbrace{c_{n}z_{1}^{n}}_{\zeta_{m-N}} \right| &< \varepsilon \\
 &\le  M(|\gamma_0 - \gamma_{1}|
        + |\gamma_1 - \gamma_{2}| +~\dots + |\gamma_{p-1} - \gamma_{p}|) 
    \end{align*}
    con \(M\) un maggiorante per le somme parziali di \(c_{n}z_{1}^{n}\). Ora
    poiché per ipotesi tale serie converge, esiste \(n_\varepsilon\) tale per
    cui per ogni \(N \ge n_\varepsilon\) e per ogni \(p \in \mathbb{N}\) si ha
    che
    \[
        \left|\sum_{m=N}^{N+p} c_{m}z_{1}^{m} \right| \le \varepsilon
    \]
    ora poiché \(1 \ge t ^{n}\) per ogni \(n \in \mathbb{N}\) abbiamo che la
    precedente disugaglianza è soddisfatta per \(M = 1\), quindi
    \[
        \left| \sum_{n=N}^{N+p} t ^{n} c_{n} z_{1}^{n} \right| \le t ^{N}
        \varepsilon \le  \varepsilon
    \]
\end{proof}

\begin{definition}
    Sia \(I\) un intervallo aperto. Diciamo che una funzione \(f: I \to
    \mathbb{R}\) è analitica se per ogni \(x_{0} \in I\) esiste \(\delta > 0\)
    tale che su \((x_{0} - \delta, x_{0} + \delta)\) la funzione sia esprimibile
    come somma di una serie di potenze 
    \begin{equation}
        f(x) = \sum_{n=0}^{\infty} a_{n} {(x - x_{0})}^{n} 
    \end{equation}
\end{definition}
Sia \(f\) come in \((2)\); Sia \(R\) il raggio di convergenza della serie. Su
ogni intervallo \(J\) tale che \(\overline{J} \subseteq (x_{0} - R, x_{0} + R)
\) sappiamo che la serie converge totalmente. Consideriamo la serie delle
derivate (cioè la \emph{serie derivata})
\[
    \sum_{n=1}^{\infty} n a_{n} {(x - x_{0})}^{n-1} 
\]
è una serie di potenze con raggio di convergenza \(R\), poiché 
\[
    \limsup_{n \to \infty} \sqrt[n]{n|a_{n}|} = \limsup_{n \to \infty}
    \sqrt[n]{|a_{n}|} \cdot \underbrace{\limsup_{n \to \infty} \sqrt[n]{n}}_{=1}
\]
Quindi la Erie delle derivate è uniformemente convergente su ogni compatto di
\(x_{0}-R, x_{0}+R\). 
\begin{lemma}[Teorema di derivazione per Serie]
    Sia \(f = \sum f_{n}\) convergente e \(g = \sum f_{n}' \) uniformemente
    convergente. Allora \(f\) è derivabile e \(f' = g\).
\end{lemma}
Per il teorema di derivazione per serie, \(f\) è derivabile e
\[
    f'(x) = \sum_{n=1}^{\infty} n a_{n} {(x - x_{0})}^{n-1} \quad \forall x \in
    (x_{0} - R, x_{0}+R)
\]
A \(f'\) applichiamo lo stesso ragionamento visto su \(f\): \(f'\) è derivabile
e si ha che 
\[
    f''(x) = \sum_{n=2}^{\infty} n(n-1) a_{n} {(x - x_{0})}^{n-2} \quad \forall
    x \in (x_{0} - R, x_{0}+R)
\]
Procedendo induttivamente otteniamo che \(f \in C^{\infty}(x_{0} - R, x_{0}+R)\)
e
\[
    f^{(k)}(x) = \sum_{n=k}^{\infty} n(n-1) \dots (n-k+1) a_{n} {(x -
    x_{0})}^{n-k}
\]
In particolare abbiamo che \(\displaystyle f^{(k)}(x_{0}) = k! a_k\) e quindi la
serie di potenze è la serie di Taylor di \(f\) centrata in \(x_{0}\). Più
precisamente
\begin{theorem}
    Sia \(\displaystyle f(x) = \sum_{n=0}^{\infty} a_{n} {(x-x_{0})}^{n} \) per
    \(x \in (x_{0} -R, x_{0} + R)\), con \(a_{n} \in \mathbb{R}\) e \(x_{0} \in
    \mathbb{R}\). Allora \(f \in C^{\infty}(x_{0}-R, x_{0}+R)\) e la serie è la
    serie di Taylor di \(f\) centrata in \(x_{0}\).
\end{theorem}
\begin{proof}
    Vedasi sopra.
\end{proof}
Non è vero che ogni funzione \(C^{\infty}\) sia sviluppabile in serie di Taylor.
\begin{example}
    Sia \(f(x) = e^{-\frac{1}{x^2}}\) per \(x \in \mathbb{R} \sminus \{0\} \) e \(f(0)
    = 0\). Allora 
    \[
        f'(x) = \begin{cases}
            \lim_{x \to 0} \frac{f(x)}{x} = \frac{1}{x} e^{-\frac{1}{x^2}} = 0  & x = 0 \\
            \frac{2}{x^3} e^{-\frac{1}{x^2}} & x \neq 0
        \end{cases}
    \]
    eccetera anche per le altre derivate si ha che \(f^{(k)}(0) = 0\). Quindi la
    serie di Taylor centrata in \(0\) è la serie nulla, ma \(f \neq 0\) in alcun
    intorno di \(0\).
    \begin{figure}[ht]
        \centering
        \begin{tikzpicture}
            \begin{axis}[
                xmin= -4, xmax= 4,
                ymin= -1, ymax = 2,
                axis lines = middle,
            ]
            \addplot[domain=-4:4, samples=100]{ e^(-(1/x^2))};
            \end{axis}
        \end{tikzpicture}
        \caption{\(\displaystyle e^{-\frac{1}{x^2}}\)}
    \end{figure}
\end{example}
\begin{theorem}
    Sia \(f \in C^{\infty}(I)\) con \(I \subseteq \mathbb{R} \) un intervallo
    aperto per la quale esistano \(M, L > 0\) tali che per ogni \(k \in \mathbb{N}\) 
    \[
        \forall  x \in I \quad |f^{(k)}(x)| \le M L^{k}
    \]
    Allora \(f\) è analitica.
\end{theorem}
\begin{proof}
    Sia \(x_{0} \in I\) e consideriamo
    \[
        \sum_{n=0}^{\infty} \frac{f^{(k)} (x_{0})}{k!} {(x - x_{0})}^{k} \quad x
        \in I
    \]
    Scriviamo lo sviluppo di Taylor con il resto di Lagrange.
    \[
        f(x) = \sum_{k=0}^{k} \frac{f^{(n)}}{k!}  
        {(x - x_{0})}^{k} + \frac{1}{(n+1)!} f^{(n+1)}(\xi_x) {(x - x_{0})}^{n+1}
    \]
    dove \(\xi_x \in (x_{0}, x)\) è un opportuno punto.
    Mostriamo ora che  
    \[
        \left| \frac{1}{(n+1)!} f^{(n+1)}(\xi_x) {(x - x_{0})}^{n+1} \right| \le
    \frac{M L^{n+1}}{(n+1)!} {(x - x_{0})}^{n+1} \to 0 \text{ per } n \to
    \infty
    \]
\end{proof}
\begin{example}
    Le funzioni \(e^{x}, \sin x, \cos x\) sono analitiche.
\end{example}

\subsection{\(\mathbb{C}\)-differenziabilità}
Sia \(\Omega \subseteq \mathbb{C}\) aperto, \(f : \Omega \to \mathbb{C}\) 
\begin{definition}[\(\mathbb{C}-\)differenziabilità]
    Sia \(a \in \Omega\). Diciamo che \(f\)  è \(\mathbb{C}-\)differenziabile in
    \(z=a\) se esiste
    \begin{equation}
        \lim_{z \to a} \frac{f(z) = f(a)}{z - a} = f'(a)
    \end{equation}
    o equivalentemente
    \begin{align}
        f(z) &= f(a) + f'(a) (z-a) + (\varepsilon(z-a))(z-a) \\
        \lim_{w \to 0} \varepsilon(w) &= 0
    \end{align}
\end{definition}
Se poniamo \(\varepsilon(0) = 0\) allora la \((1)'\) vale per ogni \(z \in
\Omega\), non solo \(z\neq a\) 

Alcune proprietà:
\begin{itemize}[label = --]
    \item Se \(f\) è \(\mathbb{C}\)-differenziabile in \(z = a\) allora è
        continua (da \((1)'\))
    \item \(f, g\) \(\mathbb{C}-\)differenziabile in \(z=a\); allora \(f \pm g\)
        è \(\mathbb{C}-\)differenziabile, \(\lambda f\), con \(\lambda \in
        \mathbb{C}\) è \(\mathbb{C}-\)differenziabile e \(fg\) è
        \(\mathbb{C}-\)differenziabile.

        Se \(g(a) \neq 0\) allora \(\frac{f}{g}\) è
        \(\mathbb{C}-\)differenziabile in \(z=a\) e \({\left( \frac{f}{g}
        \right)}' = \frac{f'g -
        fg'}{g^2}\) 
\end{itemize}
\begin{example}
    \(z \mapsto z\) è \(\mathbb{C}-\)differenziabile in ogni \(z \in
    \mathbb{C}\). Ne consegue dalle proprietà che i polinomi sono
    \(\mathbb{C}-\)differenziabili, e anche le funzioni razionali.
\end{example}
\begin{example}
    \(z \mapsto \overline{z}\) \textbf{non} è \(\mathbb{C}\)-differenziabile.
    Infatti
    \[
        \frac{f(z) - f(a)}{z-a} = \frac{\overline{z} - \overline{a} }{z-a} =
        \frac{\overline{z-a}}{z-a}
    \]
    che non ha limite perché assume valori diversi ad esempio sulla retta \(a +
    \delta \) e \(a + \delta i\) al variare di \(\delta \in \mathbb{R}\).
\end{example}

Una funzione \(f : \Omega \to \mathbb{C}\) può essere vista come \(f : \Omega
\subseteq \mathbb{R}^2 \to \mathbb{R}^2 \) tralasciando la struttura di campo di
\(\mathbb{C}\). Allora possiamo scrivere \(f(x, y) = (u{\left( x, y \right)} ,
v{\left( x, y \right)} ) \in \mathbb{R}^2\), con \(u, v : \Omega \subseteq
\mathbb{R}^2 \to \mathbb{R} \). Si utilizza spesso la scrittura
\[
    f(x, y) = u(x,y) + iv(x,y)
\]
che è una sorta di ``ibrido''.
Possiamo ora scrivere \(\frac{\partial f}{\partial x}\), \(\frac{\partial
f}{\partial y}\), \(\frac{\partial u}{\partial x}\), \(\frac{\partial
v}{\partial y}\) ecc.

Supponiamo ora che \(f\) sia \(\mathbb{C}-\)differenziabile in
\(z=a=x_{0}+iy_{0}\). Esiste quindi
\[
    \lim_{z\to a}  \frac{f(z) - f(a)}{z-a} = f'(a)
\]
Guardiamo ora la retta \(z=a+\delta\), con \(\delta \in \mathbb{R}\), quindi
\[
    f'(a) = \lim_{\delta \to 0} \frac{f(x_{0})+\delta, y_{0}) -
    f(x_{0},y_{0})}{\delta} = \frac{\partial f}{\partial x}(x_{0},y_{0}) =
    \frac{\partial f}{\partial x}(a)
\] 
In maniera analoga, per \(z= a + \delta i\) abbiamo
\[
    f'(a) = \lim_{\delta \to 0} \frac{f(x_{0}, y_{0}+\delta i) -
    f(x_{0},y_{0})}{\delta i} = \frac{1}{i} \frac{\partial f}{\partial
y}f(x_{0},y_{0}) = -i \frac{\partial f}{\partial y}f(a)
\]
In breve abbiamo che deve essere 
\[
    \frac{\partial f}{\partial x}(a) = -i \frac{\partial f}{\partial y}(a)
\]
Che in termini di \(u\) e \(v\) equivale a dire che
\[
    u_x + iv_x = -i {\left( u_y + iv_y \right)} = v_y - iu_y \iff \begin{cases}
        u_x &= v_y \\
        u_y &= -v_x
    \end{cases}
\]
\begin{proposition}[Condizioni necessarie]
    Se \(f\) è \(\mathbb{C}-\)differenziabile in \(z=a\) allora valgono le
    condizioni di \textbf{Cauchy-Riemann}, cioè
    \[
        \frac{\partial f}{\partial x} = -i \frac{\partial f}{\partial y} \text{
        o equivalentemente } \begin{cases}
        u_x &= v_y \\
        u_y &= -v_x
        \end{cases}
    \]
\end{proposition}
\begin{proof}
    Vedasi sopra.
\end{proof}
\begin{proposition}
    Sia \(f\) differenziabile in \(a = (x_{0}, y_{0})\) come funzione \(\mathbb{R}^2
    \supseteq \Omega \to \mathbb{R}^2 \). Se valgono le condizioni di
    Cauchy-Riemann, allora \(f : C \supseteq \Omega \to \mathbb{C} \) è
    \(\mathbb{C}-\)differenziabile in \(z = x_{0} + iy_{0}\) 
\end{proposition}
\begin{proof}
    Per ipotesi (con \(h = (h_{1}, h_{2})\))
    \[
        f(a + h) - f(a) = \frac{\partial f}{\partial x}(a) h_{1} +
        \frac{\partial f}{\partial y}(a) h_{2} + o(h)
    \]
    Poiché \(\frac{\partial f}{\partial y} = i \frac{\partial f}{\partial x}\)
    si ha 
    \begin{align*}
        f(a+h)  - f(a) &= \frac{\partial f}{\partial x}(a) h_{1} +
        i\frac{\partial f}{\partial x}(a) h_{2} + o(h) = \frac{\partial
    f}{\partial x}(a) (h_{1} + ih_{2}) + o(h) \\
        &= \frac{\partial f}{\partial x} (a) h + o(h) = \frac{\partial
        f}{\partial x} (a) (z-a) + o(z-a)
    \end{align*}
\end{proof}
\begin{theorem}[Looman-Menchoff]
    Sia \(f : \Omega \to \mathbb{C}\) continua e dotata di \(\frac{\partial
    f}{\partial x}\), \(\frac{\partial f}{\partial y}\) in \(z=a\). Se valgono
    le condizioni di Cauchy Riemann, allora \(f\) è
    \(\mathbb{C}-\)differenziabile in \(z=a\) 
\end{theorem}
Abbiamo già visto sui reali che analitica implica \(C^{\infty}\). Ora spiace lo
spoiler ma dimostreremo che \(\mathbb{C}\)-differenziabile implica analitica,
quindi \(\mathbb{C}-\)differenziabilità, \(C^{\infty}\), analitica saranno
nozioni equivalenti e gli assegneremo la dicitura di \textbf{olomorfe}.

\begin{definition}[Derivata complessa]
    \begin{align*}
        \frac{\partial f}{\partial z} &= \frac{1}{2} {\left( \frac{\partial
        f}{\partial x} - i \frac{\partial f}{\partial y} \right)} \\
                    \frac{\partial f}{\partial \overline{z}} &= \frac{1}{2}
                    {\left( \frac{\partial f}{\partial x} + i \frac{\partial
                    f}{\partial y} \right)}
    \end{align*}
\end{definition}
Ciò è motivato dal seguente passaggio formale: Sia \(z = x+iy\) con \(x,y \in
\mathbb{R}\), allora \(f(x,y) = f{\left( \frac{z+\overline{z}}{2},
\frac{z-\overline{z}}{2} \right)} \) e quindi si ottiene formalmente il
risultato come sopra definito.
\begin{remark}
    Le condizioni di Cauchy-Riemann diventano
    \( \displaystyle
        \frac{\partial f}{\partial \overline{z}} = 0
    \)
\end{remark}

\subsection{\(\mathbb{C}\)-differenziabilità delle funzioni analitiche}
\begin{theorem}
    Si consideri la serie 
    \[
        \sum_{n=0}^{\infty} c_{n} {\left( z-a \right)}^{n} \quad c_{n} \in \mathbb{C} 
    \]
    con \(R\) il raggio di convergenza.
    Allora la serie derivata 
    \[
        \sum_{n=1}^{\infty} nc_{n} {\left( z-a \right)}^{n-1}
    \] ha lo stesso raggio di convergenza \(R\). Inoltre se \(f(z)\) è la somma
    della serie data e \(g(z)\) la somma della serie derivata, allora avremo che
    \(f\) è \(\mathbb{C}\)-differenziabile e \(f'(z) = g(z)\) per ogni \(z \in
    D_R(a)\)
\end{theorem}
\begin{proof}
    Come nel caso reale,
    \[
        \limsup_{n \to \infty} \sqrt[n]{|nc_{n}|} = \limsup_{n\to \infty}{\sqrt[n]{|c_{n}|}} 
    \]
    quindi i due raggi di convergenza coincidono. Supponiamo \(a=0\). Fissiamo
    \(w \in D_R(0)\) e consideriamo
    \[
        \frac{f(w+h) - f(w)}{h}
    \] con \(h\) tale che \(w+h \in D_R(0)\). Scriviamo, per \(N \in
    \mathbb{N}\),
    \[
        f(z) = S_N(z) + R_N(z) \quad \text{ con } \quad S_N(z) = \sum_{n=0}^{N}
        c_{n} {\left( z - a \right)}^{n}
    \]
    e \(R_N(z)\) il resto della serie. Sappiamo che 
    \[
        \lim_{n \to 0} \frac{S_N(w + h) - S_N(w)}{h} = S'_N(w) \to g(z) \text{
        per \(N\to \infty\)  }
    \]
    Consideriamo il resto
    \[
        \frac{R_N(w +h) - R_N(w)}{h} = \frac{1}{h} \sum_{n=N=1}^{\infty} {\left(
        c_{n} {\left( {\left( w+h \right)}^{n} - w^{n} \right)} \right)}  
    \]
    essendo
    \[
        {\left( w + h \right)}^{n} - w^{n} = {\left( w + h - w \right)} {\left(
        {\left( w+h \right)}^{n-1} + {\left( w+h \right)}^{n-2} w + \dots +
w^{n-1}\right)} 
    \]
    ottengo
    \[
        \left| \frac{R_N(w+h)-R_N(w)}{h} \right| \le \sum_{n=N+1}^{\infty} |c_N| {\left( |w+h|^{n-1}
        +|w+h|^{n-2} |w| +~\dots + |w|^{n-1}\right)} 
    \]
    Ora, per \(h\) tale che \(w+h \in D_\rho(0)\), con \(|w| \le \rho < R\) si
    ha che \(|w+h|^{n-k} |w|^{k-1} \le \rho^{n-1}\) quindi
    \[
        \left| \frac{R_N(w+h) -R_N(w)}{h} \right| \le \sum_{n=N+1}^{\infty}
        |c_n| n \rho^{n-1} \to 0
    \]
    Poiché la serie \(\sum_{n=1}^{\infty} nc_{n} \zeta^{n-1} \) la serie
    derivata converge assolutamente in \(D_R(0)\) in particolare per \(\zeta =
    \rho\). 
    Concludiamo ora 
    \begin{align*}
        \limsup_{h\to 0} \left| \frac{f(w+h) - f(w)}{h} -g(w)\right| \le
        \limsup_{h\to 0} \left| \frac{S_N(w+h) - S_N(w)}{h} - S_N'(w) \right| +
        \\ +
        \limsup_{h \to 0} \left| S_N'(w) - g(w) \right| + \limsup_{h\to 0}
        \left| \frac{R_N(w+h) - R_N(w)}{h} \right| = |S'_N(w) - g(w)| +
        \varepsilon
    \end{align*}
    per \(N\) sufficientemente grande. Si conclude per l'abitrarietà di \(\varepsilon\) 
\end{proof}

\subsection{Integrazione su Curve}
\begin{definition}[Curva in \(\mathbb{C}\) ]
    Diremo \textbf{curva} in \(\mathbb{C}\) ogni funzione continua \(\gamma : [a,b] \to
    \mathbb{C}\). Si dice \textbf{chiusa} se \(\gamma(a) = \gamma(b)\). Il
    \textbf{sostegno} di \(\gamma\) è l'immagine di \(\gamma\), cioè
    \(\gamma([a,b])\). Inoltre \(\gamma\) si dice \(C^{1}\) a tratti se esistono
    \(a=t_{0}<t_{1}<\dots<t_{n}=b\) tali che 
    \[
        \gamma|_{[t_{k-1} , t_k]} \in C^{1}([t_{k-1}, t_k]) \quad \forall k = 1,
        \dots, n
    \]
    Diciamo \textbf{curva opposta} di \(\gamma\) la curva percorsa in
    ``senso opposto'' ossia:
    \[
        -\gamma : [a,b] \to \mathbb{C} \quad -\gamma(t) = \gamma(a+b-t)
    \]
    Chiamiamo \textbf{saldatura} di due curve \(\gamma_1 : [a_{1}, b_{1}] \to
    \mathbb{C}, \gamma_2 [a_{2}, b_{2}] \to \mathbb{C}\), con
    \(\gamma_{1}(b_{1}) = \gamma_{2}(a_{2}). \), la curva 
    \[
        (\gamma_{1} + \gamma_{2})(t) = \begin{cases}
            \gamma_{1}(t) & t \in [a_{1}, b_{1}] \\
            \gamma_{2}(a_{2} + t - b_{1}) & t \in [b_{1}, b_{1} + b_{2} - a_{2}]
        \end{cases}
        \quad \forall t \in [a_{1}, b_{1} + (b_{2}-a_{2})]
    \]
    (Notare che esiste anche la notazione moltiplicativa per saldatura e curva
    opposta).

Siano ora \(\gamma: [a,b] \to \mathbb{C}\) e \(\tilde{\gamma} : [\alpha, \beta]
\to \mathbb{C}\) due curve. Allora diciamo che le due curve sono
\textbf{equivalenti} se esiste \(\varphi : [\alpha, \beta] \to [a,b]\) \(C^{1}\)
a tratti, biettiva, con \(\varphi'>0\), tale che
\(
    \tilde\gamma = \gamma \circ \varphi 
\)
\end{definition}

Per convenzione, se non espressamente specificato diversamente considereremo
curve \(C^{1}\) a tratti.

\begin{definition}[Integrale su curva]
    Sia \(\gamma : [a,b] \to \mathbb{C}\) una curva \(C^{1}\) a tratti e sia
    \(f\) continua a valori in \(\mathbb{C}\) definita (almeno) sul sostegno di
    \(\gamma\). Allora si definisce
    \[
        \int_{\gamma} f(z) dz = \int_{a}^{b} f(\gamma(t)) \gamma'(t) dt
    \]
\end{definition}
Abbiamo le seguenti proprietà:
\begin{itemize}[label = --]
    \item (linearità) \(\displaystyle \int_{\gamma} \left( \lambda f + \mu g
        \right) dz = \lambda \int_{\gamma} f dz + \mu \int_{\gamma} g dz\)
    \item (additività) \(\displaystyle \int_{\gamma_1 + \gamma_2} f dz =
        \int_{\gamma_1} f dz + \int_{\gamma_2} f dz\)
    \item \(\displaystyle \left| \int_{\gamma} f dz \right| \le \text{ lungh
        }(\gamma) \cdot \max_{\text{spt } \gamma } |f|  \) 
    \item \(\displaystyle \int_{-\gamma} f(z) dz = - \int_{\gamma} f(z)dz\) 
\end{itemize}

Sia \(\mathcal{C}\) una curva in \(\mathbb{C}\) assegnata come ``oggetto
geometrico'': circonferenza, retangolo, segmento eccetera. Allora scriveremo
\(
    \int_{\mathcal{C}} f(z) dz
\)
purché il contesto chiarisca il tipo di parametrizzazione. Ad esempio
\(
    \int_{\partial D_R} 
\) o \(
    \int_{\partial R}
\) (rispettivamente integrale su circonferenza e su bordo di un rettangolo) si intenderà a meno di specificare in orientamento antiorario.

\begin{proposition}
    Siano \(\gamma : [a,b] \to \mathbb{C}\) e \(\tilde{\gamma} : [\alpha, \beta]
    \to \mathbb{C}\) due curve equivalenti. Allora
    \[
        \int_{\gamma} f(z) dz = \int_{\tilde{\gamma}} f(z) dz
    \]
\end{proposition}
\begin{proof}
    Sia \(\varphi : [\alpha, \beta] \to [a,b]\) la funzione di equivalenza. Allora
    \begin{align*}
        \int_{\tilde{\gamma}} f(z) dz &= \int_{\alpha}^{\beta} f(\tilde{\gamma}(t))
        \tilde{\gamma}'(t) dt = \int_{\alpha}^{\beta} f(\gamma(\varphi(t)))
        \gamma'(\varphi(t)) \varphi'(t) dt = \\&= \int_{a}^{b} f(\gamma(s)) \gamma'(s)
        ds = \int_{\gamma} f(z) dz
    \end{align*}
\end{proof}

\begin{example}
    Se consideriamo \(\int_{\partial D_R (a)} \) allora la parametrizzazione che
    prendiamo sarà \(\gamma(t) = a + Re^{it}\) con \(t \in [0, 2\pi]\). Quindi
    abbiamo \(\gamma'(t) = iRe^{it}\) e 
    \[
        \int_{\partial D_R(a)} f(z) dz = \int_{0}^{2\pi} f(a + Re^{it}) iRe^{it}
        dt   
    \]
    ad esempio se \(\displaystyle f(z) = \frac{1}{z-a}\) 
    \[
        \int_{\partial D_R(a)} \frac{1}{z-a} dz = \int_{0}^{2\pi} \frac{1}{Re^{it}}
        iRe^{it} dt = \int_{0}^{2\pi} i dt = 2\pi i
    \]
\end{example}
\begin{example}
    Se consideriamo \(R\) rettangolo, \(a \in R\sminus \partial R\). Calcoliamo
    quindi
    \[
        \int_{\partial R} \frac{1}{z-a} dz
    \]
    dove \(z = a + \rho(\theta) e^{i\theta} \) dove \(\theta \in [0, 2\pi]\) e
    \(\rho\) è \(C^{1}\) a tratti. Allora otteniamo che
    \[
        \int_{\partial R} \frac{1}{z-a} dz = \int_{0}^{2\pi} \frac{1}{\rho(\theta)
        e^{i\theta}} i \rho(\theta) e^{i\theta} d\theta = \int_{0}^{2\pi} i d\theta
        = 2\pi i
    \]
    % TODO: fare meglio
\end{example}
\begin{remark}
    Se ho \(F : \Omega \subseteq \mathbb{C} \to \mathbb{C} \), 
    \(\mathbb{C}-\)differenziabile e \(\gamma : [a,b] \to \Omega\) \(C^{1}\) a
    tratti, allora \(\displaystyle \frac{d}{dt}F{(\gamma{(t)})} =
    F'{(\gamma{(t)})}\gamma'{(t)}\). Infatti, fissato \(t_{0}\in [a,b]\)
    Consideriamo
    \[
        \frac{F{(\gamma{(t)})}-F{(\gamma{(t_{0})})}}{t-t_{0}}
    \]
    Ricordiamo che \(F(z) = F{(a)} + F'{(a)}{(z-a)} + {(\varepsilon
    {(z-a)})}{(z-a)}\) con \(\varepsilon {(w)}\) infinitesimo per \(w\to 0\) e
    \(\varepsilon {(0)} = 0\). Allora
    \[
        \frac{F{(\gamma{(t)})}- F{(\gamma{(t_{0})})}}{t-t_{0}} =
        F'{(\gamma{(t_{0})})}\frac{\gamma{(t)}-\gamma{(t_{0})}}{t-t_{0}} +
        \varepsilon{(\gamma{(t)}-\gamma{(t_{0})})}
        \frac{\gamma{(t)}-\gamma{(t_{0})}}{t-t_{0}}
    \]
    e passando al limite otteniamo la tesi.
\end{remark}
\begin{remark}
    \(\int_{\gamma} f(z) dz\) è l'integrale su un intervallo di una funzione
    vettoriale \(f{(\gamma{(t)})}\gamma'{(t)}\). Come tale possiamo applicare i
    risultati visti di passaggio al limite sotto il segno di integrale. Ad
    esempio supponiamo di avere \(\gamma : [a,b] \to \Omega\) e una successione
    e una funzione \(f_{n}, f : \text{spt}\gamma \to \mathbb{C}\) continua e
    \(f_{n} \to f\) uniformemente su \(\text{spt}\gamma\). Allora
    \[
        \int_{\gamma} f_{n}(z) dz \to \int_{\gamma} f(z) dz
    \]
    Infatti 
    per ipotesi sappiamo che
    \[
        \forall \varepsilon>0\,\, \exists n_\varepsilon : \forall n \ge
        n_\varepsilon \,\, \forall z \in \text{spt}\gamma \quad |f_{n}(z) -
        f(z)| < \varepsilon
    \]
    ma quindi anche \(\forall t \in [a,b]\) abbiamo che \(|f_{n}{(\gamma{(t)})}
    - f{(\gamma{(t)})} < \varepsilon\) e quindi
    \[
        \left| f_{n}{(g{(t)})}\gamma'{(t)} - f{(\gamma{(t)})}\gamma'{(t)}\right|
        \le |f_{n}{(\gamma{(t)})} - f{(\gamma{(t)})}| \max_{a\le s\le b}
        |\gamma'{(s)}| < M\varepsilon 
    \]
    cioè \(f_{n}{(\gamma{(\cdot )})}\gamma'{(\cdot )} \to f{(\gamma{(\cdot
    )})}\gamma'{(\cdot )}\) uniformemente.

    In particolare (come successione si consideri la successione delle somme
    parziali di una serie) si ha che se \(\sum_{n=0}^{\infty} f_{n}(z) \)
    converge uniformemente sul supporto di \(\gamma\) allora
    \[
        \int_{\gamma} \sum_{n=0}^{\infty} f_{n}(z) dz = \sum_{n=0}^{\infty}
        \int_{\gamma} f_{n}(z) dz
    \]
\end{remark}

\begin{definition}[Primitiva]
    Sia \(\Omega \subseteq \mathbb{C}\) aperto e \(f : \Omega \to \mathbb{C}\)
    continua. Una funzione \(F : \Omega \to \mathbb{C}\) si dice \textbf{primitiva} di
    \(f\) se \(F\) è \(\mathbb{C}-\)differenziabile e \(F'(z) = f(z)\) per ogni
    \(z \in \Omega\).
\end{definition}
\begin{proposition}
    Sia \(F\) primitiva di \(f\) e \(\gamma:[a,b] \to \Omega\) una curva
    \(C^{1}\) a tratti. Allora 
    \[
        \int_{\gamma} f(z) dz = F{(\gamma{(b)})} - F{(\gamma{(a)})}
    \]
\end{proposition}
\begin{proof}
    \[
        \int_{\gamma} f(z) dz = \int_{a}^{b} f{(\gamma{(t)})}\gamma'(t) dt =
        \int_{a}^{b} F'{(\gamma{(t)})}\gamma'(t) dt = F{(\gamma{(b)})} -
        F{(\gamma{(a)})}
    \]
\end{proof}
\begin{corollary}
    Se \(F\) ammette primitiva in \(\Omega\) allora \(\int_{\gamma} f(z) dz =
    0\) per ogni curva chiusa \(\gamma\) in \(\Omega\).
\end{corollary}
\begin{proof}
    ovvia
\end{proof}
\begin{corollary}
    Sia \(\Omega\) un aperto \textbf{connesso}, allora se \(f\) è
    \(\mathbb{C}-\)differenziabile e \(f'=0\) allora \(f\) è costante.
\end{corollary}
\begin{proof}
    Fissiamo \(z_{0}, z_{1} \in \Omega\), allora esiste (connessione per archi)
    una \(\gamma\) \(C^{1}\) a tratti (poligonale) con \(\gamma{(a)}=z_{0}\) e
    \(g{(b)}=z_{1}\) e allora poiché \(f\) è primitiva di \(f'\) abbiamo che 
    \[
        0 = \int_{\gamma} f'(z) dz = f{(\gamma{(b)})} - f{(\gamma{(a)})} =
        f(z_{1}) - f(z_{0})
    \]
\end{proof}

\end{document}

