\documentclass{article}
\usepackage{layout}
\usepackage[a4paper, total={5in,9in}]{geometry}
\usepackage[T1]{fontenc}
\usepackage[italian]{babel}
\usepackage{mathtools}
\usepackage{amsthm}
\usepackage[framemethod=TikZ]{mdframed}
\usepackage{amsmath}
\usepackage{amssymb}
\usepackage{cancel}
\usepackage{xcolor}
\usepackage{tikz}
\usepackage{tikz-cd}
\usepackage{pgfplots}
\pgfplotsset{compat=1.18}
\usepackage{tcolorbox}
\usepackage{import}
\usepackage{pdfpages}
\usepackage{transparent}
\usepackage{xcolor}
\usepackage{enumitem}


\newcommand*{\transp}[2][-3mu]{\ensuremath{\mskip1mu\prescript{\smash{\mathrm t\mkern#1}}{}{\mathstrut#2}}}%

\newcommand{\incfig}[2][1]{%
    \def\svgwidth{#1\columnwidth}
    \import{./figures/}{#2.pdf_tex}
}

\pdfsuppresswarningpagegroup=1

\newcounter{theo}[section]\setcounter{theo}{0}
\renewcommand{\thetheo}{\arabic{section}.\arabic{theo}}

\newcounter{excounter}[section]\setcounter{excounter}{0}
\renewcommand{\theexcounter}{\arabic{section}.\arabic{excounter}}

\newenvironment{theorem}[1][]{
    \refstepcounter{theo}
     \ifstrempty{#1}
    {\mdfsetup{
        frametitle={
            \tikz[baseline=(current bounding box.east),outer sep=0pt]
            \node[anchor=east,rectangle,fill=blue!20,rounded corners=5pt]
            {\strut Teorema~\thetheo};}
        }
    }{\mdfsetup{
        frametitle={
            \tikz[baseline=(current bounding box.east),outer sep=0pt]
            \node[anchor=east,rectangle,fill=blue!20,rounded corners=5pt]
            {\strut Teorema~\thetheo:~#1};}
        }
    }
    \mdfsetup{
        roundcorner=10pt,
        innertopmargin=10pt,linecolor=blue!20,
        linewidth=2pt,topline=true,
        frametitleaboveskip=\dimexpr-\ht\strutbox\relax%
    }
\begin{mdframed}[]\relax}{
\end{mdframed}}

\newenvironment{definition}[1][]{
    \refstepcounter{theo}
     \ifstrempty{#1}
    {\mdfsetup{
        frametitle={
            \tikz[baseline=(current bounding box.east),outer sep=0pt]
            \node[anchor=east,rectangle,fill=violet!20,rounded corners=5pt]
            {\strut Definizione~\thetheo};}
        }
    }{\mdfsetup{
        frametitle={
            \tikz[baseline=(current bounding box.east),outer sep=0pt]
            \node[anchor=east,rectangle,fill=violet!20,rounded corners=5pt]
            {\strut Definizione~\thetheo:~#1};}
        }
    }
    \mdfsetup{
        roundcorner=10pt,
        innertopmargin=10pt,linecolor=violet!20,
        linewidth=2pt,topline=true,
        frametitleaboveskip=\dimexpr-\ht\strutbox\relax%
    }
\begin{mdframed}[]\relax}{
\end{mdframed}}

\newenvironment{lemmao}[1][]{
    \refstepcounter{theo}
     \ifstrempty{#1}
    {\mdfsetup{
        frametitle={
            \tikz[baseline=(current bounding box.east),outer sep=0pt]
            \node[anchor=east,rectangle,fill=green!20,rounded corners=5pt]
            {\strut Lemma~\thetheo};}
        }
    }{\mdfsetup{
        frametitle={
            \tikz[baseline=(current bounding box.east),outer sep=0pt]
            \node[anchor=east,rectangle,fill=green!20,rounded corners=5pt]
            {\strut Lemma~\thetheo:~#1};}
        }
    }
    \mdfsetup{
        roundcorner=10pt,
        innertopmargin=10pt,linecolor=green!20,
        linewidth=2pt,topline=true,
        frametitleaboveskip=\dimexpr-\ht\strutbox\relax%
    }
\begin{mdframed}[]\relax}{
\end{mdframed}}

\theoremstyle{plain}
\newtheorem{lemma}[theo]{Lemma}
\newtheorem{corollary}{Corollario}[theo]
\newtheorem{proposition}[theo]{Proposizione}

\theoremstyle{definition}
\newtheorem{example}[excounter]{Esempio}

\theoremstyle{remark}
\newtheorem*{note}{Nota}
\newtheorem*{remark}{Osservazione}

\newtcolorbox{notebox}{
  colback=gray!10,
  colframe=black,
  arc=5pt,
  boxrule=1pt,
  left=15pt,
  right=15pt,
  top=15pt,
  bottom=15pt,
}

\title{Appunti di Geometria I}
\author{Osea}
\date{Secondo semestre 2023~--~2024}

\begin{document}
\maketitle

Libri: Sernesi 2 oppure (più avanzato) Manetti.

\tableofcontents

\section{Spazi Topologici}
\begin{definition}[Spazio topologico]
    Sia \(X\) un insieme non vuoto sul quale è
    presente una famiglia di sottoinsiemi \(\tau \subseteq P(X)\), i cui
    elementi sono chiamati \textbf{aperti}. La famiglia \(\tau\) deve avere
    queste proprietà:
\begin{enumerate}[label = \arabic*)]
    \item \(\varnothing, X \in \tau\) 
    \item \(\{A_{i}\}_{i \in J} A_{i} \in \tau \implies \bigcup_{i \in I} A_{i} \in \tau\)
    \item \(A_{1}, A_{2} \in  \tau \implies  A_{1} \cap A_{2} \in  \tau\) 
\end{enumerate}
    Allora \((X, \tau)\) è uno \textbf{spazio topologico}. Gli elementi di \(X\)
    si chiamano solitamente \textbf{punti}
\end{definition}
\begin{note}
    se 3') è \(A_{1}, \dots, A_{n} \in  \tau \implies  \bigcap_{i \in
    \{1,\dots,n\}} \in  \tau\) allora per il principio di induzione \(3) \iff 3')\) 
    per cui \textbf{l'intersezione finita di aperti è aperta}
\end{note}
\begin{definition}[insieme chiuso]
    Sia \((X, \tau)\) uno spazio topologico. Se \(A\) è un insieme aperto,
    allora \(X - A\) è \textbf{chiuso}
\end{definition}
\begin{note}
    Sia \(\mathcal{C}\) la famiglia degli insiemi chiusi, allora
\begin{enumerate}[label = \arabic*)]
    \item \(\varnothing, X \in \mathcal{C}\) 
    \item Intersezione arbitraria di chiusi è chiusa
    \item Unione finita di chiusi è chiusa
\end{enumerate}
Data una famiglia di chiusi si può definire la topologia dove gli aperti
sono complementari di chiusi.

Infatti se \(\varnothing, X\) sono chiusi allora sono anche aperti
(complementari), se \(A_{i}\) sono insiemi aperti allora \(\bigcup_{i \in  I} A_{i} = X - (\bigcap_{i \in  I} (X -
A_{i}))\) che è aperto, quindi l'unione arbitraria di aperti è aperta, se \(A\)
e \(B\) sono aperti allora \(A \cap  B = X - ((X - A) \cup (X - B))\) che è
aperto, quindi l'intersezione finita di aperti è aperta.
\end{note}

\begin{example}[Topologia Discreta]
    \(\tau = P(x)\) Tutti i sottoinsiemi di \(X\) sono aperti, si chiama
        \textbf{topologia discreta}.
\end{example}
\begin{example}[Topologia Indiscreta]
\(\tau = \{\varnothing, X\} \) più piccola
        topologia possibile.
\end{example}
\begin{example}[Alcune topologie su \(X\) di tre elementi]
\begin{enumerate}[label = \roman*-]
    \item Se \(X = \{1, 2, 3\} \) allora una topologia può essere \(\tau =
        \{\varnothing, \{1, 2, 3\}, \{1\}\} \) 
    \item Con lo stesso \(X\), \(\tau = \{\varnothing, \{1\} , \{2\} , \{1,2,3\}
        \}\) \textbf{non} è una topologia. Infatti \(\{1,2\} \not\in \tau\) 
\end{enumerate}
\end{example}
\begin{example}[Topologia Cofinita]
    \item Sia \(X\) un insieme, la topologia \textbf{cofinita} è quella tale che
        \(\mathcal{C} = \{\varnothing, X\} \cup \{S \in X : \# S < +\infty\}
        \). Verifico le proprietà:
\begin{enumerate}[label = \arabic*)]
    \item \(\varnothing, X\) sono chiusi
    \item L'intersezione arbitraria di chiusi ha cardinalità minore o eguale a
        ognuna delle cardinalità degli insiemi nell'intersezione, quindi in
        particolare è ancora finita, quindi è chiusa
    \item L'unione di due chiusi ha cardinalità al più eguale alla somma delle
        cardinalità dei due insiemi, per cui è finita, quindi chiusa.
\end{enumerate}

\end{example}
\begin{example}[Topologia Euclidea su \(\mathbb{R}\) ]
    Sia \(X = \mathbb{R}\). \(A \in  R\) aperto se è
        unione (anche infinita) di intervalli aperti \((a, b)\), quindi \(\tau =
        A \in  \mathbb{R} : A = \bigcup_{i \in  I} (a_{i}, b_{i})\) dove
        \((a_{i}, b_{i}) = \{x \in R : a_{i} < x < b_{i}\} \) 
\end{example}
\begin{proof}
\begin{enumerate}[label = \arabic*)]
    \item \(\varnothing, \mathbb{R} \in \tau\) 
    \item Siano \(A_{i}\) insiemi del tipo \(A_{i} = \bigcup_{j \in  I_{i}}
        (a_{ij}, b_{ij})\), con \(i \in J\). Allora \(\bigcup_{i \in  J}A_{i} =
        \bigcup_{i \in  J}(\bigcup_{j \in I_{i}} ) (a_{ij}, b_{ij})\) è ancora
        unione di intervalli aperti, quindi è in \(\tau\) 
    \item Siano \(A_{1}, A_{2} \in \tau\) allora \(A_{1} \cap A_{2}\) per la
        distributività di \(\cap \) su \(\cup \) è un'unione di intervalli
        aperti, quindi è aperto.
\end{enumerate}

\end{proof}

\begin{definition}
    Sia \(X\) un insieme non vuoto con una topologia \(\tau\) allora
    \(\mathcal{B} \subseteq 
    \tau\) si dice \textbf{base della topologia} se \(\forall A \in \tau, A =
    \bigcup_{i \in  I} B_{i}\), con \(B_{i} \in \mathcal{B}\,\,\forall i \in I\) 

\end{definition}
\begin{note}
\begin{enumerate}[label = \arabic*)]
    Quindi se \(\mathcal{B}\) è base di una topologia se e solo se
    \item \(\bigcup_{B \in \mathcal{B}} B = X\) 
    \item \(B_{1}, B_{2} \in \mathcal{B} \implies B_{1} \cap  B_{2} = \bigcup_{i \in I}
        B_{i}, \, B_{i} \in \mathcal{B}\) 
\end{enumerate}
\end{note}

\begin{example}
    Sia \(X = \mathbb{R}\), \(\mathcal{B} = \text{intervalli aperti}\) e
    consideriamo \(\mathcal{B}' \subseteq  \mathcal{B}\), con \(\mathcal{B}' =
    (a, b) \text{ con } a, b \in \mathbb{Q} \).
\begin{enumerate}[label = \arabic*)]
    \item \(\bigcup_{B \in  \mathcal{B}'} = \mathbb{R} \) 
    \item Intersezione di due intervalli razionali è ancora razionale, quindi
        vale anche la (2).
\end{enumerate}
    Quindi \(\mathcal{B}'\) è una base per una topologia. È la topologia
    euclidea? Sia \((a, b) \subseteq R\). Allora esistono due successioni \(a_n
    \to a\) monotona decrescente con \(a_n \in \mathbb{Q}\) e \(b_n \to  b\) monotona
    crescente con \(b_n \in \mathbb{Q}\). Di conseguenza \((a, b) = \bigcup_{i
    \in  \mathbb{N}} (a_i, b_{i})\) per cui è aperto nella topologia di
    \(\mathcal{B}'\). Di conseguenza la topologia euclidea si può riscrivere
    con le unioni di insiemi di \(\mathcal{B}'\). Quindi questo mostra che
    \textbf{la topologia euclidea ha una base numerabile}.
\end{example}
\begin{example}
    Anche la topologia \(\mathcal{B}''\) data dagli intervalli interi \((n, m)
    \text{ con } n, m \in \mathbb{Z}\) è una base, ma ovviamente non della
    topologia euclidea perché ad esempio \(\left(0, \frac{1}{2}\right)\) non è esprimibile
    come unione di intervalli interi.
\end{example}

\begin{definition}
    Sia \(S \subseteq  X\), allora \(\overline{S}\) è la \textbf{chiusura} di
    \(S\) se è il più piccolo chiuso che contiene \(S\) 
\end{definition}
\begin{proof}[Esistenza e Unicità]
    Sia \(\mathcal{S}\) la famiglia di chiusi che contengono \(S\). È non vuota,
    perché \(X \in \mathcal{S}\). Sia \(\overline{S} = \bigcap_{C \in
    \mathcal{S}} \) è chiuso perché è intersezione di chiusi. Ovviamente
    contiene \(S\) perché ogni elemento di \(\mathcal{S}\) contiene S. Inoltre
    ogni altro chiuso che contiene S contiene anche \(\overline{S}\) perché è
    stato usato nell'intersezione.
\end{proof}

\begin{proposition}
    \(K := \{x \in X : \forall A \in \tau\,,x \in A\,, A \cap S \neq \varnothing\}
    = \overline{S}\) 
\end{proposition}
\begin{proof}
    Sia \(x \in  K\), supponiamo per assurdo che \(x \not\in \overline{S} \iff x \in
X - \overline{S} \in \tau\) perché per definizione \(\overline{S}\) è chiuso.
Poiché \(S \subseteq \overline{S}\),  \(X - \overline{S} \cap S = \varnothing\) ma per
costruzione di \(K\),  \(X - \overline{S} \cap  S \neq \varnothing\), che è una
contraddizione.

    Per l'altra inclusione iniziamo mostrando che \(S \subseteq  K\), poi mostrando che
    \(K\) è chiuso concludiamo che \(\overline{S} \subseteq  K\). Sia \(x \in
    S\), allora per ogni aperto \(A\) contenente \(x\), \(x \in A \cap  S\),
    quindi necessariamente \(A \cap  S \neq \varnothing\), ossia \(x \in K\).
    Consideriamo ora l'insieme aperto \(X - \overline{K}\). Se \(K \neq
    \overline{K}\) allora esiste \(x \in \overline{K}, x \not\in K\).
\end{proof}
\section{Spazi Metrici}

\section{Funzioni Continue}
La definizione conosciuta è che \(f: (a, b) \subseteq \mathbb{R} \to  \mathbb{R}\) è continua se
\[\forall x \in (a, b), \forall \varepsilon \in  \mathbb{R}^{+},\,\, \exists \delta
    \in  \mathbb{R}^{+}: x' \in  (a, b), |x - x'| < \delta \implies  |f(x) - f(x')| <
\varepsilon\]

Per gli spazi metrici otteniamo:
Siano \((X, d)\) e \((Y, d')\) due spazi metrici. Diremo che \(f: X \to  Y\)
è continua se \[\forall x \in  X, \forall \varepsilon \in \mathbb{R}^{+}\,\,
\exists \delta \in \mathbb{R}^{+}: \forall x' \in
X, d(x, x') < \delta \implies  d'(f(x), f(x')) < \varepsilon\]

Per migliorare questa definizione notiamo che \(\{x' \in  X : d(x, x') <
\delta\} = D(x, \delta)\). Stessa cosa per il codominio. Otteniamo quindi che,
per ogni \(x\)  e per ogni \(\delta\) positivo \(f(D(x, \delta)) \subseteq
D(f(x), \varepsilon)\) 

\textbf{Claim}: quindi \(\forall x \in X, \forall \varepsilon \in \mathbb{R},
\varepsilon > 0, f^{-1}(D(f(x), \varepsilon))\) è un aperto di X

\textbf{Dimostrazione}. Sia \(f\) continua e sia \(x' \in f^{-1}(D(f(x),
\varepsilon))\), \(x' \in X\). Allora \(f(x') \in D(f(x), \varepsilon)\). Per la
definizione ``provvisoria'' di continuità esiste un \(\delta\) tale che \(D()\) 

\begin{proposition}[continuità in senso topologico]
    Siano \(X\) e \(Y\) spazi metrici.\ \(f: X \to  Y\) è continua in senso metrico se è solo se 
    \[
        \forall A \in \tau_Y, f^{-1}(A) \in \tau_X
    \]
    In altre parole \textbf{la controimmagine di aperti è aperta}
\end{proposition}
\begin{proof}
    Supponiamo che sia continua in senso topologico. Sia \(x \in X, \varepsilon
    \in \mathbb{R}^{+}\). Il disco \(D(f(x), \varepsilon)\) è aperto in \(Y\),
    quindi \(f^{-1}(D(f(x), \varepsilon))\) è aperto in \(X\). Chiaramente \(x
    \in f^{-1}(D(f(x), \varepsilon))\) ma quindi significa che esiste un disco
    centrato in \(x\) completamente contenuto in \(f^{-1}(d(f(x),
    \varepsilon))\), ossia \(\exists \delta \in \mathbb{R}^{+}:\,D(x, \delta)
    \subseteq f^{-1}(D(f(x), \varepsilon))\), quindi \(f(D(x, \delta)) \subseteq
    D(f(x), \varepsilon)\).

    Viceversa, supponendo \(f\) continua in senso metrico, 
\end{proof}

\begin{definition}
    Dati due spazi topologici \(X\) e \(Y\), rispettivamente con topologie
    \(\tau_X\) e \(\tau_Y\), una funzione \(F: X \to  Y\) si
    dice continua se \(\forall A \in \tau_Y,\, F^{-1}(A) \in \tau_X\) 
\end{definition}

\begin{remark}
    Se \(X, Y, Z\) sono spazi topologici, \(F: X \to  Y\) e \(G: Y \to  Z\) sono
    continue allora \(G \circ F\) è continua.
\end{remark}
\begin{proof}
    Sia \(A\) un aperto di \(Z\), allora \((G \circ F)^{-1}(A) =
    F^{-1}(G^{-1}(A))\). Dato che G è continua, \(G^{-1}(A)\) è aperto in \(Y\)
    e dato che \(F\) è continua la sua controimmagine tramite \(F\) è aperta in
    \(X\)
\end{proof}
\begin{remark}
    L'identità (da \(X\) a \(X\) con la stessa topologia) è una funzione continua, infatti la controimmagine di ogni aperto
    è se stesso, che è aperto.
\end{remark}
Le precedenti due osservazioni danno agli spazi topologici la struttura di
\textbf{categoria}, con le funzioni continue come morfismi.

\begin{example}
    L'identità da \(X\) a \(X'\), insiemi con gli stessi elementi ma diversa
    topologia, è continua se \(\tau_X' \subseteq \tau_X\) 
\end{example}
\begin{example}
    \(X\) con topologia discreta (ogni sottoinsieme di \(X\) è aperto).
    Allora qualsiasi funzione \(f: X \to Y\) è continua.
\end{example}
\begin{example}
    \(F: \mathbb{R}_e \to \mathbb{R}_d\) da $\mathbb{R}$ con topologia euclidea a
    \(\mathbb{R}\) con topologia discreta. Le funzioni non continue da
    \(\mathbb{R}_e\) a \(\mathbb{R}_e\) non sono continue neanche da
    \(\mathbb{R}_e\) a \(\mathbb{R}_d\) perché gli stessi insiemi aperti per cui
    la definizione di continuità non era rispettata sono aperti anche
    della topologia discreta. Una funzione continua per la topologia euclidea
    invece può essere non continua con la discreta. Ad esempio l'identità
    non è continua da \(\mathbb{R}_e\) a \(\mathbb{R}_d\) perché \([0, 1]\) è un aperto della discreta ma non
    dell'euclidea.

    Soluzione: solo le funzioni costanti sono continue, infatti se \(f\) è una
    funzione non costante, è possibile trovare un sottoinsieme \(A \subseteq f(X)\) in
    \(f(X)\), \(\varnothing \neq A \neq f(X)\). Dato che nella
    topologia discreta tutti gli insiemi sono aperti, \(A\) è aperto
    e chiuso. Se \(f\) fosse continua, la controimmagine di \(A\) sarebbe aperta
    e chiusa, ma in \(\mathbb{R}_e\) gli unici insiemi aperti e chiusi sono
    \(\varnothing \text{ e } \mathbb{R}_e\).
    
\end{example}

\begin{proposition}
    \(F: X \to Y\) continua se e solo se \(\forall C\) chiuso in \(Y\),
    \(F^{-1}(C)\) è chiuso in \(X\).
\end{proposition}
\begin{proof}
    Sia \(C\) chiuso (aperto) in \(Y\), allora \(Y - C\) è aperto (chiuso)
        in \(Y\), quindi \(F^{-1}(Y - C) = X - F^{-1}(C)\) è aperto (chiuso), quindi
        \(F^{-1}(C)\) è chiuso (aperto).
\end{proof}

\begin{definition}[Omeomorfismo]
    \(F: X^{\tau_X} \to Y^{\tau_Y}\) è un omeomorfismo se \(F\) è biettiva
    continua e \(F^{-1}\) è continua.
\end{definition}
\begin{remark}
    L'omeomorfismo è una relazione di equivalenza: infatti
\begin{itemize}[label = --]
    \item se esistono \(F: X \to Y\) e \(G: Y \to  Z\) biettive continue con
        inversa continua allora anche \(F \circ G\) è biettiva continua con
        inversa continua
    \item L'identità è un omeomorfismo, quindi ogni spazio topologico è isomorfo
        a se stesso
    \item Se esiste \(F: X \to  Y\) biettiva continua con inversa continua
        allora \(F^{-1}: Y \to  X\) è biettiva continua con inversa continua
\end{itemize}

\end{remark}

\begin{example}
    Le funzioni costanti sono sempre continue:

    Siano \(X\) e \(Y\) spazi
    topologici. Sia \(F : X \to  Y\) tale che \(F(x) = y\) per ogni \(x \in X\) e per
    un \(y \in Y\) fissato. Allora preso un aperto \(A\) di \(Y\) ci sono due
    possibilità: \(y \in A\) oppure \(y \not\in A\). Nel primo caso \(F^{-1}(A)
    = X\) e nel secondo \(F^{-1}(A) = \varnothing\). In entrambi i casi la
    controimmagine è un insieme aperto, quindi \(F\) è continua.
\end{example}

\begin{example}
    Tutti gli intervalli aperti di \(\mathbb{R}\) sono \emph{omeomorfi} tra di
    loro:
\begin{itemize}[label = --]
    \item se \(a, b \in \mathbb{R}, (a, b) \cong (0, 1)\), \(f(x) = \frac{x -
        a}{b - a}\) è continua e monotona crescente, quindi iniettiva. Essendo
        \(f((a, b)) = (0, 1)\) è suriettiva. L'inversa è ovviamente anche
        biettiva (da \(\mathbb{R}\) a \(\mathbb{R}\) tutte le funzioni biettive
        continue sono monotone e hanno inversa continua)
    \item \((1, +\infty) \cong (0, 1)\), \(f(x) = \frac{1}{x}\) 
    \item Se \(a \in \mathbb{R}, (a, +\infty) \cong  (1, +\infty)\), \(f(x) = x
        - a + 1\) 
    \item Se \(a \in \mathbb{R}, (-\infty, -a) \cong  (a, +\infty)\), \(f(x) =
        -x\) 
    \item \(\mathbb{R} \cong (-\frac{\pi}{2}, \frac{\pi}{2})\), \(f(x) =
        \arctan(x)\) 
\end{itemize}

\end{example}

Quindi anche se i diversi intervalli sono ``diversi'' dal punto di vista
metrico, sono topologicamente indistinguibili.
\begin{example}
    \(\mathbb{R}\) con la topologia dove i chiusi sono gli insiemi finiti, oltre
    a \(\mathbb{R}\) stesso (topologia cofinita). In tal caso dato che i chiusi
    sono unioni di singoletti, una funzione è continua se la controimmagine di
    ogni punto è finita (quindi chiusa) oppure tutto \(\mathbb{R}\) (quindi
    ancora chiusa). Ad esempio la funzione \(x \mapsto p(x)\) polinomio è
    continua per la topologia euclidea e la controimmagine di ogni punto è
    l'insieme delle radici del polinomio, ossia al più contiene un numero di
    elementi corrispondenti al grado del polinomio. I polinomi sono quindi
    continui anche secondo la topologia cofinita.

    Tuttavia ad esempio la funzione definita come \(0\) per \(x < 0\), \(1\) per
    \(x > 1\) e \(x\) per \(x \in [0, 1]\) è continua per l'euclidea ma non per
    la cofinita, dato che la controimmagine di \(1\), ad esempio, ha cardinalità
    non numerabile.
\end{example}

\begin{example}
    \(\mathbb{R}\) con la topologia dove gli aperti sono le semirette \((a,
    +\infty)\) con \(a \in \mathbb{R} \cup \{-\infty\} \). Per esercizio
    dimostrare che è una topologia e dare esempi di funzioni continue.
\end{example}

\section{Successioni}
Se abbiamo un'applicazione \(f : \text{aperto }A \subseteq  \mathbb{R} \to  \mathbb{R}\) è
continua se e solo se ogni successione \(x_{n} \in A\), \(x_{n} \to  x\) allora
\(f(x_{n}) \to  f(x)\). Questo vuol dire che \(\lim_{n \to \infty}(x_{n} - x) =
0\) ossia 
\[
    \forall \varepsilon > 0, \varepsilon \in \mathbb{R}, \, \exists \overline{n}
    : \forall n \ge \overline{n}, |x_{n} - x| < \varepsilon
\]
In uno \textbf{spazio metrico}, definiamo una successione convergente 
\(x_{n} \to x \in X\) se \(\lim_{n \to  \infty} {d (x_{n}, x)} = 0\) 
Quindi se una successione \(x_{n}\) converge a \(x \in  X\) il limite precedente diventa
\[
    \forall  \varepsilon > 0 \exists  \overline{n} : \forall n \ge
    \overline{n}, d(x_{n}, x) < \varepsilon \implies x_{n} \in D(x, \varepsilon)
\]
Possiamo limitare gli \(\varepsilon\) solo a un insieme numerabile (ad esempio
\(\frac{1}{n} \forall  n \in \mathbb{N}\) )


Riscrivendo la notazione di continuità precedente in forma topologica:
\[
    \forall A \text{ aperto } x \in  A \exists \overline{n} : \forall n \ge
    \overline{n} f(x_{n}) \in A
\]
Infine possiamo ancora allargare la richiesta a un qualsiasi intorno di \(x\),
ossia un insieme contenente un aperto contenente \(x\) 

\begin{definition}[Convergenza di successione]
    Una successione \(x_{n} \in X\) converge a \(x \in  X\)  se
    \[
        \forall  A \text{ aperto }, x \in A\, \exists \overline{n} : \forall n >
        \overline{n} x_{n} \in A
    \]
\end{definition}
Notare che se non siamo in uno spazio metrico non è così difficile né utile la
convergenza: ad esempio nella concreta qualsiasi successione converge a
qualsiasi elemento.

Caratterizziamo prima di procedere la chiusura di \(S\) 
\begin{proposition}
\[\overline{S} = \{s \in  X : \exists \{x_{n}\}_{n \in  N}, x_{n} \in S, x_{n}
\to  s \} =: K\]
Ossia \(\overline{S}\) è l'insieme dei punti di accumulazione di \(S\) 
\end{proposition}

\begin{proof}\( \) 
\begin{itemize}[label = --]
    \item[\(\supseteq\) ] Prendiamo un punto di accumulazione \(x \in K\), allora
        esiste una successione \(x_{n} \in S\) tale che \(x_{n} \to  x\). Dobbiamo
        mostrare che \(x \in \overline{S}\). Prendiamo un aperto \(A\) tale che
        \(x \in A\), allora esiste \(\overline{n}\) tale che \(\forall n \ge
        \overline{n}, x_{n} \in A\), quindi \(A \cap  S \neq \varnothing\), quindi
        \(x \in \overline{S}\)
    \item[\(\subseteq \) ] Prendiamo un punto della chiusura, \(x \in  \overline{S}\). Vogliamo
        far vedere che sta in \(K\). Prendiamo \(D(x, \frac{1}{n})\) è aperto.
        Il criterio della chiusura dice quindi che \(D(x, \frac{1}{n}) \cap  S =
        \varnothing\). Preso \(x_{n} \in X \) che appartiene ad entrambi, quindi
        costruendo così una successione otteniamo che \(\lim d(x_{n}, x) = 0\)
        quindi \(x_{n} \to x\) quindi \(x\) è punto di accumulazione
\end{itemize}
\end{proof}
Da questo quindi troviamo che \(C \subseteq  (X, d)\) è chiuso se e solo se
\(c_n \in C\), \(c_n \to y\) allora \(y \in C\) 
% \begin{definition}[Primo assioma di numerabilità]
%     \(X\) soddisfa il primo assioma di numerabilità se \(\forall x \in X\)
%     esiste una famiglia numerabile \(U_n\) di intorni di \(x\) tale che
%     \(\forall A\) aperto \(\exists n : U_n \subseteq  A\) 
%     \textbf{Sistema di intorni}
% \end{definition}
\begin{theorem}
    Tra due \textbf{spazi metrici} \(X, d\) e \(Y, d'\), \(F : X \to  Y\) è
    continua se e solo se per ogni successione convergente \(x_{n} \to x \in X
    \implies f(x_{n}) \to f(x) \in Y\) 
\end{theorem}

\begin{proof}
    Supponiamo \(f : X \to  Y\) continua in senso topologico, allora \(\forall
    A\) aperto in \(Y, f^{-1}(A)\) è aperto in \(X\). Sia \(x_{n} \to x \in X\)
    una successione convergente in \(X\). 
    Sia \(A\) tale che \(f(x) = y \in  A\). \(f^{-1}(A) = B\) aperto e contiene
    \(x\). \(\exists  \overline{n} : \forall n \ge \overline{n}, x_{n} \in B\)
    ma quindi \(f(x_{n}) = y_{n} \in A\) per cui \(f(x_{n}) \to f(x)\).

    Viceversa, supponiamo di avere uno spazio metrico \(X, d\) e uno spazio \(Y,
    d'\), supponiamo che \(\forall x_{n} \to  x \implies f(x_{n}) \to  f(x)\) e
    cerchiamo di dimostrare che allora è continua in senso topologico. A tale
    scopo dimostriamo che la controimmagine di un chiuso di \(Y\) è sempre un
    chiuso di \(X\). Sia quindi \(C \subseteq Y\) chiuso. Prendiamo una
    successione di elementi \(x_{n} \in  f^{-1}(C)\) che convergono \(x_{n} \to
    x\), dobbiamo dimostrare che \(x \in  f^{-1}(C)\), e questo significherebbe
    che \(f^{-1}(C)\) contiene i suoi punti di accumulazione ed è quindi chiuso.
    Sappiamo che \(f(x_{n}) \to f(x)\). Quindi, dato che \(C\) è chiuso \(f(x)
    \in C \implies x \in f^{-1}(C)\) quindi \(f\) è continua. 
\end{proof}

\begin{example}
    Nella \textbf{topologia discreta} sono convergenti solo le successioni
    definitivamente costanti, infatti se prendiamo come aperto il punto \(x\),
    una successione convergente deve definitivamente avere punti in \(\{x\} \),
    quindi è definitivamente costante.
\end{example}
\begin{example}
    Sia \(\mathbb{R}\) con la topologia dove una base di aperti sono gli
    intervalli del tipo \([a , b)\). 
    Gli aperti sono quindi del tipo \(A = \bigcup_{i \in  I} [a_{i}, b_{i})\). È
    una topologia più fine di quella euclidea perché ogni intervallo \((a,
    b)\) è ottenibile come unione
    \[
        \bigcup_{n \in  \mathbb{N}} \big[a + \frac{1}{n}, b\big) = (a, b)
    \]
\end{example}

\section{Funzione Lipschitziana}
\begin{definition}
    Dati due spazi metrici \((X, d)\) e \((Y, d')\), una funzione \(F: X \to Y\)
    Si dice \textbf{Lipschitziana} se 
    \[
        \forall x, x' \in X \,\, \exists 0 < K \in \mathbb{R} : d'(f(x), f(x'))
        \le K d(x, x')
    \]
\end{definition}

\begin{proposition}
    Sia \(F : X \to Y\) una funzione Lipschitziana, allora \(F\) è continua
\end{proposition}
\begin{proof}
    Sia \(x_{n} \to x\) una successione in \(X\) convergente a \(x \in X\).
    Allora \(d'(f(x), f(x_{n})) \le Kd(x, x_{n}) \to 0\) quindi \(f(x_{n}) \to
    f(x)\).
\end{proof}

\begin{example}[Distanza da un insieme]
    Sia \(\varnothing \neq C \subseteq  X\), e sia \(f_C : X \to \mathbb{R}\)
    definita da \[f_C(x) = \inf_{c \in C} d(x, c) =: d(x, C)\] 
    Ossia \(f_C\) definisce la distanza dall'insieme \(C\).

    Ora notiamo che, per la disuguaglianza triangolare,
    \[
        \forall c \in  C, \forall x, y \in X, f_C(x) \le d(x, c) \le d(x, y) +
        d(y, c)
    \]
    Sostituendo a \(d(y, c)\) il suo estremo inferiore otteniamo che
    \[
        f_C(x) \le d(x, y) + f_C(y)
    \]
    Da cui
    \[
        |f_C(y) - f_C(x)| = \le d(x, y), \forall x, y \in X
    \]
    Quindi \(f_C\) è una funzione Lipschitziana con \(K = 1\), quindi è una
    funzione continua
\end{example}

\begin{proposition}
    \[
        \overline{C} = \{x \in X : f_C(x) = 0\} 
    \]
\end{proposition}
\begin{proof}
    Chiamando l'insieme alla destra \(K\):
\begin{itemize}[label = --]
    \item[\(\subseteq  \) ] \(K\) è chiuso perché è la controimmagine di \(\{0\} \subseteq
        \mathbb{R}\), che è chiuso. Inoltre \(C \subseteq  K\) perché se \(c \in
        C\) allora \(d(c, c) = 0\). Quindi \(\overline{C} \subseteq  K\) 
    \item[\(\supseteq  \) ] Prendiamo un elemento \(s \in \overline{C}\), quindi esiste una
        successione convergente \(c_n \to  s\) ossia \(d(c_n, s) \to s\) quindi
        l'estremo inferiore della distanza è \(0\), ossia \(\overline{C}
        \subseteq  K\) 
\end{itemize}
\end{proof}

\begin{proposition}
    In uno spazio metrico ogni punto è chiuso, infatti la controimmagine di
    \(0\) della funzione distanza \(d(x, y)\) con \(x\) fissato è \(\{x\}\) che
    è quindi chiuso.

    Dato che ad esempio in \(\mathbb{R}\) con la topologia dove gli aperti sono del
    tipo \((-\infty, a)\) i punti non sono chiusi, gli spazi topologici sono di
    più degli spazi metrici.
\end{proposition}
\begin{example}
    Sia \(\mathbb{R}\) con la topologia \textbf{cofinita}. Dimostrare che
    \textbf{non è metrizzabile} (anche se i punti sono chiusi).

    Hint: notare che non è \(T_{2}\) 
\end{example}

\begin{definition}
    Un'applicazione \(f: X \to Y\), con \(X, Y\) spazi topologici si dice
    \textbf{aperta} se
    \[
        \forall A \subseteq  X \text{aperto}, f(A) \text{ è aperto }
    \]
    e similmente si dice \textbf{chiusa} se
    \[
        \forall A \subseteq  X \text{chiuso}, f(A) \text{ è chiuso }
    \]
\end{definition}
\begin{proposition}
    Ogni omeomorfismo è aperto e chiuso
\end{proposition}
\begin{proof}
    Sia \(f: X \to Y\) omeomorfismo da \(X\) a \(Y\). Sia \(g := f^{-1}\)
    Preso \(A\) aperto di \(X\), \(g(f(A)) = A\) e quindi \(g^{-1}(A) = f(A)\)
    Ossia la controimmagine di \(A\) della funzione inversa di \(f\) è
    l'immagine attraverso \(f\) di \(A\). Ma dato che \(g\) è continua, \(f(A)\)
    è aperto. Per la stessa ragione la funzione è anche chiusa.
\end{proof}
Anzi
\begin{proposition}
    \(f: X \to Y\) continua e biettiva allora è un omeomorfismo se e solo se è
    aperta (oppure chiusa).
\end{proposition}
\begin{proof}
    Simile a prima
\end{proof}

\section{Costruzioni}
\subsection{Sottospazi topologici}
\begin{definition}[Topologia Indotta]
Sia \(X\) uno spazio topologico, e \(S \subseteq  X\). Per indurre una topologia
su \(S\) dichiariamo aperti di \(S\) gli insiemi \(A \cap  S\), con A aperto di
\(X\). Tale topologia viene chiamata \textbf{topologia indotta}
\end{definition}
\begin{proof}[Buona definizione] Controlliamo le proprietà della topologia:
\begin{enumerate}[label = \arabic*)]
    \item \(\varnothing \cap S = \varnothing \in \tau_S\), \(X \cap  S = S \in
        \tau_S\) 
    \item \(\bigcup_{i \in I} (A_{i} \cap S) = \left(\bigcup_{i \in  I}
            A_{i}\right) \cap S\) 
    \item \((A_{1}\cap S) \cap (A_{2} \cap S) = (A_{1} \cap A_{2}) \cap S\) 
\end{enumerate}
Quindi le proprietà della topologia sono verificate
\end{proof}

\begin{proposition}
    I chiusi di \(S\) sono le intersezioni di chiusi di \(X\) con \(S\) 
\end{proposition}
\begin{proof}
    \(S - (S \cap A) = S \cap  (X - A)\) 
    Con \(A\) aperto di \(X\) 
\end{proof}
\begin{example}
    Sia \(S_r^{n}\) la sfera \(n\)-dimensionale  di raggio \(r\) 
    \[
        S_r^{n} = \{\mathbf{x} \in \mathbb{R}^{n+1} : \|\mathbf{x}\| = r\}   
    \]
    Ha una struttura di spazio topologico.
\end{example}

Ora dopo aver definito la topologia indotta su \(S \subseteq X\), consideriamo
la funzione \(i_S : S \to  X\) l'inclusione. L'inclusione è continua. Infatti
preso \(A \subseteq  X\) aperto, \(i_S^{-1}(A) = S \cap  A\) che è un aperto
di \(S\), per cui \(i_S\) è continua.

In particolare \(\tau_S\) è la topologia \textbf{meno fine} su \(S\) che rende
\(i_S\) continua

\begin{proposition}[Proprietà Universale]
    Siano \(X\) e \(Y\) spazi topologici, si consideri una funzione \(F : X \to
    S\), con \(S \subseteq Y \) un sottospazio e sia \(i_S : S \to Y\) l'inclusione.

    Allora \(F\) è continua se e solo se \(i_S \circ F : X \to  Y\) è continua
\end{proposition}
\begin{tikzcd}
    X \arrow[r, "i_S \circ F"]  \arrow[rd, "F"] & Y \\
& S \arrow[u, "i_S", hook]
\end{tikzcd}

\begin{proof}\( \) 
\begin{itemize}[label = --]
    \item[\(\implies \) ] Se \(F\) è continua allora \(i_S \circ
        F \) è continua, perché \(i_S\) è continua
    \item[\(\impliedby  \) ] Se \(i_S \circ F\) è continua, sia \(A \cap S\) un aperto di
        \(S\), con \(A\) aperto di \(Y\), allora \(F^{-1}(A \cap S) = \{x \in X
        : F(x) \in A \cap S\} = \{x \in X : F(x) \in A\} = F^{-1}(A) = {(i_S
    \circ F)}^{-1}(A)\) che è aperto perché controimmagine di un aperto
        tramite funzione continua
\end{itemize}
\end{proof}
\begin{corollary}
    Sia \(F : X \to Y\) e \(S \subseteq X \), \(T \subseteq Y \) sottospazi tali
    che \(F(S) \subseteq T \). Allora \(F \text{ è continua } \implies  F|_S \text{ è
    continua}\).
\end{corollary}
\begin{tikzcd}
X \arrow[r, "F"]  & Y \\
S \arrow[r, "F\vert_S"'] \arrow[u, "i_S", hook]       &  T \arrow[u, "i_T", hook]
\end{tikzcd}
\begin{proof}
    Sia \(F\) continua, allora \(i_T \circ F|_S = F \circ i_S\) è continua
    perché composizione di funzioni continue. Ma allora per la proprietà
    universale anche \(F|_S\) è continua
\end{proof}

\begin{proposition}
    Sia \(S \subseteq Y \subseteq X  \). Da X indurre una topologia su \(Y\) e
    su \(S\). È indifferente prendere la topologia indotta su \(Y\) e
    indurla su \(S\).
\end{proposition}
\begin{proof}
    \(\tau_S = \{U = A \cap  S, \text{ a aperto di \(X\)  } \} \) ma quindi ogni
    aperto di \(S\) è della forma \(U = A \cap  S = (A \cap  Y) \cap  S\), dove
    \(A \cap  Y\) è un aperto di \(Y\). Viceversa se \(W\) è indotto da \(Y\)
    allora \(W = B \cap S\) con \(B\) aperto di \(Y\), quindi \(W = (A \cap Y)
    \cap S\) con \(A\) aperto di \(X\) ma \(S \subseteq Y \) quindi \(W = A \cap
    S\) 
\end{proof}

\begin{remark}
    Sia \(\varnothing \neq A \subseteq X \) un aperto. La topologia indotta è
    \(\tau_A = \{A \cap  A', \text{ \(A'\) aperto di \(X\)  }\} \) ma quindi
    ogni aperto è intersezione di due aperti di \(X\), che è quindi ancora
    aperto di \(X\), quindi \textbf{tutti gli aperti di \(A\) sono gli aperti
        \(A'\) di \(X\) tali che \(A' \subseteq A  \) }. La stessa osservazione
        vale anche per i chiusi.
\end{remark}

\begin{proposition}
    Supponiamo di avere \(S \subseteq Y \subseteq X  \). Siano
    \(\overline{S}^{X}\) e \(\overline{S}^{Y}\) rispettivamente i più piccoli
    chiusi di \(x\) e di \(Y\) contenenti \(S\).
    \textbf{Allora} \(\overline{\overline{S}^{Y}}^{X} = \overline{S}^{X}\) 
\end{proposition}
\begin{proof}
    Una inclusione è ovvia: \(\overline{S}^{X} \subseteq
    \overline{\overline{S}^{Y}}^{X} \) perché \(S \subseteq \overline{S}^{Y} \).

    Viceversa sia \(x \in \overline{\overline{S}^{Y}}^{X}\). Allora \(x \in
    \overline{S}^{X} \iff A \cap S \neq \varnothing\) per ogni \(A \ni x\) aperto di
    \(X\). Notare però che \(A \cap \overline{S}^{Y} \neq \varnothing\) Sia
    quindi \(y \in  A \cap  \overline{S}^{Y}\) e in particolare \(y \in Y\)
    quindi \(y \in A \cap Y = B\) che è un aperto di \(Y\) ma quindi \(B \cap  S
    \neq \varnothing\) e allora \(A \cap S \neq \varnothing\).
\end{proof}

\begin{proposition}
    Sia \(\varnothing \neq S \subseteq X \), con \(X\) spazio topologico. Allora
    Sia \(D \subseteq Y \subseteq X  \) con \(D\) denso in \(Y\), \(Y\) denso in
    \(X\) allora \(D\) è denso in \(X\). Un insieme \(A \subseteq X \) si dice
    denso se \(\overline{A}^{X} = X\).
\end{proposition}
\begin{proof}
    Per ipotesi \(\overline{D}^{Y} = Y\) e \(\overline{Y}^{X} = X\) ma allora
    \(\overline{D}^{X} = \overline{\overline{D}^{Y}}^{X} = \overline{Y}^{X} =
    X\) quindi \(D\) è denso in \(X\) 
\end{proof}


\begin{example}
    Sia \(S^{n}(r)\) la sfera n-dimensionale di raggio \(r\).

    \(\implies  S^{n}(r)\) omeomorfa a \(S^{n}(r'), \forall r, r' > 0\) 
\end{example}
\begin{proof}
    Dimostriamo che \(S^{n}(r) \equiv S^{n}(1), \forall r > 0\). Sia \(F_\rho :
    \mathbb{R}^{n+1} \to  \mathbb{R}^{n+1}\) tale che \(F(x) = \rho x\), con
    \(\rho > 0\). Notiamo che \(d(F_\rho (x), F_\rho(y)) = d(\rho x, \rho y) =
    \| \rho x, \rho y\| = \rho d(x, y)\) quindi è (lineare) Lipschitziana e continua.
    Inoltre \(F_\rho ^{-1} = F_{\rho ^{-1}}\). A questo punto per restrizione
    possiamo dire che anche \(F_r|_{S(1)} : S(1) \to S(r)\) è continua, e dato
    che anche la sua inversa è continua allora è un omeomorfismo.
\end{proof}

\subsection{Topologia Prodotto}
Provando a definire una topologia sull'insieme \(X \times  Y\) iniziamo
ipotizzando come aperti gli insiemi del tipo \(A \times  B\), con \(A\) aperto
in \(X\) e \(B \) aperto in \(Y\).
\begin{enumerate}[label = \arabic*)]
    \item \(\varnothing \times  \varnothing = \varnothing\) aperto
    \item \(X \times Y\) aperto
    \item \((A_{1} \times B_{1}) \cap (A_{2} \times B_{2}) = (A_{1} \cap A_{2})
        \times (B_{1} \cap B_{2})\) è aperto
\end{enumerate}
Tuttavia l'unione di due insiemi definiti aperti in questo modo non è
necessariamente un prodotto di aperti. Dobbiamo quindi dire che gli insiemi \(A
\times B\) sono una \textbf{base} della topologia.

\begin{definition}[Topologia Prodotto]
    Dati due spazi topologici \(X^{\tau_X}\) e \(Y^{\tau_Y}\). Diamo
    sull'insieme \(X \times Y\) una struttura di spazio topologico dichiarandone
    gli aperti tutti gli insiemi del tipo
    \[
        A = \bigcup_{i \in  I} A_{i} \times  B_{i}\,\,,\,A_{i} \text{ aperto di
        \(X\) }, B_{i} \text{ aperto di \(Y\)  }
    \]
    Ossia i prodotti di aperti sono una \textbf{base della topologia}.
\end{definition}

\begin{example}
    In \(\mathbb{R}^2\) abbiamo adesso due modi di definire una topologia: con
    la topologia prodotto (\(\mathbb{R}^2 = \mathbb{R} \times  \mathbb{R}\)) e
    anche la topologia indotta dalla metrica euclidea. Sono la stessa topologia?

    Sì, perché usando la metrica del massimo, che è equivalente alla metrica
    euclidea, dà origine ad una base di aperti costituita da quadrati. Essendo
    che la topologia prodotto contiene tutti i rettangoli, \textbf{la topologia
    prodotto è più fine della topologia euclidea}. Viceversa prendiamo che base
    della topologia prodotto i prodotti di rettangoli prodotti di intervalli.
    Ogni rettangolo si può semplicemente scrivere come unione di quadrati, ossia
    elementi della topologia indotta dalla metrica del massimo. Quindi anche
    \textbf{la topologia euclidea è più fine della topologia prodotto}.
\end{example}

\begin{theorem}
    Siano \(\pi_X\) e \(\pi_Y\) le proiezioni \(\pi_X : X \times Y \to X, (x, y)
    \mapsto x\) e \(\pi_Y : X \times Y \to Y, (x, y) \mapsto y\). Allora
    \(\pi_X\) e \(\pi_Y\) sono continue e la topologia \(\tau_{X \times Y}\) è
    la topologia meno fine che rende continue le due proiezioni.
\end{theorem}
\begin{proof}
    Sia \(A\) aperto di \(X\), \(\pi_X^{-1}(A) = A \times Y\) che è aperto in
    \(X \times Y\) perché elemento della base.
    Similmente per \(\pi_Y\).

    Se \(\pi_X\) e \(\pi_Y\) sono continue allora \(A \times Y\) e \(X \times
    B\) con \(A\) aperto di \(X\)  e \(B\) aperto di \(Y\) devono essere aperti,
    ma quindi anche \((A\times Y) \cap (X \times B) = A \times B\) deve essere
    aperto.
\end{proof}

\begin{proposition}[Proprietà Universale]
    Sia \(G : Z \to (X \times Y)\) allora \(G\) è continua se e solo se \(G_{1}
    = \pi_X \circ G\) e \(G_{2} = \pi_Y \circ G\) sono continue.
\end{proposition}

\begin{tikzcd}
    Z \arrow[r, "F"] & X \times Y
\end{tikzcd}

\begin{proof}
    Una implicazione è ovvia perché se \(G\) è continua allora lo sono anche
    \(G_{1}\) e \(G_{2}\) perché composizione di funzioni continue.

    Viceversa siano \(G_{1}\) e \(G_{2}\) continue e siano \(A\) aperto di \(X\)
    e \(B\) aperto di \(Y\). Allora \(sia z \in G^{-1}(A \times B)\) quindi
    \((G_{1}(z), G_{2}(z)) \in A \times B\) ma quindi \(z \in  G_{1}^{-1}(A)
    \cap G_{2}^{-1}(B)\) che è intersezione di aperti quindi è aperta.
    A questo punto la controimmagine di un generico aperto \(U\) di \(X \times
    Y\) è \(G^{-1}(U) = \bigcup_{i \in I} (A_{i}\times B_{i}) = \bigcup_{i \in
    I} G^{-1}(A_{i} \times B_{i})\) che è unione di aperti quindi aperta.   
\end{proof}

\begin{remark}
    Se \(\mathcal{B}_X\) è una base della topologia di \(X\) e \(\mathcal{B}_Y\)
    è una base della topologia di \(Y\), allora anche solo \(\{U_{i} \times
    K_{i}\}, U_{i} \in  \mathcal{B}_X, K_{i} \in  \mathcal{B}_Y\) è una base
    della topologia prodotto su \(X \times  Y\) 
\end{remark}

Ora fissiamo un \(\overline{y} \in Y\), denotando l'insieme \(\{(x, y) \in X
\times Y : y = \overline{y}\} \) con \(X \times \{\overline{y}\}  \subseteq X
\times Y \). Similmente denotiamo anche \(\overline{x} \times Y \). Notare che
se \(\overline{y}\) è chiuso allora \(X \times \overline{y} =
\pi_Y^{-1}(\overline{y})\) quindi è chiuso.
\begin{proposition}
    Consideriamo la funzione \(\rho : X \to X \times \overline{y}; x \mapsto (x,
    \overline{y})\). È un omeomorfismo.
\end{proposition}
\begin{proof}
    Usando la proprietà universale, \(\pi_X \circ \rho : X \to X\) è l'identità,
    quindi è continua; \(\pi_Y \circ \rho : X \to Y\) è costante (sempre
    \(\overline{y}\)) perciò è continua. Ne concludiamo che \(\rho\) è continua.
    Ha come inversa la proiezione quindi ha inversa continua.
\end{proof}

\begin{proposition}
    Le funzioni
\begin{itemize}[label = --]
    \item \(s: \mathbb{R}\times \mathbb{R} \to \mathbb{R}; (x, y) \mapsto x+y\)
    \item \(m: \mathbb{R}\times \mathbb{R} \to \mathbb{R}; (x,y) \mapsto xy\) 
    \item \(\rho: \mathbb{R}^{+} \to \mathbb{R}^{+}; x \mapsto \frac{1}{x}\) 
\end{itemize}
sono continue
\end{proposition}

\begin{proposition}
    Siano \(f, g: X \to \mathbb{R}\) continue, allora \(f+g\) e \(fg\) sono
    continue. Se \(h: X \to \mathbb{R}^{+}\) è continua, allora \(\frac{1}{h}\)
    è continua
\end{proposition}

\begin{proof}
    Consideriamo la funzione continua per proprietà universale \(F: X \to
    \mathbb{R}^2; x \mapsto (f(x),
    g(x))\) e la componiamo con \(s\) e \(m\). Similmente componiamo \(h\) con
    \(\rho\) per ottenere l'ultimo risultato.
\end{proof}

\begin{proposition}
    Supponiamo di avere uno spazio metrico \(X\) con distanza \(d: X\times X \to
    \mathbb{R}\). Vediamo se la distanza è una funzione continua.
\end{proposition}
\begin{proof}
    Consideriamo le controimmagini degli intervalli \(-\infty, a\) e \(a,
    \infty\). Si chiama \textbf{sottobase} perché l'intersezione finita dà una
    base.
\begin{itemize}[label = --]
    \item[\(d^{-1}(-\infty, a)\)] Un elemento nella controimmagine è una coppia
        \((x, y) \in X\times X\)  tale che \(d(x,y) = d_{0} < d\). Vogliamo dire
        che esiste un aperto \(W\) che contiene \((x,y)\) e tale che \(W
        \subseteq d^{-1}(-\infty,a) \). Sia \(\varepsilon = \frac{b-b_{0}}{3}\),
        consideriamo \(D(x, \varepsilon) \times D(y, \varepsilon)\) e prendiamo
        un punto \((x', y')\) in questo sottoinsieme. A questo punto \(d(x',
        y')\le d(x', x) + d(x, y') \le d(x',x) + d(x,y)+d(y,y') \le
        \frac{2}{3}(b-b_{0}) + b_{0} < b\) quindi è incluso in
        \(d^{-1}(-\infty,a)\) 
    \item[\(d^{-1}(a, \infty)\) ] Simile
\end{itemize}
    Poiché ogni insieme aperto di \(\mathbb{R}\) si può scrivere come
    unione di elementi ottenibili come intersezione finita di elementi di questa
    sottobase, la controimmagine di un aperto è aperta, quindi la distanza è
    continua.
\end{proof}

\begin{proposition}
    Le proiezioni \(\pi_X\) e \(\pi_Y\) sono aperte (notare che generalmente non
    sono chiuse) 
\end{proposition}
\begin{example}
    Sia \(I = \{xy=1\} \subseteq \mathbb{R}^2\). Mostrare che è chiusa ma che ha
    proiezione non chiusa.

    È chiusa perché controimmagine di \(1\) (chiiuso) rispetto a \(f: (x, y) \mapsto xy\)
    che è continua. Tuttavia \(\pi_X(I) = \mathbb{R} \smallsetminus \{0\} \)
    poiché \(\forall x \in \mathbb{R} \smallsetminus \{0\}, x \cdot \frac{1}{x}
    = 1\) ma 0 non ha inverso moltiplicativo.
\end{example}

\begin{definition}
    Su \((X, d) \times (Y, d')\) vogliamo aggiungere una metrica.
\begin{itemize}[label = --]
    \item \(d_\infty((x,y), (x', y')) = \max(d(x,x'), d'(y', y'))\) 
    \item \(d_{1}((x,y), (x', y')) = d(x, x') + d'(y, y')\) 
    \item \(d_{2}((x,y),(x', y')) = \sqrt{d{(x,x')}^2+d'{(y,y')}^2}\) 
\end{itemize}
Sono metriche equivalenti.
\end{definition}
\begin{proposition}
    \(\mathbb{R}^{n+1} - \mathbf{0} \approx S^{n}\times \mathbb{R}\)
    omeomorfismo
\end{proposition}
\begin{proof}
    Sia \(F: v \mapsto (\frac{v}{\|v\|}, \|v\|)\). È continua per la proprietà
    universale da \(\mathbb{R}^{n+1} - \mathbf{0}\) a \(\mathbb{R}^{n+1}\) e per l'altra
    proprietà anche nel nostro sottospazio. Inoltre ha inversa \((y, \rho)
    \mapsto \rho y\) continua in quanto prodotto di proiezioni (che sono
    continue).
\end{proof}

\begin{proposition}
    Sia \(C\) chiuso in \(X\) e \(D\) chiuso in \(Y\). Allora \(C \times D\) è
    chiuso in \(X\times Y\) 
\end{proposition}
\begin{proof}
    (Vedasi figura~\ref{fig:chiuso-prodotto}) \(X \times Y - (C \times D) = (X
    - C) \times Y \cup (X\times (Y - D))\) 
\end{proof}
\begin{figure}[ht]
    \centering
    \tiny{\incfig[.5]{chiuso-prodotto}}
    \caption{Prodotto di chiusi è chiuso}
    \label{fig:chiuso-prodotto}
\end{figure}
\begin{proposition}
    Sia \(D_{1} \subseteq X \) denso in \(X\)  e \(D_{2} \subseteq Y \) denso in
    \(Y\). Allora \(D_{1} \times D_{2}\) è denso in \(X \times Y\) 
\end{proposition}
\begin{proof}
    Dobbiamo dimostrare che \(\overline{D_{1} \times D_{2}} = X \times Y\),
    ossia \(\forall A \subseteq X \times Y \) aperto, \(A \cap (D_{1} \times
    D_{2}) \neq \varnothing\) 
    L'aperto \(A\) contiene un elemento \(A_{1} \times A_{2} \subseteq A \) tale
    che \(A_{1}\) è aperto di \(X\) e \(A_{2}\) è aperto di \(Y\) poiché tali
    aperti costituiscono una base della topologia prodotto.
    Ora esiste \(x \in A_{1} \cap D_{1} \) e esiste \(y \in A_{2} \cap D_{2}\)
    per densità. Ne consegue che \((x, y) \in {(A_{1} \times A_{2})} \cap (D_{1}
    \times D_{2}) \subseteq A \cap {(D_{1} \times D_{2})} \).
\end{proof}
\begin{example}
    Poiché \(\mathbb{Q}\) è denso in \(\mathbb{R}\), \(\mathbb{Q}^{n}\) è denso
    in \(\mathbb{R}^{n}\) 
\end{example}

Ora supponiamo di avere \(X\times X\) con la topologia prodotto rispetto a una
topologia su \(X\). Consideriamo lo spazio diagionale \(\Delta \subseteq X\times
X = \{(x, x) \forall x in X\} \) 
\begin{proposition}
    \(\Delta\) è omeomorfo a \(X\)
\end{proposition}
\begin{proof}
    Consideriamo la funzione \(X \to \Delta; x \mapsto (x, x)\) e la sua inversa
    \(\Delta \to X; (x, x) \mapsto x\). Sono entrambe continue per la continuità
    della proiezione, dell'identità e per la proprietà universale della
    topologia prodotto.
\end{proof}
Se invece si considera la diagonale di \(X\times X\) con due topologie diverse,
la diagonale ha poi una topologia che contiene entrambe, ed è la più piccola
topologia con questa proprietà.
\begin{example}
    Sia \(\mathbb{R}_e\) con l'euclidea e \(\mathbb{R}_C\) con la cofinita.
    Consideriamo ora \(\Delta_\mathbb{R}\subseteq \mathbb{R}_e \times \mathbb{R}_C \).
    Gli aperti della cofinita sono anche aperti nella topologia euclidea. Di
    conseguenza \(\Delta_ \mathbb{R}\) ha la topologia euclidea.
\end{example}
\begin{example}
    Consideriamo \(S_{1} \times \mathbb{R}\) (figura~\ref{fig:cilindro-infinito})
\end{example}
\begin{figure}[ht]
    \centering
    \incfig[.6]{cilindro-infinito}
    \caption{\(S_{1}\times \mathbb{R}\) proiettato su \(S_2\) meno i poli}
    \label{fig:cilindro-infinito}
\end{figure}
\begin{example}
Sia \(\mathbb{R}_{\aleph_0}\) con la topologia dove i chiusi sono \(\mathbb{R},
\varnothing\) e i sottoinsiemi numerabili o finiti di \(\mathbb{R}\). L'intersezione di
arbitrari insiemi numerabili è finita o numerabile, quindi ok. L'unione di due
numerabili deve essere ancora numerabile, lo è (prendendone uno e uno). Non è
confrontabile con la topologia euclidea, infatti \(\mathbb{Q}\) è chiuso per la
conumerabile, ma non per la euclidea, e l'intervallo \([0, 1]\) il contrario. La
topologia sulla diagonale di \(\mathbb{R}_{\aleph_0} \times \mathbb{R}_e\) è la
topologia dove ogni aperto è intersezione di aperti di \(\mathbb{R}_e\) e
\(\mathbb{R}_{\aleph_0}\) 
\end{example}



\subsection{Topologia Quoziente}
Per definire la topologia quoziente, consideriamo come costruire una
topologia su un insieme che è codominio di una funzione suriettiva.
\begin{definition}[Topologia Quoziente \-- Funzione Suriettiva]

Sia
\(f: X \twoheadrightarrow Z\) suriettiva. Dichiariamo su \(Z\) la topologia più
fine che rende \(f\) continua, ossia la topologia dove tutti e soli gli insiemi
tali che la loro controimmagine sia aperta, ossia
\[
    \tau_f = \{A \subseteq Z : f^{-1}(A) \text{ aperto di \(X\)  }\}
\]
\end{definition}
Verifichiamo che questa sia effettivamente una topologia:
\begin{enumerate}[label = \arabic*)]
    \item \(f^{-1}(\varnothing) = \varnothing\) aperto di \(X \implies
        \varnothing\) aperto di \(Z\) 
    
        \(f^{-1}(Z) = X\) aperto di \(X \implies Z\) aperto di \(Z\) 
    \item Supponiamo \(\{A_{i}\}_{i in I} \subseteq \tau_Z \implies
        f^{-1}(A_{i}) \text{ aperti di \(X\)} \implies\)

        \(\implies \bigcup_{i \in  I}
        f^{-1}(A_{i}) \text{ (aperto di \(X\))}= f^{-1}{\left( \bigcup_{i \in
        I} A_{i}\right)} \implies \bigcup_{i \in  I}A_{i} \) aperto di \(Z\) 
    \item \(A_{1}, A_{2}\) aperti di \(Z\) \(\implies f^{-1}(A_{1}) \cap
        f^{-1}(A_{2}) = f^{-1}(A_{1} \cap A_{2})\) è aperto di \(X\) perché
        intersezioni di aperti di \(X\), quindi \(A_{1}\cap A_{2} \in \tau_Z\) 
\end{enumerate}
\begin{remark}
    Inoltre \(C\) chiuso \(\iff f^{-1}(C)\) chiuso, infatti:
    \(C \text{ chiuso in \(Z\)} \iff Z-C \text{ aperto } \iff f^{-1}(Z - C)
    \text{ aperto in \(X\) } \iff X - f^{-1}(C) \text{ aperto in \(X\)  }\)
    ossia \(f^{-1}(C)\) chiuso in \(X\) 
\end{remark}
Tuttavia possiamo anche vedere la funzione suriettiva come una relazione di
equivalenza, infatti consideriamo la relazione \(\sim\) su \(X\) definita da
\[
    x \sim x' \iff f(x) = f(x')
\]
Vediamo che questa è una relazione di equivalenza:
\begin{itemize}[label = --]
    \item Chiaramente \(x \sim x\) perché \(f(x) = f(x)\) 
    \item \(x \sim x' \iff f(x) = f(x') \iff f(x') = f(x) \iff x' \sim x\) 
    \item Se \(x \sim x'\) e \(x' \sim x''\) allora \(f(x) = f(x') = f(x'')\) da
        cui \(f(x) = f(x'') \iff x \sim x''\) 
\end{itemize}
Viceversa se \(\sim\) è una relazione di equivalenza allora \(\pi : X \to
X/\sim\) è una funzione suriettiva, quindi possiamo definire la topologia
quoziente. In particolare possiamo quindi definire una topologia quoziente su un
insieme solo avendo una relazione di equivalenza.

\begin{definition}[Topologia Quoziente \-- Relazione di Equivalenza]
    Sia \(\sim\) una relazione di equivalenza su \(X\). Dichiariamo su \(X/\sim\)
    la topologia più fine che rende continua la proiezione \(\pi: X \to X/\sim\)
\end{definition}

\begin{definition}
    Un sottoinsieme \(Y \subseteq X \) si dice \textbf{bilanciato} rispetto a una
    funzione suriettiva \(f : X \twoheadrightarrow Z\) se
    \[
        \forall y \in Y, f^{-1}(f(y)) \subseteq Y 
    \]
\end{definition}
\begin{proposition}
    \(A \subseteq Z \) è aperto (chiuso) di \(Z\) se \(A\) è immagine di un
    aperto (chiuso) bilanciato di \(X\) 
\end{proposition}
\begin{proof}
    \(A\) aperto di \(Z \iff f^{-1}(A)\) aperto di \(X\), che è bilanciato per
    costruzione della topologia, infatti \(ff^{-1}(A) = A\) perché \(f\) è
    suriettiva.
\end{proof}

\begin{definition}
    \(y : X_{\tau_X} \to  Y_{\tau_Y}\) è chiamata \textbf{identificazione} se
    \(\tau_Y = \tau_y\) 
\end{definition}
\begin{example}
    Sia \(X\times Y\) e consideriamo lo spazio \(X = \pi_X(X\times Y)\) immagine
della proiezione (quindi suriettiva). Poiché \(\pi_X\) e \(\pi_Y\) sono aperte
    \(\tau_{X \times Y|_{\pi_X}} = \tau_X\) 
\end{example}
\begin{proposition}
    Siano \(f: X \to Y\) e \(g: Y \to Z\) suriettive, quindi \(h = gf\) è
    suriettiva. Adesso possiamo costruire su Y una topologia \(\tau_f\) e su
    \(Z\) la topologia \(\tau_h\), ma avendo construto una topologia su \(Y\)
    ora possiamo controllare se \(\tau_h = {(\tau_f)}_g\) ossia se \textbf{la
    composizione di identificazioni è una identificazione}
\end{proposition}
\begin{proof}
    Sia \(U \subseteq \mathbb{Z} \) aperto in \((\tau_f)_g\), quindi
    \(g^{-1}(U)\) è aperto in \(Y\), ossia \( g^{-1}(U) \in \tau_f\), ma quindi
    \(f^{-1}g^{-1}(U) = h^{-1}(U)\) è aperto in \(X\), ma allora \(U \in
    \tau_h\). Essendo tutte queste implicazioni dei se e solo se, vale anche il
    viceversa, quindi \(\tau_h = {\tau_f}_g\) 
\end{proof}
\begin{tikzcd}
    X_{\tau_X} \arrow[d, "f", two heads] \arrow[dr, "h"] & \\
    Y_{\tau_f} \arrow[r, "g"] & Z
\end{tikzcd}
\begin{proposition}[Proprietà universale]
    Dire che \(h = gf\) equivale a dire che è costante nella controimmagine di
    \(Y\). Vale anche il viceversa.
    
    La proprietà universale dice che \textbf{\(g\) continua \(\iff h\)
    continua}. Inoltre se \(h\) è aperta \(g\) è aperta e se \(h\) è chiusa,
    \(g\) è chiusa.
\end{proposition}
\begin{proposition}[Stessa, ma con il termine identificazione]
    Se \(f\) è continua ed è identificazione, le funzioni continue da \(Y\) a
    \(Z\) sono in corrispondenza biunivoca con le funzioni continue da \(X\) a
    \(Z\) che sono ottenute per composizione.
\end{proposition}
\begin{example}
    Supponiamo di avere \(X\) e un suo sottoinsieme \(S \subseteq X\).
    Consideriamo la relazione di equivalenza dove tutti gli elementi di \(S\)
    collassano in uno solo. Quindi \(x \sim x' \iff x = x'\) oppure \(x \in S,
    x' \in S\). Il risultato è che \(X / S= X - S \cup \{S\} \) 
\end{example}
\begin{example}
    Sia \(D_C = \{(x,y) \in \mathbb{R}^2:x^2+y^2\le 1\} \). Consideriamo il
    sottoinsieme \(\partial D_C = \overline{D} - \dot{D} \supseteq D_C \). È
    quindi ovviamente \(S^{1} = \{(x,y) \in \mathbb{R}^2 : x^2+y^2 = 1\} \). Ora
    consideriamo \(D_C / S^{1} = \dot{D}_C \cup \{S^{1}\} \). Intuitivamente è
    omeomorfa a \(S^{2}\). Per dimostrarlo devo trovare una funzione continua da
    \(D_C\) a \(S^{2}\) (vedasi figura~\ref{fig:esempio}). Dopo un po' di
    calcoli si trova che la funzione è:
    \[
        (x, y) \mapsto (x, y -\sqrt{1- x^2 - y^2}) \mapsto (\alpha x, \alpha y,
        -\sqrt{1 - x^2 - y^2})
    \]
    dove 
    \[
        \alpha = \sqrt{((\sqrt{1-x^2-y^2})(1-\sqrt{1-x^2-y^2}))}
    \]
    Notare che quando siamo sul bordo, ossia \(x^2+y^2=1\), allora \(\alpha =
    0\) quindi i punti collassano sullo stesso punto (0,0,0)
\end{example}
\begin{figure}[ht]
    \centering
    \incfig[0.8]{esempio}
    \caption{Esempio 6.11: Omeomorfismo tra \(D_C\) e \(S^{2}\) }\label{fig:esempio}
\end{figure}
\begin{example}
    Vari attaccamenti del quadratino \([0,1] \times [0,1]\) in modi diversi
    escono cilindro, Möbius, toro, piano proiettivo, otre di Klein
\end{example}
\begin{example}
    Consideriamo lo spazio \(\mathbb{R}^{n+1} - \mathbf{O}\) e impostiamo una
    relazione di equivalenza \(v\sim v' \iff \exists \lambda \in \mathbb{R} -
    \mathbf{O} : \lambda v = v'\). Facilmente verificabile che è una relazione
    di equivalenza perché partiziona in classi separate lo spazio, infatti solo
    l'origine apparterrebbe a tutte le classi. Denotiamo
    \[
        P_ \mathbb{R}^{n} = \frac{\mathbb{R}^{n+1} - \mathbf{O}}{\sim }
    \]
    (Stessa cosa si può ovviamente fare anche con \(\mathbb{C}\)). In
    \(\mathbb{R}\) dividiamo la
    relazione di equivalenza in due parti:
    \begin{align*}
        \sim ^{+} : & v \sim^{+} v' &\iff \exists \lambda \in \mathbb{R}^{+}:v =
        \lambda v' \\
        \sim^{\pm} : & v\sim ^{\pm}v' &\iff \mathbf{x} = -\mathbf{x}
    \end{align*}

    Possiamo quindi vedere \(P_ \mathbb{R}^{n}\) come \(\frac{\mathbb{R}^{n+1} -
    \mathbf{0}}{\sim^{+}} / \sim^{\pm} = S^{n} / \sim^{\pm}\) 

\end{example}

\begin{example}
    Sia \(\mathbb{Q} \subseteq \mathbb{R} \) e consideriamo due relazioni di
    equivalenza:
\begin{enumerate}[label = \arabic*)]
    \item \(x \sim^{1} y \iff x = y \text{ oppure } x, y \in  \mathbb{Q}\) 
    \item  \(x \sim^{2} y \iff x - y \in  \mathbb{Q}\) 
\end{enumerate}
La seconda (o forse la prima) non è ``buona'' perché non è ``localmente'' come
\(\mathbb{R}^{n}\), dove \textbf{localmente}, in genere vuol dire che ogni punto ha un
intorno di \(\dots\) 

\begin{enumerate}[label = \arabic*)]
    \item Nella relazione \(\sim^{1}\), l'insieme risultante è \((\mathbb{R} -
        \mathbb{Q}) \cup \{\mathbb{Q}\} \).

        Dato che ogni punto irrazionale ha come controimmagine solo sé stesso,
        tutti gli irrazionali sono chiusi, ma \(\pi ^{-1}(\{\mathbb{Q}\} ) =
        \mathbb{Q}\) che quindi non è chiuso. In particolare la sua chiusura
        deve essere tale che la controimmagine sia un chiuso di \(\mathbb{R}\)
        che contiene \(\mathbb{Q}\), quindi è al minimo \(\mathbb{R}\), ma se
        \(\pi ^{-1}(\overline{\{\mathbb{Q}\}}) = \mathbb{R}\) allora
        \(\overline{\{\mathbb{Q}\} } = \mathbb{R}\), per cui \(\{\mathbb{Q}\} \)
        è denso.
    \item Nella relazione \(\sim^{2}\) invece gli elementi di \(\mathbb{R} /
        \mathbb{Q}\) sono del tipo \(x + \mathbb{Q}\), tuttavia gli aperti di
        questo insieme devono immagine di aperti bilanciati, ma gli
        unici aperti bilanciati sono \(\varnothing, \mathbb{R}\) perché preso un
        intervallo aperto \((a, b)\), \((a, b) + \mathbb{Q} = \mathbb{R}\). 
        Di conseguenza la topologia su \(\mathbb{R} / \mathbb{Q}\)  è la
        concreta (indiscreta)
\end{enumerate}
\end{example}

\begin{example}
    Sia \(C_n = \{(x, y) \in \mathbb{R}^2 : x + ny = 1\} \) al variare di \(n
    \in \mathbb{Z}, n \ge 1\).
\begin{enumerate}[label = \arabic*.]
    \item Per quali valori di \(n\) \(\mathbb{R}^2 - ((C_n \cup C_{n+1}) \cap
        C_{n+2})\) è aperto?

        I vari \(C_n\) sono chiusi, perché controimmagine di \(1\) della
        funzione \(f_n: (x,y) \mapsto x + ny\). Quindi intersezione e unione
        finite rimangono chiusi, quindi il complementare è aperto.
    \item Sia \(E = \cup_n C_n\). Dire se è \(E\) è chiuso.
        
        Ad esempio il punto (0,0) è punto di accumulazione, ma non è
        nell'insieme, quindi \(E\) non è chiuso.
\end{enumerate}
\end{example}

\begin{example}
    Supponiamo di prendere in \(\mathbb{R}\) la topologia euclidea e la
    topologia \(\tau'\) strettamente più fine di quella euclidea. Dire se
    \(\mathbb{R}_{\tau'}\) e \(\mathbb{R}_e\) possono essere omeomorfi.
\end{example}
\begin{example}
    Sia \(c_n = \{(x, y )  \in  \mathbb{R}^2: x^2+y^2 = n^2\} \) e \(d_n = \{(x,
    y) \in  \mathbb{R}^2 : x^2 - y^2 = n^2\}\)
\begin{enumerate}[label = \arabic*.]
    \item Per quali \(n\) \(\mathbb{R}^2 - (c_n \cup d_n)\) è aperto?
    \item \(E = \bigcup_{n} d_n \cup \bigcup_{n} c_n\) è chiuso?
\end{enumerate}
\end{example}

\section{Assiomi di Separazione}
\begin{definition}[Assioma di Separazione \(T_1\) ]
    Uno spazio topologico \(X\) soddisfa l'assioma di separazione \(T_1\) se per
    ogni coppia di punti \(P, Q \in  X\), con \(P \neq Q\) esiste un aperto
    \(U\) tale che \(P \in  U\), \(Q \not\in  U\) 
\end{definition}
\begin{proposition}
    \(X\) è \(T_{1}\) se e solo se ogni punto di \(X\) è chiuso
\end{proposition}
\begin{proof}
    Se ogni punto di \(X\) è chiuso allora presi \(P, Q \in X\), \(P \neq Q\),
    \(U = X \smallsetminus \{Q\}  \) è aperto, inoltre \(P \in U, Q \not\in U\).

    Viceversa, se \(X\) è \(T_{1}\) allora mostriamo che \(Q \in X\) è chiuso.
    Prendiamo quindi, per ogni altro punto \(P \in X\), un aperto \(U_P\) tale
    che rispetti l'assioma di separazione, ossia \(P \in U_P, Q \not\in U_P\).
    Allora \(\bigcup_{P \in X} U_P = X \smallsetminus \{Q\} \) è aperto, ma
    quindi \(Q\) è chiuso.
\end{proof}

\begin{definition}[Assioma di Separazione \(T_2\) o di Hausdorff]
    Uno spazio topologico \(X\) soddisfa l'assioma di separazione \(T_2\) o di
    Hausdorff se per ogni coppia di punti \(P, Q \in  X\), con \(P \neq Q\)
    esistono due aperti disgiunti \(U, V\) tali che \(P \in  U\), \(Q \in  V\).
    Ossia
    \[
        \forall P, Q \in X, P \neq Q,\,\, \exists U \ni P, \,\exists V \ni Q : U \cap V = \varnothing
    \]
\end{definition}

\begin{example}
    Se \((X, d)\) è uno spazio metrico, allora è \(T_{2}\) 
\end{example}

\begin{example}[Spazio \(T_{1}\) non \(T_{2}\) ]
    Sia \(X\) un insieme infinito con la topologia cofinita. È \(T_{1}\) perché
    i punti sono chiusi, ma non è \(T_{2}\) perché due aperti si intersecano
    sempre.
\end{example}

\begin{proposition}
    Se \(X\) è \(T_{2}\) abbiamo unicità del limite di successioni.
\end{proposition}

\begin{proposition}
    Se \(X\) è \(T_{2}\) allora \(S \subseteq X \) è \(T_{2}\). Inoltre se anche
    \(Y\) è \(T_{2}\) allora \(X \times Y\) è \(T_{2}\) 
\end{proposition}

\begin{proposition}
    \(X\) è \(T_{2}\) se e solo se la diagonale \(\Delta = \{(x, x) \forall x
    \in X\} \subseteq X \times X \) è chiusa (con la topologia prodotto).
\end{proposition}

\begin{definition}[Assioma di Separazione \(T_3\) o Regolare]
    Uno spazio topologico \(X\) che è \(T_{1}\)  soddisfa l'assioma di
    separazione \(T_3\) e
    viene detto anche \emph{regolare} se per ogni chiuso \(C \subseteq X\) e per ogni punto \(P \in  X\)
    con \(P \not\in  C\) esistono due aperti disgiunti \(U, V\) tali che \(P \in
    U\), \(C \subseteq V\).
\end{definition}
\begin{definition}[Assioma di Separazione \(T_4\) o Normale]
    Uno spazio topologico \(X\) che è \(T_{1}\) soddisfa l'assioma di
    separazione \(T_{4}\) e viene detto anche \emph{normale} se dati \(C_{1}\) e \(C_{2}\) chiusi e disgiunti,
    esiste \(A_{1} \supseteq C_{1}  \) e \(A_{2} \supseteq C_{2} \) tali che
    \(A_{1} \cap A_{2} = \varnothing\) con \(A_{1}\) e \(A_{2}\) aperti.
\end{definition}

\begin{example}
    Uno spazio metrico \((X, d)\) è \(T_{3}\) e \(T_{4}\) 
\end{example}

\begin{example}[\(T_{2} \) non \( T_{3}\) ]
    Sia \(X\) con una topologia \(\tau\) che lo rende \(T_{2}\). Se prendo su
    \(X\) una topologia più fine rimane di Hausdorff, perché possiamo separare i
    punti con gli stessi aperti di prima. Non è detto però che se per \(\tau\)
    era \(T_{3}\) o \(T_{4}\) rimanga tale, perché aggiungiamo anche nuovi
    chiusi.

    Consideriamo \(\mathbb{R}_\text{S.Gius}\) con la topologia più fine che
    include sia la euclidea che la conumerabile. Per l'osservazione precedente questo spazio è \(T_{2}\). Ora
    consideriamo il chiuso (poiché numerabile) \(\mathbb{Q}\) e il punto
    \(\sqrt{2}\); chiaramente \(\sqrt{2} \not\in \mathbb{Q}\).
    Una base di aperti di \(\mathbb{R}_\text{S.Gius} \) è costituita dagli
    insiemi del tipo \((a, b) \smallsetminus S\), dove \(S\) è una successione
    al più numerabile, quindi ogni aperto \(A\) contenente il punto \(\sqrt{2}\)
    contiene un elemento della base del tipo
    \(A' := (\sqrt{2} - \varepsilon, \sqrt{2} + \varepsilon) \smallsetminus
    S_{\sqrt{2}}\), con
    \(\varepsilon \in \mathbb{R}\) e \(S_{\sqrt{2}}\) al più numerabile. Sia ora
    \(a\) un razionale tale che \(a \in (\sqrt{2} - \varepsilon, \sqrt{2} +
    \varepsilon)\). Possiamo essere certi dell'esistenza di un tale \(a\)  per la
    densità di \(\mathbb{Q}\) in \(\mathbb{R}\) secondo l'euclidea. A questo
    punto ogni aperto \(B\) contenente \(\mathbb{Q}\) contiene un elemento della base
    del tipo \(B' := (a - \delta, a + \delta) \smallsetminus S_a\), dove \(\delta \in
    \mathbb{R}\) e \(S\) è al più numerabile, perché \(B \ni a\). Considerando
    adesso \(B \cap A\) abbiamo che \(A' \cap  B' \subseteq A \cap B \), ma \(A'
    \cap B' = ((\sqrt{2} - \varepsilon, \sqrt{2} + \varepsilon) \cap (a -
    \delta, a + \delta)) \smallsetminus (S_{\sqrt{2}} \cup S_a )\). Ora notiamo
    che poiché \(a \in (\sqrt{2} - \varepsilon, \sqrt{2} + \varepsilon)\), \(a\)
    appartiene all'intersezione dei due intervalli, ma poiché gli intervalli
    sono aperti per l'euclidea, la loro intersezione è un aperto dell'euclidea e
    poiché è non vuota, contiene un elemento della base di aperti dell'euclidea,
    chiamiamo tale intervallo \((\alpha, \beta)\).
    Inoltre notare che \(S := S_{\sqrt{2}} \cup S_a\) è al più
    numerabile, perché unione di tali insiemi.
    Quindi \(A' \cap B' = (\alpha, \beta) \smallsetminus S\) è non vuoto e di
    conseguenza \(\mathbb{R}_\text{S.Gius} \) non è \(T_{3}\).
\end{example}

\begin{example}[\(T_{3}\) non \(T_{4}\) ]
    v. Sernesi pag. 96
\end{example}

\begin{lemmao}[Lemma di Uryson]
    Se \(X\) è \(T_{4}\) allora per ogni \(C_{1}\) e \(C_{2}\) chiusi e
    disgiunti esiste una funzione continua \(f: X \to \mathbb{R}\) tale
    che \(f(C_{1}) = 0\) e \(f(C_{2}) = 1\) 
\end{lemmao}

\section{Assiomi di Numerabilità}
\begin{definition}[Primo assioma di numerabilità]
    \(X\) soddisfa il primo assioma di numerabilità se 
    \[
        \forall P \in X \exists \{U_n\}_{n \in \mathbb{N}}
    \]
    dove \(\{U_n\}_{n \in \mathbb{N}} \) è un sistema di intorni numerabili tali
    che \(U_n+1 \subseteq U_n \) per ogni \(n > 1\) naturale e \(\forall  A\)
    aperto, con \(A \ni P\), \(\exists n \in \mathbb{N} : U_n \subseteq A \)

    Si dice allora anche che \(X\) è I-numerabile
\end{definition}

\begin{example}
    In uno spazio metrico, dato un punto \(P\), \(\{D(P, \frac{1}{n})\}_{n \in
    \mathbb{N}} \) è un sistema numerabile di intorni.
\end{example}

\begin{definition}[X è Separabile]
    \(X\) è detto \textbf{separabile} se
    \[
        \exists D \subseteq X, \text{ con \(D\) denso e numerabile }
    \]
\end{definition}
\begin{example}
    \(\mathbb{R}^{n} \supseteq \mathbb{Q}^{n} \) 
\end{example}

\begin{definition}[Secondo assioma di Numerabilità]
    \(X\) soddisfa il secondo assioma di numerabilità se ammette una base di
    aperti numerabile.

    Si dice allora anche che \(X\) è II-numerabile.
\end{definition}

\begin{remark}
    Se X è I(II)-numerabile, \(S \subseteq X \) è I(II)-numerabile. Questo segue
    banalmente dal fatto che un sistema di intorni di ogni punto \(P\) (una base
    di aperti) può essere trovata intersecando \(S\) con un sistema di intorni
    di \(P \in X\) (con una base di aperti di \(X\)).
\end{remark}
\begin{theorem}[Teorema di Uryson]
    Se \(X, \tau\) è \(T_{4}\) e soddisfa il secondo assioma di numerabilità allora è
    metrizzabile (quindi esiste una distanza \(d: X^2 \to  \mathbb{R}\)) e in
    particolare la topologia data dalla distanza \(d\) è la stessa di \(\tau\) 
\end{theorem}
\begin{definition}[Varietà Topologica]
    Una varietà topologica di dimensione \(n \in \mathbb{N}\) è uno spazio
    topologico \(M\),  \(T_{2} + \text{II-numerabile}\) ``localmente come
    \(\mathbb{R}^{n}\)'', ossia
    \[
        \forall P \in M\,\,\exists U \ni P \text{ aperto },\,\exists \varphi :
        U_P \to \mathbb{R}^{n}
    \]
    tale che \(\varphi(U)\) è un aperto di \(\mathbb{R}^{n}\) e \(\varphi\) è
    iniettiva. In
    altre parole, esiste per ogni punto \(P\) esiste un aperto che lo contiene
    omeomorfo ad un aperto di \(\mathbb{R}^{n}\).
\end{definition}

\begin{example}
    \(S^{n} = \{x \in \mathbb{R}^{n+1} : \|x\| = 1\} \) è una \(n-\)varietà. È
    infatti \(T_{2}\) e II-numerabile in quanto sottospazio di
    \(\mathbb{R}^{n+1}\).
    Inoltre l'aperto \(S^{n} \smallsetminus \{P\} \stackrel{\text{omeo}}{\approx }
    \mathbb{R}^{n}\), con \(P \in S^{n}\) un punto sulla sfera, per la
    proiezione stereografica, e il punto \(P\) ha un intorno aperto omeomorfo a un
    aperto di \(\mathbb{R}^{n}\) perché preso il suo punto antipodale \(Q\),
    l'aperto \(S^{n} \smallsetminus \{Q\} \stackrel{\text{omeo}}{\approx}
    \mathbb{R}^{n}\) per la proiezione stereografica in \(Q\) (notare che si può anche
    ragionare sulle piccole calotte omeomorfe ad un disco).
\end{example}
\begin{example}
    Se abbiamo \(M\) e \(N\) due varietà di dimensione rispettivamente \(n\) e
    \(m\) allora \(N \times M\) è una varietà di dimensione \(n+m\) 
\end{example}
Inoltre le \(1-\)varietà si chiamano \emph{curve} e le \(2-\)varietà si chiamano
\emph{superfici}

\begin{example}
    Prendiamo \(\mathbb{R}^{n}_{\tau_\text{aff} }\) con la topologia che ha come base di chiusi
    \(\varnothing\) e i sottospazi affini.
\begin{enumerate}[label = \arabic*.]
    \item[\(T_{1}\) ] Sì, perché i punti sono chiusi in quanto sottospazi
        affini.
    \item[\(T_{2}\) ] No, perché due aperti non vuoti si intersecano
        \textbf{sempre} (dimostrazione per induzione)
\end{enumerate}
A questo punto conideriamo \(\Delta \subseteq \mathbb{R}^{n}_{\tau_\text{aff} }
\times \mathbb{R}^{n}_{\tau_\text{aff}  }\). Se la topologia prodotto fosse
ancora quella degli affini allora la diagonale sarebbe chiusa, ma noi sappiamo
che non può essere (altrimenti sarebbe \(T_{2}\) ), quindi la dopologia prodotto è diversa.
\end{example}

\begin{example}
    Supponiamo di avere uno spazio \(X\) che sia \(T_{3}\) e prendiamo \(C\) un
    chiuso. Consideriamo la relazione di equivalenza \(x \sim y \iff x = y \lor
    x, y \in C\) (ossia identifichiamo \(C\) a un punto, il resto rimane
    invariato).
    Vogliamo mostrare che \(X / \sim \) è \(T_{2}\).
    Prima consideriamo se prendiamo come coppia di punti \(P \in X
    \smallsetminus C\) e \(\{C\} \). Allora dato che \(X\) è \(T_{3}\)
    consideriamo \(P\) e \(C\) ed esistono \(V\) e \(U\) aperti disgiunti tali che \(U \ni P, U
    \not\supseteq C \) e \(V \not\ni P, V \supseteq C\), ora se \(f\) è la
    funzione di identificazione \(f(V) \cap f(U) = \varnothing\) e \(f(V),
    f(U)\) sono aperti perché \(V\) e \(U\) sono aperti bilanciati.

    Con due punti la procedura è simile ma basta usare che \(X\) è \(T_{2}\) 
\end{example}

\begin{example}
    Prendiamo il quadratino identificato \([0, 1] \times [0, 1]\) e
    identifichiamo tutti i punti \((x, 0), 0 \le x < 1\). Questo spazio non è
    \(T_{1}\) perché il punto costituito dalla classe di equivalenza non è
    chiuso.
\end{example}

\begin{example}
    Consideriamo \(\mathbb{R}^{n}\) e diciamo che due vettori \(v \sim w \iff
\exists T \in \mathcal{L}(\mathbb{R}^{n}) : T(v) = w\), con \(T\) invertibile.
    Quindi abbiamo solo due classi di equivalenza, \(\{\mathbf{0}\} \) e
    \(\mathbb{R}^{n} \smallsetminus \{\mathbf{0}\} \), dove \(\{\mathbf{0}\}\) è chiuso
    e \(\{\mathbb{R}^{n} \smallsetminus \{\mathbf{0}\}\} \) è aperto, quindi
    questo spazio è \(T_{0}\) ma non \(T_{1}\), l'unico aperto che contiene il
    punto aperto è infatti tutto lo spazio, che contiene anche il punto chiuso.
\end{example}

\begin{example}
    Stesso di prima con le matrici ortogonali, l'insieme risultante è omeomorfo
    a \(\mathbb{R}^{+} = [0, +\infty)\) perché le trasformazioni ortogonali preservano la
    norma dei vettori. 
\end{example}



\section{Compattezza}
\begin{definition}[Ricoprimento, Sottoricoprimento]
    Sia \(X\) uno spazio topologico. Una famiglia di sottoinsiemi
    \(\{S_{i}\}_{i \in  I}, \, S_{i} \subseteq X\) tali che \(\bigcup_{i \in  I} S_{i} = X\) viene
    detto \textbf{ricoprimento} di \(X\).

    Se esiste un \(J \subseteq I\) tale che
    \(
        \bigcup_{j \in J} S_{j} = X
        \) allora la famiglia \(\{S_{j}\}_{j \in J}\) viene detto
        \textbf{sottoricoprimento}.

        Un ricoprimento è aperto (chiuso) se ogni \(S_{i}\) è
        aperto (chiuso).

        Un ricoprimento è finito se \(I\) è finito.
\end{definition}
\begin{example}
    Considerato \(X = [0, 1]\), \(\{[0, 0.5], [0.5, 1]\} \) è un ricoprimento
    finito chiuso; \(\{[0, 0.5), (0.3, 7), (0.5, 1]\} \) è un ricoprimento finito
    aperto; un ricoprimento non finito di \(X\) è
    \(\{(0.9, 1]\} \cup \{[0, b)\}_{b \in [0, 1]}\) (che è un ricoprimento aperto)
\end{example}

\begin{definition}[Compattezza (per ricoprimenti)]
    \(X, \tau\) è \textbf{compatto} per ricoprimenti se \emph{per ogni
    ricoprimento aperto di \(X\) esiste un sottoricoprimento finito}. In altre
    parole
    \[
        \forall \{S_i\}_{i \in I} \subseteq \tau : \bigcup_{i \in  I} S_{i} =
        X,\,\,\exists J \subseteq I, \#J \in \mathbb{N}: \bigcup_{j \in J} S_j =
        X
    \]
\end{definition}
\begin{example}[Spazi compatti] \(\) 
\begin{itemize}[label = --]
    \item Se \(X\) è finito, la topologia è finita, quindi ogni ricoprimento è
        finito, quindi \(X\) è compatto.
    \item Similmente se \(\tau\) è finito \(X\) è compatto
    \item \(X\) con la cofinita. Se \(X = \bigcup_{i in I} A_{i}\) con \(A_{i}\)
        aperti, allora \(X \smallsetminus A_{1} = \{x_{1}, \dots, x_k\} \)
        perché è un chiuso. Ma poiché \(\{A_{i}\} \) è un ricoprimento, per ogni
        \(x_{j}\) esiste un \(A_{i_j}\) che lo contiene. Quindi \(X = A_{1} \cup
        \bigcup_{j \in 1\dots k} A_{i_j}\), per cui \(X\) è compatto.
\end{itemize}
\end{example}
\begin{example}[Spazi non compatti] \(\) 
\begin{itemize}[label = --]
    \item \(\mathbb{R}\). Sia \(A_{i} = (i - 1, i + 1), i \in \mathbb{Z}\).
        Chiaramente \(\bigcup_{i \in  \mathbb{Z}} A_{i} = \mathbb{R}\) ma
        rimuovendo ogni particolare aperto \((i - 1, i + 1)\) il punto \(i\) non
        è più nell'unione, per cui non esiste nessun sottoricoprimento di
        \(\{A_{i}\} \).
    \item \(\mathbb{R}^{n}\). Si considerino i dischi centrati in \(\mathbf{x} \in
        \mathbb{R}^{n}\). Chiaramente \(\bigcup_{r \in
        \mathbb{R}^{+}}D(\mathbf{x}, r) = \mathbb{R}^{n} \) ma se \(F \subseteq
        \mathbb{R}^{+}\) è finito allora \(\bigcup_{r \in F} D(\mathbf{x}, r) =
        D(\mathbf{x}, \max F) \neq \mathbb{R}^{n}\) 
    \item \([0, 1) = \bigcup_{n \in \mathbb{Z}^{+}}[0, 1 - \frac{1}{n})\) ma di
        nuovo se \(F \subseteq \mathbb{Z}^{+} \) è finito allora \(\bigcup_{n
        \in F} [0, 1 - \frac{1}{n}) = [0, 1 - \frac{1}{\max F}) \neq [0, 1)\) 
\end{itemize}
\begin{example} \(\mathbf{[0, 1]}\) \textbf{è compatto}.
\end{example}
\begin{proof}
    Sia \(\{A_{i}\}_{i \in  I}\), con \(A_{i}\) aperti in \([0, 1]\) e
    \(\bigcup_{i \in  I} A_{i} = [0, 1]\) un ricoprimento aperto di \([0, 1]\).
    Sia \[C = \{x \in [0, 1] : \exists i_{1}, \dots, i_k \in I, \, \bigcup_{j
    \in 1\dots k} A_{i_j} \supseteq [0, x] \} \]
    Procediamo col dimostrare \(1 \in C\).
\begin{enumerate}[label = \alph*.]
    \item \(C\) è non vuoto.

        Deve esistere un \(A_{i_0} \ni 0\), per cui
        \(\exists \varepsilon : [0, \varepsilon) \subseteq A_{i_0}\). Ora per
        ogni \(\varepsilon' < \varepsilon, \varepsilon' \in C\) poiché \([0,
        \varepsilon'] \subseteq [0, \varepsilon) \). In particolare \(\exists 0
        < \varepsilon'\in C\) 
    \item \((0, 1] \ni c := \sup C \in C\).

        \(\exists A_{i_c} \ni c \implies \exists \varepsilon : (c - \varepsilon,
        c] \subseteq A_{i_c} \). Sia ora \(\gamma \in (c - \varepsilon,
        \varepsilon)\). Poiché \(\gamma < c\), esiste un sottoricoprimento
        finito di \([0, \gamma]\). Aggiungendo \(A_{i_c}\) a tale
        sottoricoprimento, otteniamo un sottoricoprimento finito di \([0, c]\),
        per cui \(c \in C\). 
    \item c = 1.

        Se \(c < 1\) allora \(\exists \varepsilon > 0: [c, c + \varepsilon)
        \subseteq A_{i_c} \). Ma allora se \(\sigma \in (0, \varepsilon)\)
        troviamo che esiste un sottoricoprimento finito di \([0, \sigma]\) (lo
        stesso di \([0, c]\) costruito nel passo precedente), quindi \(\sigma
        \in C\) ma \(\sigma > c\) che è assurdo.
\end{enumerate}
\end{proof}
\end{example}
\begin{remark}
    Si può anche ragionare sui chiusi. Se esiste un sottoricoprimento finito con
    unione \(X\) allora, ragionando ai complementari, esiste un insieme finito
    di chiusi con intersezione vuota.
    Infatti se \(\{{A_{i}}\}_{i \in  I}\) è un ricoprimento aperto di \(X\)
    compatto significa che esiste un sottoricoprimento finito. Esiste quindi \(F
    \subseteq I \) finito tale che \(\bigcup_{i \in  F} A_{i} = X\). Ma allora:
    \[
        \varnothing = X \smallsetminus \left(\bigcup_{i \in F}A_{i} \right) = \bigcap_{i
        \in F} \left(X \smallsetminus A_{i}\right)
    \]
    Ossia esiste un numero finito di chiusi \(C_{i} := X \smallsetminus A_{i}\)
    con intersezione vuota. Notare che vale anche il viceversa.
\end{remark}
\begin{proposition}
    Sia \(X\) compatto, \(f: X \twoheadrightarrow Y\) continua e suriettiva.
    Allora \(Y\) è compatto. In altre parole \textbf{l'immagine continua di
    compatti è compatta}
\end{proposition}
\begin{proof}
    Sia \(\{A_{i}\}_{i \in I}\) un ricoprimento aperto di \(Y\). Allora
    \(f^{-1}(A_{i}) =: B_{i}\) aperti di \(X\) e \(\bigcup_{i \in  I} B_{i} =
    f^{-1}(Y) = X\). Ma allora per compattezza di \(X\) esiste un
    sottoricoprimento di \(\{B_{i}\} \), indicizzato da \(F \subseteq I \)
    finito. Poiché \(f(B_{i}) = A_{i}\) per suriettività, 
    \[
        \bigcup_{i \in F} A_{i} = \bigcup_{i \in F} f\left( B_{i} \right) =
        f\left( \bigcup_{i \in F} B_{i} \right) = f\left( X \right) = Y 
    \]
    Per cui \(Y\) è compatto
\end{proof}
\begin{proposition}
    \(X\) compatto, \(S \subseteq X \) chiuso, allora \(S\) è compatto (con la
    topologia indotta).
\end{proposition}
\begin{proof}
    Sia \(\{A_{i}\}_{i \in I}\) un ricoprimento aperto di \(S\). Ogni \(A_{i}\)
    è quindi \(A_{i} = A_{i}' \cap S\), con \(A_{i}' \subseteq X \) un aperto di
    \(X\). Ora considerato l'aperto \(A' := X \smallsetminus S\) abbiamo che
    \(A' \cup \bigcup_{i \in  I} A'_i = X\), per cui per compattezza esiste un
    sottoricoprimento finito, quindi \(\exists F \subseteq I \) finito tale che
    \(A' \cup \bigcup_{i \in F} A'_i = X\). Ma allora \(\bigcup_{i \in F} A'_i
    \supseteq S \implies \bigcup_{i \in F} A_{i} = S \) 
\end{proof}
\begin{remark}
    Il viceversa generalmente non vale, infatti i punti sono sempre compatti ma
    alcuni spazi non sono \(T_{1}\). Un altro esempio è \(\mathbb{N}\) con la
    cofinita, dove \(\{2, 4, \dots\} \) è compatto (perché la topologia indotta
    è la cofinita) ma non chiuso, perché non è finito.
\end{remark}
\begin{theorem}
    Se \(X\) è \(T_{2}\) e \(K \subseteq X \) è compatto, allora \(K\) è chiuso.
\end{theorem}
\begin{proof}
    Mostreremo che se \(X \ni x \not\in K\), allora esiste un aperto di \(X\)
    contenente \(x\) e disgiunto da \(K\), ossia
    \begin{equation}
        \forall X \ni x \not\in K\,\,\exists \text{ aperto }A_x \ni x, A_x \cap  K
        = \varnothing
    \end{equation}
    L'idea è che se questa osservazione è corretta allora possiamo prendere
    l'unione di tutti gli aperti trovati da ogni punto di \(X\smallsetminus K\)
    e quindi avere che
    \[
        \bigcup_{x \in  X \smallsetminus K} A_x = X \smallsetminus K \text{
        aperto } \implies K \text{ chiuso }
    \]
    Per dimostrare \((1)\), sia quindi \(x \not\in K, y \in K\), ovviamente
    quindi \(x \neq y\). Poiché \(X\) è \(T_{2}\), \(\exists A_{x, y} \ni x, B_y
    \ni y, A_{x, y} \cap B_y = \varnothing\). Ora \(\bigcup_{y \in K} B_y
    \supseteq K \), quindi se \(B'_y := B_y \cap K\), \(\bigcup_{y \in K} B'_y =
    K\) e ogni \(B'_y\) è un aperto di \(K\). Abbiamo quindi che \(\{B'_y\} \) è
    un ricoprimento aperto di \(K\), per compattezza quindi esiste un
    sottoricoprimento finito, sia quindi \(F \subseteq K \) finito tale che
    \(\bigcup_{y \in F} B'_y = K\), ma quindi
    \[
        \bigcup_{y \in F} B_y \supseteq K \implies \bigcap_{y \in F} A_{x, y}
        \cap K = \varnothing
    \]
    Ma poiché \(F\) è finito, \(A_x := \bigcap_{y \in F} A_{x, y}\) è un aperto
    di \(X\), per cui verifica la tesi (1).
\end{proof}
\begin{corollary}
    Se \(X\) è \(T_{2}\) e compatto, allora i sottoinsiemi compatti di \(X\)
    sono esattamente i chiusi.
\end{corollary}
\begin{proof}
    Segue direttamente da 9.5 e 9.4
\end{proof}
La stessa dimostrazione di \(9.5\) mostra però anche un'altra cosa, infatti se
\(X\) è compatto e \(T_{2}\) siamo nell'ipotesi del corollario, quindi
notiamo che scelto un chiuso \(K\) e un punto \(x \not\in K\), si possono
separare con aperti, poiché \(K\) è compatto, in altre parole \(T_{2}\) +
compatto \(= T_{3}\). Ma il divertimento non finisce qui, infatti
\begin{proposition}
    Se \(X\) è uno spazio \(T_{2}\) e compatto, allora \(X\) è \(T_{4}\) 
\end{proposition}
\begin{proof}
    Per quanto visto prima, \(X\) è \(T_{3}\), userò questa ipotesi più forte
    nella dimostrazione.
    Presi due chiusi disgiunti \(C_{1}\) e \(C_{2}\), sono compatti per il corollario
    9.5.1.
    Prendiamo ora un punto \(x \in C_{1}\). Poiché \(X\) è \(T_{3}\) esistono
    due aperti \(A_{1, x}\) e \(A_{2, x}\) tali che \(A_{1, x} \ni x, A_{2, x}
    \supseteq C_{2} \) e \(A_{1,x} \cap A_{2, x} = \varnothing\). Procedendo in
    maniera simile alla dimostrazione di \(9.5\), ponendo \(A'_{1, x} = A_{1, x}
    \cap C_{1}\) otteniamo che \(\{A'_{1, x}\}_{x \in C_{1}} \) è un
    ricoprimento aperto di \(C_{1}\), che essendo compatto ammette quindi un
    sottoricoprimento finito. Sia quindi \(F \subseteq C_{1} \) finito tale che
    \(\bigcup_{x \in F} A'_{1, x} = C_{1}\) e quindi
    \(A_1 := \bigcup_{x \in F} A_{1, x} \supseteq C_{1} \). Consideriamo adesso
    \(A_{2} := \bigcap_{x \in F} A_{2, x}\). Chiaramente \(A_{2} \supseteq C_{2}
    \) poiché ogni sua componente lo contiene. Inoltre \(A_{2}\) è aperto in
    quanto intersezione finita di aperti. Infine
    \[
        A_{1} \cap A_{2} = \left( \bigcup_{x \in F} A_{1, x} \right) \cap A_{2}
        = \bigcup_{x \in F} \left( A_{1, x} \cap A_{2} \right) \subseteq
        \bigcup_{x \in F} \left( A_{1, x} \cap A_{2, x} \right) = \varnothing
    \]
    Quindi \(A_{1}\) e \(A_{2}\) sono aperti che separano \(C_{1}\) e \(C_{2}\)
    e verificano la condizione di normalità (\(T_{4}\)).
\end{proof}
\begin{proposition}[Heine-Borel debole]
    I compatti di \(\mathbb{R}\) sono tutti e soli i chiusi limitati. Ossia
    \(C\) chiuso e \(C \subseteq [a, b] \) con \(a, b \in \mathbb{R}\) 
\end{proposition}
\begin{proof}\( \)
\begin{itemize}
    \item[\(\implies \)] Sia \(C \subseteq [a, b] \) chiuso, allora \(C\) è
        compatto (chiuso in un compatto)
    \item[\(\impliedby \)] Sia \(C\) un compatto, è un chiuso per 9.5. Consideriamo il ricoprimento
        aperto \({(-a, a) \cap C}_{a \in \mathbb{R}^{+}}\). Essendo \(C\)
        compatto, esiste un \(F \subseteq \mathbb{R}^{+} \) finito tale che
        \(\{(-a, a)\}_{a \in F}\) sia un sottoricoprimento finito. Ma quindi se
        \(M := \max F\) abbiamo che \((-M, M) \supseteq C \), e quindi \([-M-1,
        M+1] \supseteq C \) 
\end{itemize}
\end{proof}
\begin{proposition}
    Sia \(X\) compatto, \(f: X \to \mathbb{R}\) continua, allora \(f\) ha un
    minimo e un massimo.
\end{proposition}
\begin{proof}
    \(f(X) \subseteq \mathbb{R} \) è un compatto di \(\mathbb{R}\) perché
    immagine continua di compatto. Ma allora per Heine-Borel debole, \(f(X)
    \subseteq [a, b] \) ed è chiuso, quindi \(\sup f(X) \in f(X) \implies
    \exists x \in X: f(x) = \sup f(X) = \max f(X)\) 

    Similmente si dimostra l'esistenza del minimo.
\end{proof}


\begin{proposition}
    Sia \(X\) compatto, \(f: X \to Y\) continua e suriettiva, \(Y\) è \(T_{2}\),
    allora \(f\) è \textbf{chiusa}
\end{proposition}
\begin{proof}
    Sia \(C \subseteq X \) un chiuso, quindi è compatto perché chiuso di un
    compatto. \(f(C) \subseteq  Y\) è quindi un compatto, ma poiché \(Y\) è di Haussdorf
    \(f(C)\) è anche un chiuso.
\end{proof}
\begin{lemmao}
    Supponiamo di avere \(X_\tau\) e \(\mathcal{B} \subseteq \tau \) una base della
    topologia. Allora \(X\) è compatto se e solo se per ogni ricoprimento
    \(\bigcup_{\ell \in  L} B_\ell = X\), con \(B_{i} \in \mathcal{B}\) ne possiamo
    estrarre un sottoricoprimento finito.
\end{lemmao}
\begin{proof}\( \)
\begin{itemize}
    \item[\(\implies \)] Ovvio, se \(X\) è compatto posso estrarre un
        sottoricoprimento finito da qualsiasi ricoprimento, incluso se
        costituiti da aperti di una base.
    \item[\(\impliedby \)] Supponiamo di poter estrarre sottoricoprimenti finiti
        da ricoprimenti aperti con elementi in \(\mathcal{B}\). Sia \(X =
        \bigcup_{i \in  I} A_{i} = \bigcup_{i \in  I} \left( \bigcup_{j \in
        J_{i}} B_{i, j} \right) \), dove l'ultimo è un ricoprimento con elementi
        di \(\mathcal{B}\). Ma quindi esiste un insieme finito \(F\) di coppie
        \((i, j)\) tali che \(\bigcup_{(i, j) \in F} B_{i, j} = X\). A questo
        punto per ogni elemento \(B_{i, j}\) prendiamo \(A_i \supseteq B_{i,
        j}\) e ne otteniamo dunque comunque un numero finito.
\end{itemize}
\end{proof}

\begin{theorem}[Teorema debole di Tichonov]
    Siano \(X\) e \(Y\) spazi compatti, allora \(X \times Y\) è compatto. In
    altre parole \textbf{il prodotto di due compatti è compatto}.
\end{theorem}
\begin{proof}
    Per il Lemma 9.10 è sufficiente considerare dei ricoprimenti \(X \times Y =
    \bigcup_{i \in  I} A_{i} \times B_{i}\) dove \(A_{i}\) sono aperti di \(X\)
    e \(B_{i}\) sono aperti di \(Y\), perché questa è una base della topologia
    su \(X \times Y\).

    Chiamiamo \(\pi_X: X \times  Y \to  X\) e \(\pi_Y: X \times Y \to  Y\) le
    proiezioni a \(X\) e \(Y\).

    Ora fissiamo \(\overline{x} \in  X\) un punto e consideriamo \(\overline{x} \times Y =
    \pi_X^{-1}(\overline{x})\) che è compatto perché è omeomorfo a \(Y\).
    Consideriamo \(\bigcup_{i \in  I} \left( A_{i} \times B_{i} \cap
    (\overline{x}\times Y) \right) = \overline{x} \times Y\) perché abbiamo
    intersecato il ricoprimento di \(X \times Y\) con un suo sottospazio.
    Inoltre è un ricoprimento aperto perché è composto da aperti di
    \(\overline{x} \times Y\). Per compattezza quindi esiste un
    sottoricoprimento finito, indicizzato da \(F_{\overline{x}} \subseteq I \) finito tale che
    \(\bigcup_{i \in F_{\overline{x}}} \left( A_{i} \times B_{i} \cap
    (\overline{x} \times Y) \right) = \overline{x} \times Y\), il che significa
    che \(\bigcup_{i \in F_{\overline{x}}} A_{i} \times B_{i} \supseteq
    \overline{x} \times Y\).

    Considero ora \(A_{\overline{x}} = \bigcap_{i \in
    F_{\overline{x}}} A_i \) che è un aperto non vuoto poiché ovviamente
    \(\overline{x} \in A_{\overline{x}} \). Dico ora che \(A_{\overline{x}}
    \times Y \subseteq \bigcup_{i \in F_{\overline{x}}} (A_{i} \times B_{i}) \).
    Ma poiché \(\{B_{i}\}_{i \in F_{\overline{x}}} \) ricopre \(Y\),
    \[A_{\overline{x}} \times Y \subseteq A_{\overline{x}} \times \bigcup_{i \in
        F_{\overline{x}}} B_{i} = \bigcup_{i \in F_{\overline{x}}}
    A_{\overline{x}} \times B_{i} \] Ma ogni \(A_{\overline{x}} \times B_{i}
    \subseteq A_{i} \times B_{i} \), da cui l'inclusione richiesta.

    Adesso ovviamente \(\bigcup_{x \in X} A_x = X \) è un ricoprimento aperto di
    \(X\), quindi essendo \(X\) compatto esiste un \(F \subseteq X \) finito
    tale che \(\bigcup_{x \in F}A_x = X \). Quindi \(\bigcup_{x \in F} A_x
    \times Y = X \times Y\). Infine, poiché ogni \(A_x \times Y\) si può
    ricoprire con finiti aperti indicizzati da \(F_x \subseteq I \),
    \[
        X \times Y = \bigcup_{x \in F} \bigcup_{i \in F_x} A_{i} \times B_{i}
    \]
\end{proof}
\begin{corollary}
    Il prodotto di finiti compatti è compatto
\end{corollary}
\begin{proof}
    Per induzione sul numero di compatti, il passo base è il teorema 9.11.
\end{proof}

\begin{theorem}[Teorema di Heine-Borel]
    Sia \(C \subseteq \mathbb{R}^{n}\). Allora \(C\) è compatto se e solo se è
    chiuso e limitato, ossia \(\exists \rho > 0 : C \subseteq D(0, \rho) \) 
\end{theorem}
\begin{proof}\( \)
\begin{itemize}
    \item[\(\implies \)] Sia \(C\) compatto, quindi è chiuso perché
        \(\mathbb{R}^{n}\) è \(T_{2}\). Quindi Consideriamo \(\bigcup_{r \in
        \mathbb{R}^{+}} D(0, r) \cap C = C \) è un ricoprimento aperto di \(C\),
        quindi esiste un sottoricoprimento indicizzato da \(F \subseteq
        \mathbb{R}^{+} \) finito. Poniamo \(\rho := \max F\) e otteniamo che
        \(D(0, \rho) \cap C = C\) da cui \(D(0, \rho) \supseteq C  \) 
    \item[\(\impliedby \)] \(C\) è chiuso e limitato, quindi è contenuto in un
        ``disco'' \(D_n = [-N, N]^{n} \supseteq C \) per un qualche \(N \in
        \mathbb{R}^{+}\), ma poiché \(D_n \stackrel{\text{omeo}}{\approx} [0, 1]^{n}
        \) che è compatto, allora \(C\) è un chiuso in un compatto, per cui è
        compatto.
\end{itemize}
\end{proof}
\begin{example}
    Dire quando \(C_n = \{(x, y) \in \mathbb{R}^{2} : x^2 - ny^2 = 1\} \).

    \(C_n\) è sempre chiuso perché è controimmagine di \(1\) che è chiuso
    rispetto a \(f: \mathbb{R}^{2} \to \mathbb{R}; (x, y) \mapsto x^2 - ny^2\)
    che è continua.
\end{example}


\begin{example}
    Sia \(C_n = \{ (x, y) : nx^2 + (n+2)y^2 = n+1\} \), con \(n \in \mathbb{Z}\) 
    Per quali valori di \(n\),
\begin{enumerate}[label = \arabic*.]
    \item \(C_n \cup C_{-n}\) è chiuso

        Sempre, l'unione di finiti chiusi è chiusa, ogni \(C_n\) è chiuso
        perché controimmagine continua di chiuso \(n+1\) 
    \item \(C_n\) è compatto

        Poiché siamo in \(\mathbb{R}^{2}\), i compatti sono i chiusi limitati,
        essendo \(C_n\) chiuso, dobbiamo solo controllare quando è limitato. Se
        \(n \ge 1\) oppure \(n \le -3\), \(C_n\) è un ellisse ed è quindi limitato.
        Se \(n = 0\) si ottengono due rette, se \(n = -1\) si ottiene \(x^2 =
        y^2 \) che è le due rette bisettrici del piano. Se \(n = -2\) abbiamo due
        rette parallele.

    \item \(\bigcup_{n \ge 1} C_n =: K\) è compatto

        Consideriamo la successione di punti che è contenuta all'interno di
        \(K\) data da tutti i punti con \(y=0\) tali punti soddisfano \(nx^2 =
        n+1\) quindi consideriamo la successione \(x_n = \sqrt{\frac{n+1}{n}}
        \to 1\). Per ogni \(n \ge 1, (x_{n}, 0) \in K \)  ma \((1, 0) \not\in
        K\) infatti significherebbe che \(n = n+1\) che è assurdo. Non essendo
        chiuso non è quindi compatto.
    \item \(\mathbb{R}^2 - K\) è denso?
\end{enumerate}
\end{example}
\begin{example}
    Sia \(C_n = \{(x, y) \in \mathbb{R}^2 : 2 n x + 2 (n+1) y = 3n\} \). Sia \(K
    = \bigcup_{n \in \mathbb{N}} C_n\). Sia \(S^{1} = \{(x, y) : x^2+y^2 = 1\}
    \) e \(D^{+} = \{(x, y) : x \ge 0, y \ge 0\} \).

    Per quali valori di \(n\):
\begin{enumerate}[label = \arabic*.]
    \item \(C_n \cup C_{n+1}\) è chiuso?

        Sempre ovvio
    \item \(C_n \cap D^{+}\) è compatto?

        Bisogna solo controllare se è limitato, essendo chiuso in quanto
        intersezione di due chiusi.
        Poiché per \(n\neq 0, y = 0\) otteniamo \(x = \frac{3}{2}\) significa
        che tutte le rette passano per il punto \((\frac{3}{2}, 0)\). Quindi per
        vedere se l'insieme è limitato basta controllare se, quando \(x=0\), \(y
        > 0\). Otteniamo quindi \(y = \frac{3n}{2(n+1)}\) oppure \(n = -1\). La
        seconda tuttavia comporta \(x = \frac{3}{2}\) che è non limitato.
        Risolvendo quindi \(\frac{3n}{2(n+1)} > 0\) otteniamo \(n > 0 \lor n <
        -1\) da cui \(n \ge 1 \lor n \le -2 \) 
    \item \(K \cap S^{1}\) è compatto?

        Vogliamo risolvere il sistema, quindi consideriamo \(x = \frac{3}{2} -
        \frac{n+1}{n} y\), e la circonferenza, quindi \(\left(\frac{3}{2} -
        \frac{n+1}{n} y \right)^2 + y^2 = 1\). Da cui \(y^2 \left( 1 +
    \frac{n+1}{n}^2 \right) - 3 \frac{n+1}{n} y + \frac{5}{4}\).
        Da cui il discriminante \(\Delta = 9 \frac{n+1}{n}^2 - 5 \left( 1 +
        \frac{n+1}{n}^2 \right) = 4 \frac{n+1}{n}^2 - 5 \ge 0\) da cui l'ultima
        diseguaglianza perché vogliamo controllare se il numero di punti è
        finito. Troviamo \(\frac{n+1}{n} \ge \sqrt{\frac{5}{4}}\) che
        effettivamente ha un numero finito di soluzioni. Quindi \(K \cap S^{1}\)
        è compatto.
\end{enumerate}
\end{example}

\begin{example}
    Sia \(C_n = \{(x, y) \in \mathbb{R}^2 : (x^2+y^2-1)(x^{n} + y^{n} - 1) = 0\} \) 

    Per quali valori di \(n\), \(C_n\) è compatto? Chiaramente \(C_n\) è chiuso,
    ma in particolare è l'unione delle controimmagini di \(0\) delle funzioni
    \(f: (x, y) \mapsto x^2 + y^2 - 1\) e \(g: (x, y) \mapsto x^{n} + y^{n} -
    1\) Chiamando \(S^{1} = f^{-1}(0) \) e \(D_n = g^{-1}(0)\) abbiamo \(C_n =
    S^{1} \cup D_n\). \(S^{1}\) è sempre compatto, mentre \(D_n\) è compatto se
    \(n\) è pari. Da questo troviamo che \(n\) deve essere pari.

    Sia ora \(K= \bigcup_{n \in \mathbb{N}} D_n\). Non è compatto perché non è
    chiuso. Infatti il punto \((1,1) \not\in K\) ma se \(x = y\) si ottiene \(2
    x^{2k} = 1\) che tende a \(x = 1\) 
\end{example}

\begin{example}
    Sia \(M_n(\mathbb{R}) \stackrel{\text{omeo}}{\approx} \mathbb{R}^{n^2}\) lo spazio delle matrici \(n \times n\) a
    coefficienti in \(\mathbb{R}\). Si considerino \(GL(n)\) il gruppo lineare,
    \(SL(n)\) il gruppo speciale, \(O(n)\) le matrici ortogonali.

    \(GL(n)\) è un aperto, perché è controimmagine di \(\mathbb{R}
    \smallsetminus \{0\} \) che è aperto di \(\mathbb{R}\) rispetto al
    determinante che è continuo. \(SL(n)\) è la
    controimmagine di \(1\) rispetto al determinante, quindi è chiuso.
    Infine \(O(n) = (A \mapsto ^{t}A A )^{-1}(I)\) quindi è chiuso. Ma inoltre
    dato che una matrice ortogonale ha colonne composte da vettori di norma
    unitaria, \(d(A, \mathbf{0}) = \sqrt{n}\) da cui \(O(n)\) è chiuso e
    limitato, quindi compatto.
\end{example}

\begin{definition}[Gruppo topologico]
    Un gruppo topologico \(G\) è un gruppo \(G\) con topologia \(\tau\) tale che
    l'operazione di gruppo \(G \times G \to G\) è continua e l'operazione di
    inversa \(G \to G\) è continua.
\end{definition}

\subsection{Compattezza in Spazi Metrici}
\begin{definition}[Compattezza per successioni]
    Uno spazio topologico \((X, \tau)\) è \textbf{compatto per successioni} se
    da ogni successione ne possiamo estrarre una sottosuccessione convergente
    \[
        \forall \{x_{n}\}_{n \in \mathbb{N}} \exists C \subseteq \mathbb{N} :
        \{x_n\}_{n \in C} \text{ converge } 
    \]
\end{definition}
\begin{lemmao}[Numero di Lebesgue]\label{lebesgue_number}
    Supponiamo di avere un ricoprimento aperto di \(X\) \(\mathcal{U} = \{A_{i}\}_{i \in I}\).
    Supponiamo che \(X\) sia compatto per successioni. Allora esiste un
    \(\sigma_{\mathcal{U}}\) tale che \(\forall x \in X\) \(\exists i \in I:
    D(x, \sigma_{\mathcal{U}}) \subseteq A_{i} \) 
\end{lemmao}
\begin{proof}
    Procediamo per assurdo, allora negando la tesi otteniamo che esiste un
    ricoprimento aperto \(\mathcal{U} = \{A_{i}\}_{i \in I} \) 
    \[
        \forall \sigma \in \mathbb{R}^{+} \,\,\exists x_\sigma \in X : D(x_\sigma,
        \sigma)\text{ non è contenuto in nessun \(A_{i}\)  }
    \]
    E ancora ``\(D(x_\sigma, \sigma)\) non è contenuto in nessun \(A_{i}\)'' quindi
    \[
        \forall i \in I : x_\sigma \in A_{i} \implies \exists y_{i, \sigma}
        \not\in A_{i} : d(x_\sigma, y_{i, \sigma}) < \sigma
    \]
    Ora consideriamo la successione \(x_{n}\) degli \(x_\sigma\) tali che
    \(\sigma = \frac{1}{n}\). Poiché \(X\) è compatto per successioni esiste una
    sottosuccessione convergente a \(z\). Allora \(\exists i \in I:
    z\in A_{i}\) e da un certo punto in poi della sottosuccessione
    tutti i valori \(x_{n_k}\), per \(k > \overline{k}\) sono contenuti in
    \(A_{i}\). Ma per ogni \(n_k\), abbiamo che \(y_{i, n_k} \not\in A_{i}\) e
    \(d(x_{n_k}, y_{i, n_k}) < \frac{1}{n_k}\).
\begin{remark}
Abbiamo \(\{x_{n}\} \to z\) e \(\lim_{n}d(x_{n}, y_{n}) = 0\). Allora \(y_{n}
\to z\) 
\begin{proof}
    \(0 \le d(y_{n}, z) \le d(y_{n}, x_{n}) + d(x_{n}, z)\) Per il squeeze
    theorem anche \(d(y_{n}, z) \to 0\) 
\end{proof}
\end{remark}
    Per l'osservazione allora \(y_{n_k} \to z\) che è assurdo perché di sicuro
    esiste un \(r\) tale che \(D(z, r) \subseteq A_{i} \) e \(y_{n_k} \not\in
    A_{i}\), quindi di sicuro \(d(y_{n_k}, z)\ge r\).
\end{proof}
\begin{definition}
    Sia \((X, d)\) uno spazio metrico. Si dice \textbf{totalmente limitato} se
    \(\forall \mathbb{R} \ni \sigma >0 \) allora \(\exists F \subseteq X \)
    finito tale che \(\bigcup_{x \in F} D(x, \sigma) = X\) ossia è possibile
    ricoprirlo con un numero finito di dischi di un raggio fissato.
\end{definition}
    Notare che la totale limitatezza non implica la compattezza (ad esempio [0, 1)). È più evidente
    però che se \(X\) è compatto allora è totalmente limitato. Basta infatti
    prendere come ricoprimento tutti i dischi di raggio \(\varepsilon\) e poi
    per la compattezza sceglierne un sottoricoprimento finito.
    Ma questo risultato (e sarà più ovvio il perché in seguito al teorema enunciato
    tra poco) è valido anche in caso di compattezza per successioni
    \begin{lemmao}\label{sequentially_compact_then_totally_bounded}
    Sia \((X, d)\) compatto per successioni, allora è totalmente limitato
\end{lemmao}
    \begin{proof}
Supponiamo che \(X\) sia non totalmente
    limitato, ossia \(\exists \sigma > 0\) tale che \(X\) non possa essere
    ricoperto da un numero finito di dischi di raggio \(\sigma\). In particolare
    scelto un \(x_{1} \in X\) tale che \(D(x_{1}, \sigma) \neq X\), scegliamo
    allora \(x_{2} \in X \smallsetminus D(x, \sigma)\) e di nuovo l'unione dei
    due dischi centrati in \(x_{1}\) e \(x_{2}\) non può essere \(X\),
    procediamo così costruendo. Inoltre abbiamo che \(d(x_{i}, x_{j}) > \sigma\)
    per ogni \(i, j \in \mathbb{N}\) poiché ogni punto successivo è fuori da
    ogni altro disco. Tuttavia se la distanza non è mai minore di \(\sigma\) non
    può esistere una sottosuccessione convergente.
    Leggendo l'implicazione al contrario otteniamo che se \(X\) è compatto per
    successioni allora \(X\) è totalmente limitato.
    \end{proof}
\begin{theorem}
    Se \((X, d)\) è uno spazio metrico, e consideriamo \(\tau_d\) la topologia
    data dalla distanza, allora \(X\) è compatto per ricoprimenti se e solo se è
    compatto per successioni.
\end{theorem}
\begin{proof}\( \text{compatto} \iff \text{compatto per successioni}\) 
\begin{itemize}
    \item[\(\implies \)] Supponiamo \(X\) compatto. Prendiamo una successione
        \(\{x_{n}\}_{n \in \mathbb{N}}\) e sia \(Y = \bigcup_{n \in \mathbb{N}}
        \{x_{n}\} \).  Supponiamo non esista una
        sottosuccessione convergente, allora non esistono punti di
        accumulazione, quindi per ogni \(x \in X\) esiste un aperto \(A_x \ni
        x\) tale che \(\{n: x_n \in A_x\} \) è finito. Ma allora \(\mathcal{U} = \{A_x
        : x \in X\} \) è un ricoprimento aperto di \(X\) e per la compattezza
        esiste un sottoricoprimento finito, indicizzato da \(F \subseteq X \)
        finito. Ma questo significa che \(\bigcup_{x \in F} \{n: x_n \in A_x\}
        = \mathbb{N}\) è finito, che è assurdo.
    \item[\(\impliedby \)] 
    Sia \(X\) compatto per successioni e sia \(\mathcal{U} = \{A_{i}\}_{i \in
    I}\) un ricoprimento aperto di \(X\). Sia \(\sigma\) il numero di Lebesgue
    per il ricoprimento \(\mathcal{U}\), che esiste per il
    Lemma~\ref{lebesgue_number}. Ora per il
    Lemma~\ref{sequentially_compact_then_totally_bounded} sappiamo che \(X\) è
    totalmente limitato, quindi esiste un ricoprimento finito composto da dischi
    di raggio \(\sigma\), di cui ognuno è contenuto in un aperto del
    ricoprimento \(\mathcal{U}\), formando un sottoricoprimento finito di
    \(\mathcal{U}\).
\end{itemize}
\end{proof}
\begin{definition}
    Una successione \(\{x_{n}\}_{n \in \mathbb{N}} \subseteq X \) si dice di
    Cauchy se 
    \[
        \forall \varepsilon \in \mathbb{R}_{> 0}\,\,\exists
        \overline{n}(\varepsilon) : d(x_{n}, x_{m}) < \varepsilon \forall n, m >
        \overline{n}
    \]
\end{definition}
\begin{proposition}
    Se \(x_{n} \to x\) allora \(x_{n}\) è di Cauchy.
\end{proposition}
\begin{proof}
    Scelto \(\frac{\varepsilon}{2}\), esiste \(\overline{n}\) tale che \(\forall
    n > \overline{n}\), \(d(x_{n}, x) < \frac{\varepsilon}{2}\). Allora per la
    diseguaglianza triangolare \(d(x_{n}, x_{m}) \le d(x_{n}, x) + d(x, x_{m}) <
    \varepsilon\) se \(n, m \ge \overline{n}\) 
\end{proof}
Notare che generalmente il viceversa non è detto. Ad esempio in (0, 1) la
successione \(\frac{1}{n}\) non è convergente (convergerebbe a \(0 \not\in X\))
ma è di Cauchy
\begin{definition}
    Uno spazio metrico \((X, d)\) si dice \textbf{completo} se ogni successione
    di Cauchy è convergente.
\end{definition}
\begin{proposition}
    Sia \(X, d\) uno spazio metrico.
    \(X\) compatto, allora \(X\) è completo
\end{proposition}
\begin{proof}
    Sia \(\{x_{n}\} \) una successione di Cauchy. Essendo \(X\) compatto ammette
    una sottosuccessione convergente \(x_{n_k} \to x\). Vogliamo mostrare che
    \(x_{n} \to x\). \(\exists \overline{k}: \forall k > \overline{k}\),
    \(d(x_{n_k}, x) < \frac{\varepsilon}{2}\). Inoltre possiamo anche trovare un
    \(\overline{n}\) tale che \(\forall n, m > \overline{n}\), \(d(x_{n}, x_{m})
    < \frac{\varepsilon}{2}\). Supponiamo che \(n_k = m\) per qualche \(k\), ma
    allora \(d(x_{m}, x) \le d(x_{m}, x_{m_k}) + d(x_{m_k}, x) < \varepsilon\) 
\end{proof}
\begin{proposition}
    Se in \((X, d)\) i chiusi limitati sono compatti, allora \((X, d)\) è
    completo. In particolare \(\mathbb{R}^{n}, d\) è completo.
\end{proposition}
\begin{proof}
    Prendiamo un punto \(z \in X\) e \(x_{n}\) una successione di Cauchy. Allora
    scelto \(\varepsilon\) esiste un \(\overline{n}\) tale che \(n, m >
    \overline{n} \implies d(x_{n}, x_{m}) < \varepsilon\). Allora \(d(x, x_{n})
    \le d(x, x_{\overline{n}}) + d(x_{n}, x_{m}) < \rho\) fissato. Allora
    aggiungendo i primi finiti termini minori di \(\overline{n}\) otteniamo che
    \(\{x_{n}\} \) è limitata.
\end{proof}
\begin{proposition}
    Totalmente Limitato + Completo \(\iff\) Compatto
\end{proposition}
\begin{proof}\( \)
\begin{itemize}
    \item[\(\implies \)] Sia \(X\) completo e totalmente limitato. Sia
        \(\{x_{n}\}\) una successione. Vogliamo mostrare che esiste \(x_{n_k}\)
        di Cauchy, quindi convergente per completezza, in modo da garantire la
        compattezza per successioni e di conseguenza la compattezza.

        Siano \(D_n\) i centri dei ricoprimenti composti da dischi di raggio
        \(\frac{1}{n}\), di cui conosciamo l'esistenza per la totale
        limitatezza. \(D_{1}\) è finito, quindi esiste un \(y_{1} \in D_{1}\)
        tale che \(A_{1} := \{n: x_{n} \in D(y_{0}, 1)\} \) è infinito. Adesso sul
        sottospazio \(D(y_{0}, 1)\) esiste un \(y_{2} \in D_{2} : A_{2} := \{n:
        x_{n} \in D(y_{1}, 1) \cap D(y_{2}, \frac{1}{2})\} \) è infinito.
        Continuando in questo modo, \[\exists y_k \in D_k : A_k := \{n : x_{n} \in
        \bigcap_{m \in 1\dots k} D\left( y_m, \frac{1}{m} \right)  \} \] 
        Ora per l'assioma della scelta estraiamo una successione di \(n_k\) tale
        che ogni \(n_k \in A_k\). Infine, per ogni \(\varepsilon > 0\), scelti
        \(\mathbb{N} \ni k_{1}, k_{2} > \frac{2}{\varepsilon}\), allora se \(N =
        \left\lceil \frac{2}{\varepsilon} \right\rceil \), quindi \(\varepsilon
        \ge \frac{2}{N}\).

        Per definizione di \(A_k\) e poiché \(k_{1}, k_{2} \ge N\), 
        \(x_{n_{k_{1}}}, x_{n_{k_{2}}} \in D\left( y_{n_N}, \frac{1}{N} \right) \).
        Infine
        \[d(x_{n_{k_{1}}}, x_{n_{k_{2}}}) \le d(x_{n_{k_{1}}}, y_{n_N}) +
            d(y_{n_N},
        x_{n_{k_{2}}}) < \frac{1}{N} + \frac{1}{N} = \frac{2}{N} \le \varepsilon\] 
        Per cui \(x_{n_k}\) è di Cauchy.
        
    \item[\(\impliedby \)] Compatto \(\implies \) compatto per successioni
        \(\implies \) totalmente limitato, per teorema e lemma enunciati
        dimostrati in precedenza. Inoltre essendo compatto è completo.
\end{itemize}
\end{proof}
\begin{proposition}
    Sia \((X, d)\) completo, e sia \(Y \subseteq X \) allora \(Y\) è completo
    \(\iff Y\) è chiuso
\end{proposition}
\begin{proof}\( \)
\begin{itemize}
    \item[\(\implies \)] Sia \(Y\) completo. Allora se \(y_{n} \in Y\) è una
        successione convergente a \(z \in X\). Ma allora \(y_{n}\) è di Cauchy e
        per la completezza di \(Y\) \(y_{n} \to \overline{z} \in Y\). Per
        l'unicità del limite \(z = \overline{z}\) \(Y\) è chiuso.
    \item[\(\impliedby \)] Sia \(Y\) chiuso. 
\end{itemize}
\end{proof}

\begin{definition}
    Un insieme non vuoto \(A \subseteq X \) è detto \textbf{magro} se \(X \smallsetminus A\) è denso in \(X\). In
    altre parole \(A\) è \textbf{magro} se non ha parte interna.
\end{definition}

\begin{theorem}[Baire]
    Sia \((X, d)\) uno spazio metrico completo.
Supponiamo di avere una famiglia \(\{A_n\}_{n \in  \mathbb{N}}\) una famiglia
numerabile di aperti densi in \(X\). Allora 
\[
    \bigcap_{n \in \mathbb{N}}A_n \text{ è denso }
\]
Similmente se \(\{C_n\}_{n \in \mathbb{N}} = \{X \smallsetminus A_n\}_{n \in
\mathbb{N}}  \) è una famiglia numerabile di chiusi magri, allora 
\[
    \bigcup_{n \in \mathbb{N}} C_n \text{ è magro }
\]
\end{theorem}
Notare l'importanza dell'ipotesi di completezza: ad esempio se \(X =
\mathbb{Q}\) allora ogni insieme del tipo \(\mathbb{Q} \smallsetminus \{q_n\} =
A_n\) è denso in \(\mathbb{Q}\) ma l'intersezione \(\bigcap_{n \in \mathbb{N}}
A_n = \varnothing\).
\begin{proof}
    \begin{lemma}[``Palla mobile'']
    Sia \((X, d)\) uno spazio completo. 
    Siano \(D(x, r) = \{y \in X : d(x, y)< r\} \) e \(D^{C}(x, r) = \{y \in X :
    d(x, y) \le r\} \). Allora ogni \(D\) è aperto perché controimmagine di
    aperto e ogni \(D^{C}\) è chiuso perché controimmagine di chiuso.
    Inoltre chiaramente \(D^{C}(x, \frac{r}{2}) \subseteq D(x, r) \subseteq
    D^{C}(x, r) \). Supponiamo quindi di avere una successione di \(x_{i} \in
    X\), \(r_{i} \in \mathbb{R}_{> 0}\), con \(r_{i} < r_{i - 1}\) e \(r_{i}\to
    0\). Inoltre supponiamo che \(D^{C}(x_{n}, r_{n}) \subseteq D(x_{n-1},
    r_{n-1}) \) e denotiamo \(D_n := D(x_{n}, r_n)\) e \(D^{C}_n := D^{C}(x_n,
    r_n)\).
    \[
        \implies \bigcap D_i = \{y\}, \text{ con } x_i \to y \in X
    \]
\begin{figure}[ht]
    \centering
    \incfig[.4]{palla-mobile}
    \caption{palla mobile}
    \label{fig:palla-mobile}
\end{figure}
\end{lemma}
\begin{proof}
    Iniziamo mostrando che \(x_{n}\) è di Cauchy. Infatti poiché \(r_n \to 0 \)
    per ogni \(\varepsilon \in \mathbb{R}\) esiste un \(\overline{n}\) tale che
    se \(n \ge \overline{n}\) allora \(0 < r_n < \frac{\varepsilon}{2}\). Ma allora presi
    \(n, m > \overline{n}\) abbiamo che \(x_{n}, x_{m} \in D_{\overline{n}}\) ma
    allora \(d(x_{n}, x_{m}) < d(x_{n}, x_{\overline{n}}) + d(x_{m},
    x_{\overline{n}}) < \varepsilon\). Quindi \(x_{n}\) è di Cauchy e per la
    completezza dello spazio \(X\) è convergente.

    Allora \(x_{n} \to y\). Inoltre \(y \in \bigcap_{n \in \mathbb{N}}
    D^{C}_n\), poiché ogni \(D^{C}\) è completo e da un punto in poi ogni
    termine di \(x_{n}\) è contenuto in ogni \(D^{C}_n\).
    Però per definizione dei \(D\) e \(D^{C}\) e della successione abbiamo che
    \(D_n \subseteq D_n^{C} \subseteq D_{n-1}  \) da cui \(\bigcap D_n =
    \bigcap D_n^{C}  \) 
\end{proof}
Supponiamo di avere \(A_{1}\) e \(A_{2}\) densi aperti in \(X\). Allora \(A_{1}
\cap A_{2}\) è aperto denso. Infatti se \(U\) è aperto in \(X\) allora \(A_{1}
\cap U \neq \varnothing\) è aperto allora \(A_{2} \cap (A_{1} \cap U) \neq
\varnothing\) da cui la tesi per associatività dell'intersezione (notare che
non abbiamo usato che \(A_{2}\) è aperto, quindi tecnicamente basterebbe che
\(A_{2}\) sia solo denso). Per induzione otteniamo che anche \(B_n = A_{1} \cap
\dots A_n\) è denso aperto se ogni \(A_{i}\) è aperto denso. Infatti se sappiamo che
\(B_{n-1}\) è denso aperto allora \(B_n = B_{n-1} \cap A_n\) che quindi è denso
aperto.

Sia ora una successione di aperti densi \(\{A_n\}_{n \in \mathbb{N}} \) e siano
\(B_{i} = \bigcap_{i = 1\dots n} A_n\). Quindi abbbiamo che \(B_i \supseteq B_{i
+ 1} \) per ogni naturale \(i\). Allora
\[
    D = \bigcap_{n \in \mathbb{N}} A_n = \bigcap_{n \in \mathbb{N}} B_n
\]
Poiché ogni \(B_i\) è intersezione di \(A_i\). Ora vogliamo mostrare che \(D\) è denso.

A questo scopo sia \(x_{1} \in B_{1} \cap U \neq \varnothing\) che esiste per la
densità di \(B_{1}\). Sia ora \(D_{1}^{C} = D^{C}(x_{1}, r_{1}) \subseteq B_{1}
\cap U\) che si può fare perché \(B_{1} \cap U\) è aperto; e inoltre sia \(0 <
r_{1} < 1\) 
Sia ora \(\varnothing \neq D_{1} \cap B_{2}\) che è aperto di \(X\). Quindi ora
sia \(D_{2}^{C} = D^{C}(x_{2}, r_{2}) \subseteq D_{1} \cap B_{2} \subseteq D_{1}
\cap B_{1} \subseteq U \) e sia \(r_{2} < \frac{1}{2}\). Continuando in questo
modo fino a \(x_{n}\) in modo che ogni \(r_{i+1} < r_{i} < \frac{1}{i}\) e
\(D^{C}_i = D^{C}(x_{i}, r_{i}) \subseteq D_{i - 1} \cap B_{i} \). Per procedere
sia l'aperto \(D_n \cap B_{n+1} \neq \varnothing\), quindi sia \(x_{n+1} \in D_n \cap
B_{n+1}\) e poiché è aperto allora esiste un \(r_{n+1} < r_n\) e \(r_{n+1} <
\frac{1}{n+1}\) tale che \(D_{n+1}^{C} = D^{C}(x_{n+1}, r_{n+1}) \subseteq D_n
\cap B_{n+1} \).
A questo punto abbiamo una famiglia di dischi \(D_n\) che soddisfa le ipotesi
del lemma della palla mobile, quindi esiste un \(y \in \bigcap D_n \subseteq U
\) ma anche \(y \in \bigcap D_n \subseteq \bigcap B_n \) da cui \(y \in U \cap \bigcap B_n\) 
\end{proof}

\begin{example}
    Sia \(K\) un compatto e \((X, d)\) uno spazio metrico. Sia \(\mathcal{C}(K,
    X) = \{f : K \to X \text{ continue }\} \). 

    Date \(f, g \in \mathcal{C}(K, X)\) sia \(\delta(f, g) = \sup_{x \in K}
    d(f(x), g(x)) = \max_{x \in K} d(f(x), g(x))\). Mostriamo che \((\mathcal{C(K, X), \delta})\) è uno spazio
    metrico. 
    \[
        K \stackrel{F = (f, g)}{\longrightarrow } X \times X \stackrel{d}{\to } \mathbb{R}
    \]
    \[
        x \stackrel{F}{ \mapsto } (f(x), g(x)) \stackrel{d}{ \mapsto } d(f(x), g(x))
    \]
    Ha un massimo perché è immagine continua del compatto \(K\). Mostriamo che è
    una metrica:
\begin{itemize}[label = --]
    \item \(\delta(f, g) \ge 0\) vero perché sup di distanze 
    \item \(\delta(f,g) = 0 \iff\) \(f(x) = g(x) \forall x \in K\) 
    \item \(\delta(f, g) = \delta(g,f)\) è un estremo superiore sullo stesso
        insieme di valori reali perché ogni elemento del sup ha la proprietà
        simmetrica
    \item Disuguaglianza triangolare. Siano \(f, g, h : K \to X\) continue.
        Allora \(\delta(f, g) = d(f(\overline{x}), g(\overline{x})) \le
        d(f\overline{x}, h\overline{x}) + d(h\overline{x}, g\overline{x}) \le
        \delta(f, h) + \delta(h, g)\) perché ovviamente preso un qualsiasi \(x
        \in K\), \(d(fx, hx) \le \sup_{x \in K}d(fx, hx) = \delta(f, h)\) 
\end{itemize}
 
\end{example}



\end{document}

